\title{The $PCP^{-1}$ Decomposition}
\subtitle{\SubTitleName}
\institute[]{\Course}
\author{\Instructor}
\maketitle   

\frame{\frametitle{Topics and Objectives}
\Emph{Topics} \\
%\TopicStatement
\begin{itemize}

    % \item complex numbers: addition, multiplication, complex conjugate

    \item factorizing $2\times 2$ real matrices that have complex eigenvalues as $A = PCP^{-1}$
    
    % \item Eigenvalue theorems

\end{itemize}

\vspace{0.5cm}

\Emph{Learning Objectives}\\

%\LearningObjectiveStatement

\begin{itemize}

    % \item add and subtract complex numbers
    % \item conjugate complex numbers
    % \item apply properties of complex numbers to determine whether mathematical statements that involve them are true or false
    
    \item construct the $PCP^{-1}$ decomposition for $2\times2$ matrices that have complex eigenvalues
    % \item Use eigenvalues to determine identify the rotation and dilation of a linear transform. 
    % \item Apply theorems to characterize matrices with complex eigenvalues.

\end{itemize}

% \vspace{0.25cm} 

% \Emph{Motivating Question}

% %Recall the rotation matrix (for $\pi/4$):
% %\[  A = \frac{1}{\sqrt{2}}  \spalignmat{ 1  -1 ; 1 1 }\]
%     What are the eigenvalues of a rotation matrix? 

} 







\begin{frame}{The $ PCP^{-1}$ Decomposition}

    \begin{center}\begin{tikzpicture} \node [mybox](box){\begin{minipage}{0.9\textwidth}\vspace{4pt}

    If $A$ is a real $2 \times 2$ matrix with eigenvalue $\lambda = a - bi$ (where $b \neq 0$) and associated eigenvector $\vec v$, then we may construct the decomposition
    
    \[ A = PCP^{-1} \]
    where
    \[ P = (\textrm{Re}\, \vec v\ \ \  \textrm{Im}\, \vec v) \quad \text{and} \quad C = \spalignmat{ a -b ; b  a }. \]
    
    \end{minipage}};
    \node[fancytitle, right=10pt] at (box.north west) {Theorem};
    \end{tikzpicture}\end{center}
    
    \vspace{4pt}


\end{frame}




\begin{frame}{The $ PCP^{-1}$ Decomposition} 

    \begin{itemize} 
        \item<2-> $C$ is referred to as a \Emph{rotation dilation} matrix, because it is the composition of a rotation by $\phi$ and dilation by $r$
        \item<3-> the proof for why we can write $A = PCP^{-1}$ and why the columns of $P$ are linearly independent is too long for a video
        \item<4-> the $A=PCP^{-1}$ decomposition allows us to compute large powers of $A$ efficiently
    \end{itemize}

\end{frame}




\begin{frame}{Example} 

    If possible, construct matrices $P$ and $C$ such that $AP = PC$. The eigenvalues of $A$ are given. 
    
    \begin{align*}  
        A =  \spalignmat{ 1  -2 ; 1  3 } , \quad \lambda = 2 \pm i
    \end{align*}

\end{frame}



\frame{\frametitle{Summary}

    \SummaryLine \vspace{4pt}
    \begin{itemize}\setlength{\itemsep}{8pt}

    \item factorizing $2\times 2$ matrices that have complex eigenvalues as $A = PCP^{-1}$
    
    \end{itemize}
    
    \vspace{8pt}

}

\frame{\frametitle{}


    
}