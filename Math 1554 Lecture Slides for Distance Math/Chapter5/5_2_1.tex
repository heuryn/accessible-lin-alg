\title{The Characteristic Polynomial}
\subtitle{\SubTitleName}
\institute[]{\Course}
\author{\Instructor}
\maketitle   



\frame{\frametitle{Topics and Objectives}
\Emph{Topics} \\
\TopicStatement
\begin{itemize}

    \item the characteristic polynomial of a matrix

    % \item Algebraic and geometric multiplicity of eigenvalues

    % \item Similar matrices

\end{itemize}

\vspace{0.5cm}

\Emph{Objectives}\\

\LearningObjectiveStatement

\begin{itemize}

    \item construct the characteristic polynomial of a matrix and use it to identify eigenvalues and their multiplicities
    \item apply the trace and determinant of a $2\times2$ matrix to determine eigenvalues
    % \item characterize the long-term behaviour of dynamical systems using eigenvalues

\end{itemize}

\vspace{0.25cm} 

}




\begin{frame}{The Characteristic Polynomial}

    \Emph{Recall:}
	\vspace{-2pt}
    % ~~ ~~ Highlight Box ~~ ~~
    \begin{center}\begin{tikzpicture} \node [mybox](box){\begin{minipage}{0.7\textwidth}\vspace{2pt}

        \begin{center}
           $\lambda$ is an eigenvalue of $A \ \Leftrightarrow \ (A-\lambda I)$ is not invertible
        \end{center}

    \end{minipage}};
    \end{tikzpicture}\end{center}
    % ~~ ~~ Highlight Box ~~ ~~
    

Therefore, to calculate the eigenvalues of $A$, we can solve
\[ 
\det(A-\lambda I)= 0
\]

\pause

\begin{itemize}
    \item $\det(A-\lambda I)$ is the \Emph{characteristic polynomial} of $A$
    \item $\det(A-\lambda I)=0$ is the  \Emph{characteristic equation} of $A$
    \item the roots of the characteristic polynomial are the eigenvalues of $A$
\end{itemize}



\end{frame}

\begin{frame}{Example: Calculating the Eigenvalues of a $2\times2$ Matrix}

The characteristic polynomial of $A = \spalignmat{ 4 2 ; 2 1} $ is: 

\vspace{2cm}

So the eigenvalues of $A$ are:

\end{frame}




\begin{frame}{Characteristic Polynomial of $2 \times 2$ Matrices}

    The \Emph{trace} of a matrix is the sum of its diagonal elements. 
    
    \vspace{12pt}
    
    \Emph{Example} \\ Express the characteristic equation of
    \begin{align*}
       M = \spalignmat{ a b ; c d }
    \end{align*}
    
    in terms of its determinant and trace. 

\end{frame}





\begin{frame}{Using the Trace to Identify Eigenvalues}

    Although the characteristic polynomial can always be used to determine eigenvalues, sometimes we can identify eigenvalues by inspection.
    
    \vspace{12pt}
    
    \Emph{Example} \\ By inspection, what are the eigenvalues of $A$? 
    $$A = \spalignmat{6 18;3 9}$$

    

\end{frame}




\begin{frame}\frametitle{Numerical Notes}

    \begin{itemize}
        
        \item The eigenvalues of any matrix larger than $2\times2$ should be found using a computer, unless the matrix has a special structure.
        
        \item Software for computing computing eigenvalues tends to avoid the characteristic polynomial.
        
        \item Nevertheless, the characteristic polynomial is important for theoretical purposes. 
    
    \end{itemize}
\end{frame}



\frame{\frametitle{Summary}

    \SummaryLine \vspace{4pt}
    \begin{itemize}\setlength{\itemsep}{8pt}

        \item the characteristic polynomial of a matrix and its use to calculate eigenvalues

    \end{itemize}
    \vspace{8pt}

}










