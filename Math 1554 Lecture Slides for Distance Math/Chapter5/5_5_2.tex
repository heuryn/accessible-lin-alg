\title{Complex Roots of the Characteristic Polynomial}
\subtitle{\SubTitleName}
\institute[]{\Course}
\author{\Instructor}
\maketitle   

\frame{\frametitle{Topics and Objectives}
\Emph{Topics} \\
%\TopicStatement
\begin{itemize}

    \item complex eigenvalues and eigenvectors
    
    % \item complex numbers: addition, multiplication, complex conjugate

    % \item Diagonalizing matrices with complex eigenvalues
    
    % \item Eigenvalue theorems

\end{itemize}

\vspace{0.5cm}

\Emph{Learning Objectives}\\

%\LearningObjectiveStatement

\begin{itemize}

    % \item add and subtract complex numbers
    % \item conjugate complex numbers
    % \item apply properties of complex numbers to determine whether mathematical statements that involve them are true or false
    
    % \item Diagonalize $2\times2$ matrices that have complex eigenvalues.
    % \item Use eigenvalues to determine identify the rotation and dilation of a linear transform. 
    
    \item identify eigenvalues and eigenvectors of matrices with complex eigenvalues

\end{itemize}

% \vspace{0.25cm} 

% \Emph{Motivating Question}

% %Recall the rotation matrix (for $\pi/4$):
% %\[  A = \frac{1}{\sqrt{2}}  \spalignmat{ 1  -1 ; 1 1 }\]
%     What are the eigenvalues of a rotation matrix? 

} 


\begin{frame}{Complex Numbers and Polynomials}


    \begin{center}\begin{tikzpicture} \node [mybox](box){\begin{minipage}{0.9\textwidth}\vspace{4pt}

        Every polynomial of degree $n$ has exactly $n$ complex roots, counting multiplicity. 
    
    \end{minipage}};
    \node[fancytitle, right=10pt] at (box.north west) {Theorem: Fundamental Theorem of Algebra};
    \end{tikzpicture}\end{center}
    
    \vspace{12pt}
    The proof of this theorem is beyond the scope of the course. 
    \vspace{12pt}
    
    \pause
    
    For example,
    \begin{itemize}
        \item $(x-2)^2$ is a 2$^{nd}$ order polynomial, it has two roots 
        \item $(x-2)^2(x-1)$ is a $3^{rd}$ order polynomial, it has three roots
    \end{itemize}
\end{frame}




\begin{frame}{Complex Numbers and Polynomials}

    \begin{center}\begin{tikzpicture} \node [mybox](box){\begin{minipage}{0.9\textwidth}

        If $\lambda \in \mathbb C$ is a root of a real polynomial $p (x)$, then the conjugate $\overline{\lambda}$ is also a root of $p(x)$.  
         

    \end{minipage}};
    \node[fancytitle, right=10pt] at (box.north west) {Theorem};
    \end{tikzpicture}\end{center}
    
    \vspace{12pt}
    Because of this theorem: if $\lambda$ is an eigenvalue of real matrix $A$ with eigenvector $\vec v$, then $\overline{\lambda}$ is an eigenvalue of $A$ with eigenvector $\vec{\overline{v}}$.
    
 
      

\medskip



\end{frame}



\begin{frame}{Examples} 

    \vspace{-12pt}
    \begin{enumerate}
    
        \item If $A$ is a $2\times2$ matrix and one of its eigenvectors is $\vec v = \spalignmat{1;i}$, give another  eigenvector of $A$ that is not a multiple of $\vec v$. 
        \vspace{24pt}
        \item Four of the eigenvalues of a $7\times7$ matrix are $-2, 4+i, -4 - i$, and $i$. What are the other eigenvalues? 

    \end{enumerate}
    
\end{frame}



\frame{\frametitle{Summary}

    \SummaryLine \vspace{4pt}
    \begin{itemize}\setlength{\itemsep}{8pt}

    \item polynomials with complex roots
    \item complex eigenvalues and eigenvectors
    
    \end{itemize}
    
    \vspace{8pt}
    

    
}










