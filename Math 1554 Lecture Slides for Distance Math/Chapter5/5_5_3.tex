\title{Rotation-Dilation Matrices}
\subtitle{\SubTitleName}
\institute[]{\Course}
\author{\Instructor}
\maketitle   

\frame{\frametitle{Topics and Objectives}
\Emph{Topics} \\
%\TopicStatement
\begin{itemize}

    \item rotation-dilation matrices and their eigenvalues
    
\end{itemize}

\vspace{0.5cm}

\Emph{Learning Objectives}\\

%\LearningObjectiveStatement

\begin{itemize}

    \item compute the eigenvalues of rotation-dilation matrices and express them in polar form
    % \item conjugate complex numbers
    % \item apply properties of complex numbers to determine whether mathematical statements that involve them are true or false
    
    % \item Diagonalize $2\times2$ matrices that have complex eigenvalues.
    \item identify the rotation and dilation of a linear transform using eigenvalues
    % \item Apply theorems to characterize matrices with complex eigenvalues.

\end{itemize}

% \vspace{0.25cm} 

% \Emph{Motivating Question}

% %Recall the rotation matrix (for $\pi/4$):
% %\[  A = \frac{1}{\sqrt{2}}  \spalignmat{ 1  -1 ; 1 1 }\]
%     What are the eigenvalues of a rotation matrix? 

} 

\begin{frame}{Rotations, Dilations, and Eigenvalues}

    The standard matrix for the transform that \pause rotates vectors by $\phi = \pi/4$ radians about the origin, and then scales (or dilates) vectors by $r = \sqrt{2}$, is 
    \pause 
    \[ A =  
    \spalignmat{r 0;0 r}\spalignmat{\cos\phi ,-\sin\phi ;\sin\phi ,\cos\phi } = 
    \begin{pmatrix} 1 & -1 \\ 1 & 1 \end{pmatrix} \]
    \pause 
    What are the eigenvalues of $A$? Express them in polar form. 
    
\end{frame}


\begin{frame}
\end{frame}


\begin{frame}{General Rotation-Dilation Case}

    The matrix in the previous example is an example of a rotation-dilation matrix. 
    
    \pause 
    
    \vspace{12pt}
    
    A rotation-dilation matrix has the form: 
    \[  C =  \spalignmat{ a  -b ; b  a } \]
    
    \pause 
    
    Calculate the eigenvalues of $C$ and express them in polar form.


\end{frame}


\begin{frame}
\end{frame}


\begin{frame}{Rotation-Dilation Matrices}

    \begin{center}\begin{tikzpicture} \node [mybox](box){\begin{minipage}{0.85\textwidth}\vspace{4pt}
        A matrix of the form $C =  \spalignmat{ a  -b ; b  a }$ is a \Emph{rotation-dilation matrix} \pause because it is the composition of a rotation by $\phi$ and dilation by $r$, \pause where 
        $$r^2 = a^2 + b^2, \quad \tan \phi = \frac ba$$
        \pause Moreover, the eigenvalues of $C$ are $\lambda = a \pm bi$. 
    \end{minipage}};
    \node[fancytitle, right=10pt] at (box.north west) {Definition  (Rotation-Dilation Matrix)};
    \end{tikzpicture}\end{center}
    
\end{frame}


\begin{frame}{Eigenvalues of Rotation-Dilation Matrices}

    \Emph{Example}: determine the eigenvalues of $A=\spalignmat{2 -3;3 2}$. 
    
    \pause 
    
    Because this is an example of a rotation-dilation matrix, we do not need to determine the characteristic polynomial. \pause The eigenvalues of this matrix are $$\lambda = 2 \pm 3i$$
\end{frame}


\frame{\frametitle{Summary}

    \SummaryLine \vspace{4pt}
    \begin{itemize}\setlength{\itemsep}{8pt}

    \item rotation-dilation matrices and their eigenvalues in polar form and Cartesian form

    \end{itemize}
    
    \vspace{8pt}
    

    
}