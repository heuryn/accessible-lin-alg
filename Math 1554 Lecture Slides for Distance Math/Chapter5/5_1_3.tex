\title{Eigenvalue Theorems}
\subtitle{\SubTitleName}
\institute[]{\Course}
\author{\Instructor}
\maketitle   

\frame{\frametitle{Topics and Objectives}
    \Emph{Topics} \\
    \TopicStatement
    \begin{itemize}

        % \item eigenspaces
        \item eigenvalue theorems

    \end{itemize}

    \vspace{0.5cm}

    \Emph{Objectives}\\

    \LearningObjectiveStatement

    \begin{itemize}

        
        % \item construct an eigenspace for a matrix
        
        \item apply theorems related to eigenvalues to characterize matrices and compute eigenvalues

    \end{itemize}
}



\begin{frame}{Motivating Questions}

    Suppose $A$ is a real $n\times n$ matrix. 
    
    \begin{itemize}
        \item<2-> How can we determine the eigenvalues of $A$? 
        \item<3-> If $A$ has some special structure (e.g. - singular, stochastic, triangular), what can we say about the eigenvalues of $A$? 
    \end{itemize}
    
    \onslide<4->{Theorems that deal with eigenvalues of a matrix can help us calculate eigenvalues. }
    
\end{frame}




\begin{frame}{Theorems}

    The following theorems can help us identify eigenvalues and eigenvectors of a matrix. 
    
    \vspace{12pt}
    
    \begin{itemize}
        \item<2-> The diagonal elements of a triangular matrix are its eigenvalues. \vspace{0.25cm}
        \item<3-> $A$ not invertible $\Leftrightarrow 0$ is an eigenvalue of $A$. \vspace{0.5cm}
        \item<4-> Stochastic matrices have an eigenvalue equal to 1.\vspace{0.5cm}
        \item<5-> If $\vec v_1, \vec v_2, \ldots , \vec v_k$ are eigenvectors that correspond to distinct eigenvalues, then $\vec v_1, \vec v_2, \ldots , \vec v_k$ are linearly independent.
    \end{itemize}
    
    \onslide<6->{Proofs of these theorems are relatively short.}\onslide<7->{There are many other eigenvalue theorems that we could explore!}
    
\end{frame}




\begin{frame}{Determining Eigenvalues by Inspection} 

    By inspection, give two eigenvalues for each of the following matrices.
    \begin{enumerate}
        \item $A=\frac 12 \spalignmat{1 1;1 1}$
        \item $B=\spalignmat{2 3;0 5}$
    \end{enumerate}


\end{frame}




\begin{frame}{\Emph{Warning!}} 

	\vspace{-12pt}
    % ~~ ~~ Highlight Box ~~ ~~
    \begin{center}\begin{tikzpicture} \node [mybox](box){\begin{minipage}{0.94\textwidth}\vspace{4pt}

        We cannot determine the eigenvalues of a matrix from its reduced form. 

    \end{minipage}};
    \end{tikzpicture}\end{center}
    % ~~ ~~ Highlight Box ~~ ~~
    
    In other words: row reductions can change the eigenvalues of a matrix. 
    
\end{frame}




\begin{frame}{Row Operations on $A$ Can Change its Eigenvalues} 

    Suppose $A = \spalignmat{1 1;1 1}$. The eigenvalues and corresponding eigenvectors are 
    
    $$\lambda_1 = 2, \ \vec v_1 = \spalignmat{1;1}, \qquad \lambda_2 = 0, \ \vec v_2 = \spalignmat{1;-1} $$
    
    We can verify this:
    \pause 
            \begin{align*}
            & A \vec v_1
            = \spalignmat{1 1;1 1} 
            \spalignmat{1;1} =  \spalignmat{2;2} = 2 \vec v_1\\
            & A \vec v_2
            = \spalignmat{1 1;1 1} 
            \spalignmat{1;-1} =  \spalignmat{0;0} = 0\vec v_2
            \end{align*}

    \pause
    
    But the RREF of $A$ is $\spalignmat{1 1;0 0}$, whose eigenvalues are $1$ and $0$. 

\end{frame}


\frame{\frametitle{Summary}

    \SummaryLine \vspace{4pt}
    \begin{itemize}\setlength{\itemsep}{8pt}

        \item applying eigenvalue theorems to characterize matrices and compute eigenvalues
        
    \end{itemize}
    \vspace{8pt}

}







