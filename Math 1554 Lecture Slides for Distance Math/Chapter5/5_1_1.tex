\title{Eigenvalues and Eigenvectors}
\subtitle{\SubTitleName}
\institute[]{\Course}
\author{\Instructor}
\maketitle   

\frame{\frametitle{Topics and Objectives}
    \Emph{Topics} \\
    \TopicStatement
    \begin{itemize}

        \item eigenvectors, eigenvalues %, eigenspaces
        % \item Eigenvalue theorems

    \end{itemize}

    \vspace{0.5cm}

    \Emph{Objectives}\\

    \LearningObjectiveStatement

    \begin{itemize}

        \item determine whether a given vector is an eigenvector of a given matrix

        \item determine whether a given number is an eigenvalue of a given matrix
        
        % \item Construct an eigenspace for a matrix.
        
        % \item Apply theorems related to eigenvalues (for example, to characterize the invertibility of a matrix).

    \end{itemize}
}


\begin{frame}{Motivating Problem}

    Consider the linear transform 
        $$T_A(\vec x) = A \vec x = \spalignmat{0 1;1 0} \spalignmat{x_1;x_2}.$$

    \pause 
    
    Can you state a non-zero vector, $\vec v \in \mathbb R^2$ that satisfies the following equation? 
    
    $$A \vec v = \lambda \vec v, \quad \lambda \in \mathbb R$$
    

    \pause 
    \begin{tikzpicture}[scale=1.2]
    
        \coordinate (O) at (0,0);  % variable for origin
              
        % axes
        \draw[-, thin,black] (-1.25,0) -- (1.5,0) node[anchor=west] {$x_1$};
        \draw[-, thin,black] (0,-0.7) -- (0.0,1.75) node[anchor=east] {$x_2$};
        
        % reflection line
        \draw[-,thick,Teal](-.7,-.7)--(1.5,1.5) node[anchor=south] {$x_2=x_1$};
        
    \end{tikzpicture}

\end{frame}


\begin{frame}{Eigenvectors and Eigenvalues}

    If $A \in \R^{n\times n}$, and there is a $\vec v \neq \vec 0$ in $\R^n$, and
        \[ A\vec v=\lambda \vec v \]
    then \onslide<2->{ $\vec v$ is an \Emph{eigenvector} for $A$, and $\lambda \in \mathbb C$ is the corresponding \Emph{eigenvalue}.} 

    \vspace{12pt}
    
    \onslide<2->{Note that}
    \begin{itemize}
        \item<3-> we will only consider the case where $A$ is square
        \item<4-> even when all entries of $A$ and $\vec v$ are real, $\lambda$ can be complex
        (a rotation of the plane has no \Emph{real} eigenvalues.) 
        \item<5-> if $\lambda \in \mathbb R$, then
        \begin{itemize}
            \item when $\lambda > 0$, $A\vec v$ and $\vec v$ point in the same direction
            \item when $\lambda < 0$, $A\vec v$ and $\vec v$ point in opposite directions
        \end{itemize}    \end{itemize}


\end{frame}



\begin{frame}{Determining Whether a Vector is an Eigenvector}

    Which of the following vectors are eigenvectors of $A=\spalignmat{ 1 1 ; 1 1}$? 

    \begin{enumerate}[a)]
        \item $\vec v_1 = \spalignmat{1;1}$ 
        \item $\vec v_2 = \spalignmat{2;2}$ 
        \item $\vec v_3 = \spalignmat{1;-1}$ 
        \item $\vec v_4 = \spalignmat{2;1}$ 
        \item $\vec v_5 = \spalignmat{0;0}$
    \end{enumerate}

\end{frame}


\begin{frame}{Determining Whether a Number is an Eigenvalue}

    Determine whether $\lambda =3$ is an eigenvalue of $A=\spalignmat{2 -4 ; -1 -1}$. 


\end{frame}


\begin{frame}{Another Interpretation of What Eigenvalues Are}

    From the previous examples, we saw how an eigenvalue of a matrix is a number, $\lambda$, that
    
    \pause 
    \begin{itemize}
        \item satisfies $A\vec v = \lambda \vec v$ for eigenvector $\vec v$
        \item makes $A - \lambda I$ singular
    \end{itemize}

    \pause 
    Recall: a singular matrix is not invertible. 

\end{frame}



\frame{\frametitle{Summary}

    \SummaryLine \vspace{4pt}
    \begin{itemize}\setlength{\itemsep}{8pt}
        \item the definitions of eigenvectors and eigenvalues %, eigenspaces

        \item determine whether a given vector is an eigenvector of a given matrix

        \item determine whether a given number is an eigenvalue of a given matrix
    \end{itemize}
    \vspace{8pt}
    Later in our course we will take explore ways to compute the eigenvalues and eigenvectors of a matrix, as well as some of their applications. 
    
}















