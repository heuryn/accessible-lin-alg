\title{PageRank}
\subtitle{\SubTitleName}
\institute[]{\Course}
\author{\Instructor}
\maketitle   

\frame{\frametitle{Topics and Objectives}
\Emph{Topics} \\
%\TopicStatement
\begin{itemize}

    % \item review of Markov chains

    % \item theorems describing the steady state of a Markov chain
    
    \item applying Markov chains to model web traffic
    
    % \item calculating the PageRank of a web

\end{itemize}

\vspace{0.5cm}

\Emph{Learning Objectives}\\

%\LearningObjectiveStatement

\begin{itemize}

    % \item determine whether a stochastic matrix is regular
    
    % \item calculate the steady-state of a Markov process
    
    % \item apply matrix powers and theorems to characterize the long-term behaviour of a Markov chain

    \item construct a transition matrix and a Markov Chain for a given web to model web traffic %, and compute the PageRank of the web

\end{itemize}

\vspace{0.25cm} 

% \Emph{Motivating Question}

} 



\begin{frame}\frametitle{Ordering Results in a Search}

    There are many search engines that we can use to find relevant information on the web. 
    
    \pause 
    
    \begin{itemize}
        \item<2-> when searching for information on the Internet using any search engine, we can be presented with \Emph{many} search results
        \item<3-> for a search engine to give useful information to the user, it must quickly order search results in some way 
        \item<4-> essentially: how can the engine quickly decide which results appear at the top of the list? 
    \end{itemize}
    
    \vspace{12pt}
    \onslide<5->{PageRank (PR) is an algorithm that was introduced to order search results. }
    
\end{frame}



\begin{frame}\frametitle{Mathematical Model for Web Traffic}

    The PageRank algorithm is based on a mathematical model that assumes that we have:
    
    \begin{itemize}
        \item<2-> a collection of web pages that have links to each other
        \item<3-> users who are navigating the web
        \item<4-> a set of rules that govern how the users navigate the web
    \end{itemize}
\end{frame}



\begin{frame}\frametitle{How Users Move Between Pages}
    We impose assumptions about how the users navigate the web: 
    \begin{itemize}\setlength{\itemsep}{4pt}
    	\item<2->[a)] A user on a web page is equally likely to go to any page that their page links to. 
	    \item<3->[b)] If a user is on a page that does not link to other pages, the user stays at their page.
	    \item<4->[c)] The distribution of users can be modeled using a Markov process, $\vec x_{k+1} = P \vec x_k$, where 

	    \begin{itemize}\setlength{\itemsep}{4pt}
	        \item $\vec x_k \in \mathbb R^n$ is a probability vector, gives the proportion of users on each page at iteration $k$
	        \item $P$ is an $n\times n$ stochastic matrix
    	    \item $n$ is the number of pages in the web
	    \end{itemize}
	\end{itemize}
	
% 	\vspace{12pt}
%     To see how this model is developed in practice, we will explore a specific example of a very small web. 

\end{frame}



\begin{frame}\frametitle{Example Web with Five Pages}
    A set of web pages link to each other according to the diagram below. Use the assumptions on the previous slide to construct a Markov chain that represents how users navigate the web.
    
    \pause 
    
    \begin{center}
        \begin{tikzpicture}
            \begin{scope}[->,>=stealth',shorten >=1pt,auto,node distance=1.5cm,thick, main node/.style={circle,fill=black!05,draw}]
            \node[main node] (1) {A};
            \node[main node] (2) [right of=1] {B};
            \node[main node] (3) [below of=1] {C};
            \node[main node] (4) [below of=2] {D};
            \node[main node] (5) [right of=4] {E};
            \path[every node/.style={font=\sffamily\small}]
            (1) edge node[below] {} (2)
            (1) edge node[below] {} (4)
            (2) edge node [above] {} (1)
            edge [left] node {}  (3) 
            (3) edge node {} (1)
            edge node[right] {} (2)
            (2) edge node [above] {} (4)
            (4) edge node [above] {} (5);
            \end{scope}
        \end{tikzpicture}  
    \end{center}    
    
    
    
\end{frame}



\begin{frame}
\frametitle{Transition Matrix, Importance, and PageRank}

    \begin{itemize}
        \item<2-> The square matrix we constructed in the previous example is a \Emph{transition matrix}. It describes how users transition between pages in the web. 
        \item<3-> The steady-state vector, $\vec q$, for the Markov-chain, can characterize the long-term behavior of users in a given web. 
        \item<4-> The \Emph{importance} of a page in a web are the entries of $\vec q$.
        \item<5-> The \Emph{PageRank} is the ranking assigned to each page based on its importance. The highest ranked page has PageRank 1, the second PageRank 2, and so on. 
        \item<6-> Two pages with same importance receive the same PageRank (some other method would be needed to resolve ties).
    \end{itemize}

    % \vspace{12pt} 
    
    % Is the transition matrix in Example 1 a regular matrix? 

\end{frame}



\begin{frame}
\frametitle{Remaining Questions}

    Our simple mathematical model has some limitations that must be addressed for it to be useful. 
    
    \begin{itemize}
        \item<2-> Will our transition be regular stochastic? 
        \item<3-> What can we do to build a model that will give us a regular stochastic matrix? 
        \item<4-> What can we do to better handle the pages that do not link to other pages? 
    \end{itemize}

\end{frame}


\frame{\frametitle{Summary}

    \SummaryLine \vspace{4pt}
    \begin{itemize}\setlength{\itemsep}{8pt}

    \item constructing a transition matrix and an associated Markov Chain for a given web to model web traffic 
    
    \item the importance vector for web pages, and the associated PageRank ranking of each page
    
    \end{itemize}
    
    \vspace{8pt}
    

    
}



% \begin{frame}
% \frametitle{Adjustment 1}

    
%     \begin{center}\begin{tikzpicture} \node [mybox](box){\begin{minipage}{0.85\textwidth}
%         \vspace{4pt} 
%         If a user reaches a page that does not link to other pages, the user will choose any page in the web, with equal probability, and move to that page.
%     \end{minipage}};
%     \node[fancytitle, right=10pt] at (box.north west) {Adjustment 1};
%     \end{tikzpicture}\end{center}


%     \vspace{6pt} 
    
%     Let's denote this modified transition matrix as $P_*$. Our transition matrix in Example 1 becomes: 

% \end{frame}


% \begin{frame}
% \frametitle{Adjustment 2}

%     \begin{center}\begin{tikzpicture} \node [mybox](box){\begin{minipage}{0.85\textwidth}
%         \vspace{4pt} 
%         A user at any page will navigate to any page among those that their page links to with equal probability $p$, and to any page in the web with equal probability $1-p$. The transition matrix becomes $$G = pP_* + (1 - p)K$$
%         All the elements of the $n\times n$ matrix $K$ are equal to $1/n$. 
%     \end{minipage}};
%     \node[fancytitle, right=10pt] at (box.north west) {Adjustment 2};
%     \end{tikzpicture}\end{center}

%     \vspace{6pt} 
    
%     $p$ is referred to as the \Emph{damping factor}, Google is said to use $p = 0.85$. \\[12pt] With adjustments 1 and 2, our the Google matrix is:
    
    
    
% \end{frame}


% \begin{frame}

% \end{frame}


% \begin{frame}

% \frametitle{Computing Page Rank}

% \begin{itemize}
% 	\item Because $G$ is stochastic, for any initial probability vector $\vec x_0$, $$\lim_{n \rightarrow \infty} G^n \vec x_0 = \vec q$$

% 	\item We can obtain steady-state evaluating $G^n\vec x_0$ for large $n$, by solving $G\vec q = \vec q$, or by evaluating $\vec x_n = G \vec x_{n-1}$ for large $n$.
% 	\item Elements of the steady-state vector give the importance of each page in the web, which can be used to determine PageRank. 
% 	\item Largest element in steady-state vector corresponds to page with PageRank 1, second largest with PageRank 2, and so on.

% \end{itemize}

% On an exam, 
% \begin{itemize} 
% 	\item problems that require a calculator will not be on your exam
% 	\item you may construct $G$ using factions instead of decimal expansions	
% \end{itemize}
    
% \end{frame}







% \begin{frame}
% \frametitle{There is (of course) Much More to PageRank}

% \begin{columns}
% \begin{column}{.35\textwidth}
% \begin{center}
%         \includegraphics[width=1.0\textwidth]{Chapter5/images/BrinPage.png} 
% \end{center}

%  The PageRank Algorithm currently used by Google is under constant development, and tailored to individual users. 
% \end{column}
% \begin{column}{.65\textwidth}

% \begin{itemize}
%     \item  When PageRank was devised, in 1996, Yahoo! used humans to provide a "index for the Internet, " which was 10 million pages.  
%     \item  The PageRank algorithm was produced as a competing method.  The patent was awarded to Stanford University, and exclusively licensed to the newly formed Google corporation. 
%     \item   Brin and Page combined the PageRank algorithm with a webcrawler to provide regular updates to the transition matrix for the web.  
%     \item  The explosive growth of the web soon overwhelmed human based approaches to searching the internet. 
% \end{itemize}

% \end{column}
% \end{columns}
%     \begin{center}
%     \end{center}

% \end{frame}


% \begin{frame}[fragile=singleslide]
% \frametitle{WolframAlpha and MATLAB/Octave Syntax}

%     Suppose we want to compute
%     \begin{align*} 
%         \spalignmat{.8 .1 .2; .2 .6 .3; .0 .3 .5}^{10} 
%     \end{align*}
    
%     \begin{itemize}
%         \item At wolframalpha.com, we can use the syntax:
%     {\small 
%     \noindent \begin{verbatim}MatrixPower[{{.8,.1,.2},{.2,.6,.3},{.0,.3,.5}},10]\end{verbatim}}
    
%         \item In MATLAB, we can use the syntax
%     {\small 
%     \noindent \begin{verbatim}[.8 .1 .2 ;.2 .6 .3;.0 .3 .5]^10\end{verbatim}}    
%         \item Octave uses the same syntax as MATLAB, and there are several free, online, Octave compilers. For example: https://octave-online.net. 
%     \end{itemize}
    
%     You will need to compute a few matrix powers in your homework, and in your future courses, depending on what courses you end up taking.  

% \end{frame}




% \begin{frame}
% \frametitle{Example 2 (if time permits)}



%     Construct the Google Matrix for the web below. Which page do you think will have the highest PageRank? How would your result depend on the damping factor $p$? Use software to explore these questions.     
%         \begin{center}
%         \begin{tikzpicture}
%             \begin{scope}[->,>=stealth',shorten >=1pt,auto,node distance=1.5cm,thick, main node/.style={circle,fill=black!05,draw}]
%             \node[main node] (1) {A};
%             \node[main node] (2) [right of=1] {B};
%             \node[main node] (3) [below of=1] {C};
%             \node[main node] (4) [below of=2] {D};
%             \path[every node/.style={font=\sffamily\small}]
%             (1) edge node [above] {} (3)
%             (3) edge node [above] {} (4)
%             (2) edge node [above] {} (3);
%             \end{scope}
%         \end{tikzpicture}  
%     \end{center}    

% \end{frame}

