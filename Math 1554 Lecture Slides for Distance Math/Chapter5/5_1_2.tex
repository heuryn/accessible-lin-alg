\title{Eigenspaces}
\subtitle{\SubTitleName}
\institute[]{\Course}
\author{\Instructor}
\maketitle   

\frame{\frametitle{Topics and Objectives}
    \Emph{Topics} \\
    \TopicStatement
    \begin{itemize}

        \item eigenspaces
        % \item Eigenvalue theorems

    \end{itemize}

    \vspace{0.5cm}

    \Emph{Objectives}\\

    \LearningObjectiveStatement

    \begin{itemize}

        \item construct and sketch the eigenvectors for a matrix given its eigenvalues
        
        \item construct and sketch an eigenspace for a matrix
        
        % \item Apply theorems related to eigenvalues (for example, to characterize the invertibility of a matrix).

    \end{itemize}
}






\begin{frame}{Eigenspaces}

    % ~~ ~~ Highlight Box ~~ ~~
    \begin{center}\begin{tikzpicture} \node [mybox](box){\begin{minipage}{0.80\textwidth}\vspace{2pt}

        Suppose $A \in \R^{n \times n}$. The eigenvectors for a given $\lambda$ span a subspace of $\R^n$ called the $\lambda$-\Emph{eigenspace} of $A$.

    \end{minipage}};
    \node[fancytitle, right=10pt] at (box.north west) {Definition};
    \end{tikzpicture}\end{center}
    % ~~ ~~ Highlight Box ~~ ~~

    \vspace{2pt} 

    \Emph{Note:} the $\lambda$-eigenspace for matrix $A$ is $\textrm{Nul} (A-\lambda I)$, because: \pause 
    \begin{align*}
        A \vec v & = \lambda \vec v \\
        A \vec v - \lambda I \vec v & = \vec 0 \\
        (A - \lambda I) \vec v &= \vec 0
    \end{align*}
    \pause If $\vec v \ne 0$, we must have that $A - \lambda I$ is singular. \pause Thus, eigenvectors span the null space of $A - \lambda I$. 
\end{frame}



\begin{frame}{Eigenspaces in $\mathbb R^2$}

    Construct a basis for the eigenspaces for the matrix whose eigenvalues are given. Sketch the eigenvectors and eigenspaces. $$A = \spalignmat{ 5 -6; 3 -4}, \quad \lambda = -1, 2$$

\end{frame}



\begin{frame}{Eigenspaces in $\mathbb R^3$}

    Construct a basis for the eigenspaces for the matrix whose eigenvalues are given. $$A= \spalignmat{ 3 0 0;0 2 1;0 1 2}, \quad \lambda = 1, 3$$

\end{frame}




\frame{\frametitle{Summary}

    \SummaryLine \vspace{4pt}
    \begin{itemize}\setlength{\itemsep}{8pt}

        \item constructing the eigenvectors for a matrix given its eigenvalues
        
        \item constructing an eigenspace for a matrix
        
    \end{itemize}
    \vspace{8pt}

}







