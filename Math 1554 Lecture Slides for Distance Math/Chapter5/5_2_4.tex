\title{Similar Matrices}
\subtitle{\SubTitleName}
\institute[]{\Course}
\author{\Instructor}
\maketitle   



\frame{\frametitle{Topics and Objectives}

    \Emph{Topics} \\
    \TopicStatement
    \begin{itemize}
    
        \item similar matrices
    
    \end{itemize}
    
    \vspace{0.5cm}
    
    \Emph{Objectives}\\
    
    \LearningObjectiveStatement
    
    \begin{itemize}
    
        \item apply the definition of similar matrices to determine whether mathematical statements that involve similar matrices are accurate
    
    \end{itemize}

}



\begin{frame}{Motivation: Matrix Powers}
    Suppose $A$ is an $n\times n$ matrix.
    \begin{itemize}
        \item in some applications we need to compute $A^k$ for large $k$
        \item computing $A^k$ directly could require many computations, especially if $n$ is large and many of the elements in $A$ are non-zero
   \end{itemize}
    Using the concept of similar matrices, we can obtain a more efficient approach. 
\end{frame}



\begin{frame}{Similar Matrices}

    \vspace{-16pt}
    
    % ~~ ~~ Highlight Box ~~ ~~
    \begin{center}\begin{tikzpicture} \node [mybox](box){\begin{minipage}{0.95\textwidth}\vspace{2pt}

        $n \times n$ matrices $A$ and $B$ are \Emph{similar} if there is a $P$ so that $A = PBP^{-1}$.

    \end{minipage}};
    \node[fancytitle, right=10pt] at (box.north west) {Definition};
    \end{tikzpicture}\end{center}
    % ~~ ~~ Highlight Box ~~ ~~
    
    \vspace{4pt}
    \pause
    \Emph{Example}: if $P = \spalignmat{1 1;0 1}$ and $B=\spalignmat{2 0;0 1}$, then $P^{-1} = \spalignmat{1 -1; 0 1}$ and
    \pause
    \begin{align*}
        PBP^{-1} = \spalignmat{1 1; 0 1}\spalignmat{2 0; 0 1}\spalignmat{1 -1; 0 1} = A
    \end{align*}
    By construction, $A$ is similar to $B$. 
    
    \vspace{6pt}
    
    Later in this course we will investigate a method for constructing $P$ and $B$ when given only a square matrix $A$. 

\end{frame}








\begin{frame}{Similar Matrices and the Characteristic Polynomial}

    % \vspace{-24pt}
    

    
    \vspace{-24pt}
    
    % ~~ ~~ Highlight Box ~~ ~~
    \begin{center}\begin{tikzpicture} \node [mybox](box){\begin{minipage}{0.95\textwidth}\vspace{2pt}

        If $A$ and $B$ similar, then they have the same characteristic polynomial.

    \end{minipage}};
    \node[fancytitle, right=10pt] at (box.north west) {Theorem};
    \end{tikzpicture}\end{center}
    % ~~ ~~ Highlight Box ~~ ~~


\end{frame}



\begin{frame}{Proof: Similar Matrices, Characteristic Polynomials}

    \onslide<2->{The characteristic polynomial of $A$ is $\det(A-\lambda I)$, and if $A = PBP^{-1}$, then}
    \begin{align*}
        \onslide<3->{A - \lambda I &= PBP^{-1} - \lambda I\\}
        \onslide<4->{&= PBP^{-1} - \lambda PP^{-1}\\}
        \onslide<5->{&= (PB - \lambda P)P^{-1}\\}
        \onslide<6->{&= P(B - \lambda I)P^{-1}\\}
        \onslide<7->{\det(A - \lambda I) &= \det( P(B - \lambda I)P^{-1} )\\}
        \onslide<8->{\det(A - \lambda I) &= \det(P) \det (B - \lambda I) \det (P^{-1}) \\}
        \onslide<9->{\det(A - \lambda I) &= \det(P)\det(P^{-1})\det (B - \lambda I)\\}
        \onslide<10->{\det(A - \lambda I) &= \det (B - \lambda I)}
    \end{align*}\onslide<10->{The characteristic polynomials of $A$ and $B$ are the same.  }

\end{frame}




\begin{frame}{Similar Matrices and Eigenvalues}

    \Emph{Note}
    \begin{itemize}
        \item<2-> If two matrices have the same characteristic polynomial, then they have the same eigenvalues. 
        \item<3-> The converse is not always true: two matrices can have the same eigenvalues but not be similar. 
    \end{itemize}
    
    \vspace{4pt}
    
    \onslide<4->{Consider $$A = \spalignmat{ 0 1 ; 0 0 }, \quad B = \spalignmat{ 0 0 ; 0 0 }, \quad \lambda = 0,0$$ Can $A$ and $B$ be similar? } \onslide<5->{If $A$ and $B$ are similar, then $A=PBP^{-1}$, but }
    \onslide<6->{$$PBP^{-1} = P\spalignmat{0 0;0 0}P^{-1} = \spalignmat{0 0;0 0} \ne A$$}
\end{frame}




\begin{frame}{True or False: Similar Matrices}

    Indicate whether the statements are true or false.
    
    \begin{enumerate}[a)]
        \item If $A$ is similar to the identity matrix, $I$, then $A = I$.
        % \item If $A$ is similar to the zero matrix, $0$, then $A = 0$.
        \item If $A$ is similar to $B$, and $A=PBP^{-1}$, then $A^2 = PB^2P^{-1}$. 
        \item If $A$ and $B$ have the same eigenvalues, then $A$ and $B$ are similar. 
    \end{enumerate}

\end{frame}


\frame{\frametitle{Summary}

    \SummaryLine \vspace{4pt}
    \begin{itemize}\setlength{\itemsep}{8pt}

        \item the definition of similar matrices
        \item using the definition of similar matrices to determine whether mathematical statements that involve them are accurate

    \end{itemize}
    \vspace{8pt}
    Later in this course we will introduce methods for constructing matrices $P$ and $B$ so that, given square matrix $A$, we might be able to write $A=PBP^{-1}$. 
    \pause
}




