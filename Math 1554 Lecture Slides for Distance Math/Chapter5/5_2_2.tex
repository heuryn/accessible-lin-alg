\title{Algebraic and Geometric Multiplicities}
\subtitle{\SubTitleName}
\institute[]{\Course}
\author{\Instructor}
\maketitle   



\frame{\frametitle{Topics and Objectives}
\Emph{Topics} \\
\TopicStatement
\begin{itemize}

    % \item the characteristic polynomial of a matrix

    \item the algebraic and geometric multiplicity of eigenvalues

    % \item Similar matrices

\end{itemize}

\vspace{0.5cm}

\Emph{Objectives}\\

\LearningObjectiveStatement

\begin{itemize}

    \item calculate algebraic and geometric multiplicities
    \item construct matrices that have eigenvalues with a given algebraic and/or geometric multiplicity

\end{itemize}

\vspace{0.25cm} 

}



\begin{frame}{Algebraic Multiplicity}

    % ~~ ~~ Highlight Box ~~ ~~
    \begin{center}\begin{tikzpicture} \node [mybox](box){\begin{minipage}{0.85\textwidth}\vspace{2pt}

        The \Emph{algebraic multiplicity} of an eigenvalue is its multiplicity as a root of the characteristic polynomial.

    \end{minipage}};
    \node[fancytitle, right=10pt] at (box.north west) {Definition};
    \end{tikzpicture}\end{center}
    % ~~ ~~ Highlight Box ~~ ~~
    
    \Emph{Example} \\ Compute the algebraic multiplicities of the eigenvalues for the matrix
    \[ \spalignmat{ 1 0 0 0 ; 0 0 0 0 ; 0 0 -1 0 ; 0 0 0 0 } \]


\end{frame}




\begin{frame}{Geometric Multiplicity}

    % ~~ ~~ Highlight Box ~~ ~~
    \begin{center}\begin{tikzpicture} \node [mybox](box){\begin{minipage}{0.85\textwidth}\vspace{2pt}

        The \Emph{geometric multiplicity} of an eigenvalue $\lambda$  is  the dimension of $ \operatorname{Null} (A - \lambda I)$. 

    \end{minipage}};
    \node[fancytitle, right=10pt] at (box.north west) {Definition};
    \end{tikzpicture}\end{center}
    % ~~ ~~ Highlight Box ~~ ~~
    \vspace{-6pt}
    \begin{itemize}
        \item  Geometric multiplicity is always at least 1.  It can be smaller than algebraic multiplicity. 
        
        \item  Here is the basic example:  
         \[ \spalignmat{ 0 1 ; 0 0 } \]
         $\lambda =0$ is the only eigenvalue. Its algebraic multiplicity is 2, but the geometric multiplicity is 1.  
    \end{itemize}

\end{frame}




\begin{frame}{Properties of Algebraic and Geometric Multiplicities}

    Suppose that, for an $n\times n$ matrix $A$,
    \begin{itemize}
        \item $a_i$ is the algebraic multiplicity of $\lambda_i$
        \item $g_i$ is the geometric multiplicity of $\lambda_i$
    \end{itemize}
    
    \pause 
    
    \vspace{12pt}
    The algebraic and geometric multiplicities have the following properties. 
    \begin{itemize}
        \item $1 \le a_i \le n$
        \item $1 \le g_i \le a_i$
    \end{itemize}
    
\end{frame}




\begin{frame}{Example}

    Give an example of a $4\times 4$ matrix with $\lambda =0$ the only eigenvalue, but the geometric multiplicity of $\lambda=0$ is one. 

\end{frame}

\begin{frame}{Example: Algebraic Multiplicity}

    For what values of $k$ does the matrix have one real eigenvalue with algebraic multiplicity 2? 
    
    $$\spalignmat{-3 k; 2 -6}$$

\end{frame}


\frame{\frametitle{Summary}

    \SummaryLine \vspace{4pt}
    \begin{itemize}\setlength{\itemsep}{8pt}

        \item the algebraic and geometric multiplicity of eigenvalues
        \item constructing matrices that have eigenvalues with given algebraic or geometric multiplicities

    \end{itemize}
    \vspace{8pt}

}










