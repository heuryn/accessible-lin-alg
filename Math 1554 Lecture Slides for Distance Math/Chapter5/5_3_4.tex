\title{Computing Matrix Powers with Diagonalization}
\subtitle{\SubTitleName}
\institute[]{\Course}
\author{\Instructor}
\maketitle   

\vspace{1cm} 

\frame{\frametitle{Topics and Objectives}
\Emph{Topics} \\
%\TopicStatement
\begin{itemize}

    % \item diagonal, similar, and diagonalizable matrices

    \item matrix powers and diagonalization

\end{itemize}

\vspace{0.5cm}

\Emph{Learning Objectives}\\

\LearningObjectiveStatement

\begin{itemize}

    % \item determine whether a matrix can be diagonalized, and if possible diagonalize a square matrix
    
    \item apply diagonalization to compute matrix powers

\end{itemize}


}



\begin{frame}{Motivation: Matrix Powers}
    Suppose $A$ is an $n\times n$ matrix. Recall that:
    \begin{itemize}
        \item in some applications we need to compute $A^k$ for large $k$
        \item computing $A^k$ directly could require many computations, especially if $n$ is large and many of the elements in $A$ are non-zero
   \end{itemize}
   \pause 
   
    Using the concept of similar matrices, we can obtain a more efficient approach. 
\end{frame}





\begin{frame}
\frametitle{Example: Matrix Powers}
    Suppose $A$ is a $2\times2$ matrix whose eigenvalues and associated eigenvectors are as below. Compute $A^{100}$. 
    $$\lambda_1 = -\frac 12, \quad \vec v_1 = \spalignmat{2;1},\qquad \lambda_2 = \frac12, \quad \vec v_2 = \spalignmat{1;-2}$$
    \onslide<2->{Because the eigenvalues of $A$ are distinct, we can diagonalize $A$. }
    \begin{align*}
       \onslide<2->{A &= PDP^{-1} \\}
       \onslide<3->{A^2 &= PDP^{-1}PDP^{-1} = PD^2P^{-1} \\}
       \onslide<4->{A^3 &= PDP^{-1}PDP^{-1}PDP^{-1} = PD^3P^{-1} \\}
       \onslide<5->{\vdots &= \vdots \\ A^k &= PD^kP^{-1} }
    \end{align*}
    \onslide<6->{Thus, to compute $A^{100}$, we can instead compute $PD^{100}P^{-1}$.}
    
\end{frame}





\begin{frame}
\frametitle{Example: Matrix Powers}

    Our given eigenvalues and eigenvectors were
    $$\lambda_1 = -\frac 12, \quad \vec v_1 = \spalignmat{2;1},\qquad \lambda_2 = \frac12, \quad \vec v_2 = \spalignmat{1;-2}$$
    Using these values, $A^{100}$ becomes
    \begin{align*}
        \onslide<2->{A^{100} = PD^{100}P^{-1} &= \spalignmat{2 1;1 -2} \ \spalignmat{-\frac12 0;0 \frac12}^{100}\ \left(\frac15 \spalignmat{2 1;1 -2} \right)} \\
        \onslide<3->{&=\frac15 \spalignmat{2 1;1 -2} \ \spalignmat{2^{-100} 0;0 2^{-100}}\ \spalignmat{2 1;1 -2} \\}
        \onslide<4->{&=\frac{2^{-100}}{5} \spalignmat{2 1;1 -2} \spalignmat{1 0;0 1}  \spalignmat{2 1;1 -2}}
        \onslide<5->{=2^{-100} \spalignmat{1 0;0 1} }
    \end{align*}
\end{frame}   





\frame{\frametitle{Summary}

    \SummaryLine \vspace{4pt}
    \begin{itemize}\setlength{\itemsep}{8pt}

        \item computing matrix powers using the diagonalization of an $n\times n$ matrix 

    \end{itemize}
    
    \vspace{8pt}
    

    
}


