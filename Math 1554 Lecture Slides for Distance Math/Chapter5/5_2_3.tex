\title{Markov Chains and Eigenvectors}
\subtitle{\SubTitleName}
\institute[]{\Course}
\author{\Instructor}
\maketitle   



\frame{\frametitle{Topics and Objectives}
\Emph{Topics} \\
\TopicStatement
\begin{itemize}

    \item the use of eigenvalues and eigenvectors to study Markov chains

\end{itemize}

\vspace{0.5cm}

\Emph{Objectives}\\

\LearningObjectiveStatement

\begin{itemize}

    \item apply eigenvalues and eigenvectors to characterize the long-term behavior of Markov chains

\end{itemize}

\vspace{0.25cm} 

}


\begin{frame}{The Long-Term Behavior of Markov Chains}

    \begin{itemize}
        \item recall that we often want to know what happens to a Markov chain $$\vec x_{k+1} = P\vec x_k, \quad k = 0, 1, 2, \ldots $$ as $k \rightarrow \infty$
        \item if $P$ is a regular stochastic matrix there will be a unique steady-state
   \end{itemize}

    \vspace{6pt} 
    
    \pause 
    
    Now we can explore the following questions.
    \begin{itemize}
        \item if we do not know whether $P$ is regular, what else might we do to describe the long-term behavior of the system? 
        \item What can eigenvalues tell us about the behavior of these systems?
    \end{itemize}
\end{frame}





\begin{frame}{Example: Eigenvalues and Markov Chains}


    Consider the Markov Chain: $$\vec x_{k+1} = P\vec x_{k} = \spalignmat{1 0.1 ;0 0.9 }\vec x_{k}, \quad k= 0, 1, 2, 3, \ldots, \quad \vec x_0 = \spalignmat{0;1}.$$

    \pause 
    
    This system can be represented schematically with two nodes, A and B: 
    
    \vspace{-6pt} 
    
    \begin{center}
    \begin{tikzpicture}
    \begin{scope}[->,>=stealth',shorten >=1pt,auto,node distance=3cm, thick,main node/.style = {circle,fill=DarkBlue!30,draw,font=\sffamily\Large\bfseries}]
    \node[main node] (1) {A};
    \node[main node] (2) [right of=1] {B};
    \path[every node/.style={font=\sffamily\small}]
    (1) edge [loop left] node {1} (1)
    (2) edge node [above] {0.1} (1)
        edge [loop right] node {0.9} (2);
        \end{scope}
        \end{tikzpicture}
    \end{center}
    
    \vspace{-6pt} 
    
    \pause

    \Emph{Goal}: use eigenvalues to describe the long-term behavior of our system. 
    
\end{frame}





\begin{frame}{Example: Eigenvalues and Markov Chains}

    Consider the Markov Chain: $$\vec x_{k+1} = P\vec x_{k} = \spalignmat{1 0.1 ;0 0.9 }\vec x_{k}, \quad k= 0, 1, 2, 3, \ldots, \quad \vec x_0 = \spalignmat{1;0}.$$

    Use the eigenvalues and eigenvectors of $P$ to determine what $\vec x_k$ tends to as $k\to\infty$. The eigenvalues and eigenvectors of $P$ are
    
    $$\lambda_1 = 1, \ \vec v_1 = \spalignmat{1;0}, \quad \lambda_2 = 0.9, \ \vec v_2 = \spalignmat{1;-1}$$
    
    
 
    
    
\end{frame}



\begin{frame}{A More General Example}

    The eigenvalues of a $3\times 3$ stochastic matrix $A$ are 
    $$\lambda_1 = 1 , \quad \lambda_2=\frac{1}{4}, \quad \lambda_3 = \frac 18$$
    The respective eigenvectors corresponding to these eigenvalues are $\vec v_1$, $\vec v_2$, $\vec v_3$. 
    
    \vspace{12pt}
    
    If $\vec p$ is a probability vector in $\mathbb R^3$, what does $A^k\vec p $ tend to as $k\to \infty$? 

\end{frame}



\frame{\frametitle{Summary}

    \SummaryLine \vspace{4pt}
    \begin{itemize}\setlength{\itemsep}{8pt}

        \item using eigenvalues and eigenvectors to study the long-term behavior of Markov chains

    \end{itemize}
    \vspace{8pt}
    
}




\begin{frame}
\end{frame}