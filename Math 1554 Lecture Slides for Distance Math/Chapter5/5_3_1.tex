\title{Diagonalization}
\subtitle{\SubTitleName}
\institute[]{\Course}
\author{\Instructor}
\maketitle   

\vspace{1cm} 

\frame{\frametitle{Topics and Objectives}
\Emph{Topics} \\
%\TopicStatement
\begin{itemize}

    \item diagonal, similar, and diagonalizable matrices

    \item diagonalizing matrices

\end{itemize}

\vspace{0.5cm}

\Emph{Learning Objectives}\\

\LearningObjectiveStatement

\begin{itemize}

    \item determine whether a matrix can be diagonalized, and if possible diagonalize a square matrix
    
    \item apply diagonalization to compute matrix powers

\end{itemize}


}



\begin{frame}\frametitle{Powers of Matrices}

\Emph{Motivation}: it can be useful to take large powers of matrices, for example $A^k$, for large $k$. \\[12pt] \pause \Emph{But}: multiplying two $n\times n$ matrices requires roughly $n^3$ computations. Is there a more efficient way to compute $A^k$? 


\end{frame}


\begin{frame}
\frametitle{Diagonal Matrices}

    \vspace{-12pt}
    % ~~ ~~ Highlight Box ~~ ~~
    \begin{center}\begin{tikzpicture} \node [mybox](box){\begin{minipage}{0.95\textwidth}\vspace{2pt}

        A matrix is \Emph{diagonal} if the only non-zero elements, if any, are on the main diagonal. 
    \end{minipage}};
    \node[fancytitle, right=10pt] at (box.north west) {Definition};
    \end{tikzpicture}\end{center}
    % ~~ ~~ Highlight Box ~~ ~~
    
    \pause 
    \vspace{12pt}
    
    The following are all diagonal matrices. 
    
    $$
    I_n, \quad 
    \begin{pmatrix} 0&0\\0&0\end{pmatrix}, \quad
    \begin{pmatrix} 2&0\\0&0\end{pmatrix}, \quad
    \begin{pmatrix} 2&0&0&0\\0&0&0&0\\0&0&1&0\\0&0&0&3\end{pmatrix}
    $$ 

    We will only be working with diagonal square matrices in this course.
\end{frame}

\begin{frame}
\frametitle{Powers of Diagonal Matrices}

    If $A$ is diagonal, then $A^k$ is easy to compute. For example, 
    \begin{align*} 
        A & = \spalignmat{3 0 ; 0 5 } \\\\
        \onslide<2->{A^2 & = \spalignmat{3 0 ; 0 5 }\spalignmat{3 0 ; 0 5 } = \spalignmat{3^2 0 ; 0 5^2 }\\\\}
        \onslide<3->{A^k &= \spalignmat{3^k 0 ; 0 5^k }}
    \end{align*} 
    
    \vspace{.5cm} 
    \onslide<4->{But what if $A$ is not diagonal? }

\end{frame}




\begin{frame}
\frametitle{Diagonalization}

    % ~~ ~~ Highlight Box ~~ ~~
    \begin{center}\begin{tikzpicture} \node [mybox](box){\begin{minipage}{0.90\textwidth}\vspace{2pt}

        Suppose $A \in \R^{n \times n}$. We say that $A$ is \Emph{diagonalizable} if it is similar to a diagonal matrix, $D$. That is, we can write $A = PDP^{-1}$.

    \end{minipage}};
    \node[fancytitle, right=10pt] at (box.north west) {Definition};
    \end{tikzpicture}\end{center}
    % ~~ ~~ Highlight Box ~~ ~~
    
    \pause
    
    Also note that $A = PDP^{-1}$ if and only if
    \begin{align*}
        A &= \left(\vec v_1 \ \vec v_2 \cdots \vec v_n\right)
        \begin{pmatrix} \lambda_1 & & & \\ & \lambda_2 & & \\ & & \ddots & \\ & & & \lambda_n \end{pmatrix}  
        \left(\vec v_1 \ \vec v_2 \cdots \vec v_n\right)^{-1} 
%    &= CDC^{-1}
    \end{align*}
    $\vec v_1,\dots,\vec v_n$ are linearly independent eigenvectors, and $\lambda_1,\dots,\lambda_n$ are the corresponding eigenvalues (\Emph{in order}).    


\end{frame}




\begin{frame}
\frametitle{Diagonalization: Proof}

    We construct $P = (\vec v_1 \ \vec v_2 \ \ldots  \vec v_n)$. Then
    \begin{align*}
        \onslide<2->{AP &= A(\vec v_1 \ \vec v_2 \ \ldots  \vec v_n) \\}
        \onslide<3->{&= (A\vec v_1 \ A\vec v_2 \ \ldots  A\vec v_n) \\}
        \onslide<4->{&= (\lambda_1\vec v_1 \ \lambda_2\vec v_2 \ \ldots  \lambda_n\vec v_n) \\}
        \onslide<5->{AP&= (\vec v_1 \ \vec v_2 \ \ldots  \vec v_n)\spalignmat{\lambda_1, , , ,;,\lambda_2,,,;,,,;,,,\lambda_n} \\
        &= PD}
    \end{align*}
    \onslide<6->{Or, $A=PDP^{-1}$. }
\end{frame}



\begin{frame}{Diagonalization}

    \vspace{-18pt} 
    \begin{center}\begin{tikzpicture} \node [mybox](box){\begin{minipage}{0.95\textwidth}\vspace{2pt}

    If $ A$ is diagonalizable $\Leftrightarrow A$ has $n$ linearly independent eigenvectors.

    \end{minipage}};
    \node[fancytitle, right=10pt] at (box.north west) {Theorem};
    \end{tikzpicture}\end{center}
    
    Note: the symbol $\Leftrightarrow$ means \Emph{if and only if}. \\[12pt]



\end{frame}



\begin{frame}\frametitle{Example 1}

    Diagonalize if possible. 
    \begin{align*}
        \spalignmat{ 2 6 ; 0 -1 }
    \end{align*}

\end{frame}


\begin{frame}\frametitle{Example 2}

    Diagonalize if possible. 
    \begin{align*}
        \spalignmat{ 1 1 ; 0 1}
    \end{align*}

\end{frame}





\frame{\frametitle{Summary}

    \SummaryLine \vspace{4pt}
    \begin{itemize}\setlength{\itemsep}{8pt}

        \item the diagonalization of an $n\times n$ matrix

    \end{itemize}
    
    \vspace{8pt}
    
    We also need to look at cases where eigenvalues can be repeated, and conditions that are needed for a matrix to be diagonalized. 
    
}











