\title{Computing the LU Factorization}
\subtitle{\SubTitleName}
\institute[]{\Course}
\author{\Instructor}
\maketitle  








\frame{\frametitle{Topics and Objectives}
\Emph{Topics} \\
\TopicStatement
\begin{itemize}
    \item the LU factorization of a matrix 
    % \item Using the $LU$ factorization to solve a system
    \item why the LU factorization works
\end{itemize}

\vspace{0.5cm}

\Emph{Objectives}\\

\LearningObjectiveStatement

\begin{itemize}
    \item compute an LU factorization of a matrix
    % \item Apply the $LU$ factorization to solve systems of equations.
    % \item Determine whether a matrix has an $LU$ factorization.
\end{itemize}

\vspace{0.25cm} 

%\Emph{Motivating Question} \\

}






\frame{\frametitle{Why We Can Compute the LU Factorization}

Suppose $ A$ can be row reduced to echelon form $ U$ without interchanging rows.
\begin{equation*}
E_p \cdots E_1 A = U 
\end{equation*}
where the $ E _{j}$ are matrices that perform elementary row operations. Because we did not swap rows, each $E_j$ happens to be lower triangular and invertible. \pause Example:
\begin{equation*}
\begin{pmatrix*}[r]
1 & 0 & 0  \\ 0 & 1 & 0  \\ 2 & 0 & 1 
\end{pmatrix*} ^{-1} 
= \begin{pmatrix*}[r]
1 & 0 & 0  \\ 0 & 1 & 0  \\ -2 & 0 & 1 
\end{pmatrix*}
\end{equation*}
Therefore,  
\begin{equation*}
A = \underbrace{E_1 ^{-1} \cdots E _{p} ^{-1} } _{ = L } U = LU.
\end{equation*}
}






\frame{\frametitle{An Algorithm for Computing LU}

    To compute the LU decomposition: 

    \begin{enumerate}
        \item Reduce $ A$ to an echelon form $ U$ by a sequence of row replacement operations, if possible. 
        \item Place entries in $ L$ such that the same sequence of row operations reduces $ L$ to $ I$.
    \end{enumerate}

   

}





\frame{\frametitle{Example}

    Compute the $LU$ factorization of $A$. $$A = \spalignmat{4 -3 -1 5;-16 12 2 -17 ; 8 -6 -12 22 }$$
}



\frame{\frametitle{Final Notes on Computing LU}

    \begin{itemize}\setlength{\itemsep}{8pt}
    \item<1-> There are other definitions of the LU factorization that you may encounter in future courses or applications. 
    \item<2-> There are several other ways of computing this decomposition. 
    \item<3-> The only row operation we use to construct $L$ and $U$: \textit{replace a row with a multiple of a row above it}.
    \item<4-> As for the other two row operations: 
    \begin{itemize}\setlength{\itemsep}{4pt}
        \vspace{4pt}
        \item Multiplying a row by a non-zero scalar is not needed. 
    \item<5-> We cannot swap rows: more advanced linear algebra and numerical analysis courses address this limitation.  
    \end{itemize}
   \end{itemize}


}


\frame{\frametitle{Summary}

    \SummaryLine \vspace{4pt}
    \begin{itemize}\setlength{\itemsep}{8pt}
        \item why we can construct $A=LU$ when $A$ can be reduced to echelon form without row swaps
        \item constructing the LU decomposition using the following process
            \begin{enumerate}
                \item reduce $ A$ to an echelon form $ U$ by a sequence of row replacement operations, if possible
                \item place entries in $ L$ such that the same sequence of row operations reduces $ L$ to $ I$
        \end{enumerate}
    \end{itemize}
    
    \vspace{4pt}
    
    
        % \item Another explanation on how to calculate the LU decomposition that students may find helpful is available from MIT OpenCourseWare: www.youtube.com/watch?v=rhNKncraJMk    


}


