\title{Homogeneous Coordinates}
\subtitle{\SubTitleName}
\institute[]{\Course}
\author{\Instructor}
\maketitle  



% \tikzstyle{E} =[rectangle, minimum width=1.5cm, minimum height=1.5cm, text centered, fill=DarkBlue!30]

% \tikzstyle{W} = [rectangle, minimum width=1.5cm, minimum height=1.5cm, text centered, fill=DarkBlue!30]

% \tikzstyle{ED} = [rectangle, minimum width=2cm, minimum height=3cm, text centered, fill=DarkBlue!30]

% \tikzstyle{arrowred} = [thick,->,>=stealth, dashed,DarkRed]
% \tikzstyle{arrowblue} = [thick,->,>=stealth, dashed,DarkBlue]


\frame{\frametitle{Topics and Objectives}
\Emph{Topics} \\
\TopicStatement
\begin{itemize}
    \item homogeneous coordinates in 2D
    \item translations and composite transforms in 2D
\end{itemize}

\vspace{0.5cm}

\Emph{Objectives}\\

\LearningObjectiveStatement

\begin{itemize}
    \item construct a data matrix to represent points in $\mathbb R^2$ using homogeneous coordinates
    \item construct and apply transformation matrices to represent composite transforms in 2D using homogeneous coordinates
\end{itemize}

% Students are not expected to be familiar with perspective projections.

\vspace{0.25cm} 
 

}





\frame{\frametitle{Motivating Questions} 

    How can we represent translations, and rotations about arbitrary points, using linear transforms? 
    \begin{itemize}\setlength{\itemsep}{4pt}
        \item transformations of the form $T(\vec x) = A\vec x$ were explored earlier in this course
        \item we introduced rotations about the origin, but not about arbitrary points
        \item we also did not explore the transform $(x,y) \to (x+h,y+k)$
    \end{itemize}}





\frame{\frametitle{Homogeneous Coordinates}

    Translations of points in $\mathbb R^n$ does not correspond directly to a linear transform. \Emph{Homogeneous coordinates} are used to model translations using matrix multiplication.  
    
    \begin{center}\begin{tikzpicture} \node [mybox](box){\begin{minipage}{0.95\textwidth} \vspace{2pt}
    Each point $(x,y)$ in $\mathbb R^2$ can be identified with the point $(x,y,H)$, $H\ne 0$, on the plane in $\mathbb R^3$ that lies $H$ units above the $xy$-plane. 
     \end{minipage}}; \node[fancytitle, right=10pt] at (box.north west) {Homogeneous Coordinates in $\mathbb R^2$}; \end{tikzpicture}\end{center}
     
    Note: we often we set $H = 1$. \\[6pt]

}





\frame{\frametitle{Homogeneous Coordinates Example}

    A translation of the form $(x,y) \to (x+h,y+k)$ can be represented as a matrix multiplication with homogeneous coordinates: 
    
    $$\spalignmat{1 0 h;0 1 k;0 0 1} \spalignmat{x;y;1} = \spalignmat{x+h;y+k;1}$$

}









\frame{\frametitle{A Composite Transform with Homogeneous Coordinates}

    Triangle $S$ is determined by three data points, $(1,1), (2,4), (3,1)$. \\[6pt]
    
    Transform $T$ rotates points by $\pi/2$ radians counterclockwise about the point $(0,1)$.
    
    \vspace{6pt}
    
    \begin{enumerate}[a)]
        \item Represent the data with a matrix, $D$. Use homogeneous coordinates. 
        \item Use matrix multiplication to determine the image of $S$ under $T$. 
        \item Sketch $S$ and its image under $T$. 
    \end{enumerate}

}



\frame{\frametitle{Summary}

    \SummaryLine \vspace{4pt}
    \begin{itemize}\setlength{\itemsep}{8pt}
        \item homogeneous coordinates
        \item constructing composite transforms that apply translations and rotations about arbitrary points in $\mathbb R^2$
    \end{itemize}
    
}
