\title{The Invertible Matrix Theorem}
\subtitle{\SubTitleName}
\institute[]{\Course}
\author{\Instructor}
\maketitle   








% STYLE SETTINGS FOR BLOCK DIAGRAMS
\tikzstyle{startstop} =[trapezium, trapezium left angle=70, trapezium right angle=110, minimum width=1cm, minimum height=1cm, text centered, fill=DarkBlue!30]
%\tikzstyle{process} = [trapezium, trapezium left angle=70, trapezium right angle=110, minimum width=1cm, minimum height=1cm, text centered, fill=orange!30]
\tikzstyle{arrow} = [thick,<-,>=stealth]



\frame{\frametitle{Topics and Objectives}
\Emph{Topics} \\
\TopicStatement
\begin{itemize} 
\item the invertible matrix theorem, which is a review/synthesis of many of the concepts we have introduced
\end{itemize}

\vspace{0.5cm}

\Emph{Objectives}\\

\LearningObjectiveStatement

\begin{itemize}
    \item characterize the invertibility of a matrix using the Invertible Matrix Theorem
    \item construct and give examples of matrices that are/are not invertible
\end{itemize}

\vspace{0.25cm} 



}


\frame{\frametitle{Equivalent Expressions}

    {\small \textit{``A synonym is a word you use when you can't spell the other one." } \\ - Baltasar Graci\'an\\} 
    
    \vspace{12pt}

    The theorem we introduce in this section of the course gives us many ways of saying the same thing. Depending on the context, some will be more convenient than others.
    
}


\frame{\frametitle{The Invertible Matrix Theorem}

\vspace{-12pt}
Let $ A$ be an $ n \times n $ matrix. These statements are all equivalent. 

%%  ENUMERATE
\begin{enumerate}[a)]
    \item $ A$ is invertible. 
    \item $ A$ is row equivalent to $ I _n$. 
    \item $ A$ has $ n$ pivotal columns (all columns are pivotal).
    \item $ A \vec x = \vec 0$ has only the trivial solution.  
    \item The columns of $ A$ are linearly independent. 
    % \item  The linear transformation $  \vec x \mapsto A \vec x$ is one-to-one. 
    \item The equation $ A \vec x =  \vec b$ has a solution for all $ \vec b \in \mathbb R ^{n}$. 
    \item The columns of $ A$ span $ \mathbb R ^{n}$. 
    % \item  The linear transformation $  \vec x \mapsto A \vec x$ is onto. 
    \item There is a  $ n \times n $ matrix $ C$ so that $ CA = I _n$ ($ A$ has a left inverse.) 
    \item There is a  $ n \times n $ matrix $ D$ so that $ AD= I _n$  ($ A$ has a right inverse.) 
    \item $ A ^{T}$ is invertible. 
\end{enumerate}
%% 

%%%%%%%%%%%%%%%%%%%%%%%%%%%%%% THEOREM THEOREM THEOREM


}

\frame{\frametitle{Invertibility and Composition}

    The diagram below gives us another perspective on the role of $A^{-1}$. 
    
    \vspace{12pt}

    \begin{center}\begin{tikzpicture}[node distance=2cm]

        \node (x) [startstop] {$\vec x$};
        \node (ax) [startstop, right of=x, xshift=2cm, yshift=-2cm] {$A\vec x$};
        \draw [arrow] (ax) |- node[anchor=west] {\small multiplication by $A$} (x);
        \draw [arrow] (x) |- node[anchor=east] {\small multiplication by $A^{-1}$} (ax);


    \end{tikzpicture}
    \end{center}

    \vspace{12pt} 
    
    The matrix inverse $A^{-1}$ transforms $Ax$ back to $\vec x$. This is because: 

    $$A^{-1} (A \vec x) = (A^{-1} A) \vec x = \hspace{2cm}$$

}






\frame{\frametitle{The Invertible Matrix Theorem: Final Notes}

    \begin{itemize} 
        \item Items (h) and (i) of the invertible matrix theorem (IMT) lead us directly to the following theorem. 

\begin{center}\begin{tikzpicture} \node [mybox](box){\begin{minipage}{0.75\textwidth}

        \item If $A$ and $B $ are $ n \times n $ matrices and $ AB=I$, then $ A$ and $ B$ are invertible, and $ B = A ^{-1} $ and $ A = B ^{-1} $.  

\end{minipage}};
\node[fancytitle, right=10pt] at (box.north west) {Theorem};
\end{tikzpicture}\end{center}

        \item The IMT is a set of equivalent statements. They divide the set of all square matrices into two separate classes: invertible, and non-invertible. 

        \item As we progress through this course, we will be able to add additional equivalent statements to the IMT (that deal with determinants, eigenvalues, etc). 

    \end{itemize} 
    
}




\frame{\frametitle{Example 1: Identifying Whether a Matrix is Invertible}
Is this matrix invertible? 
\begin{equation*}
\begin{pmatrix*}[r]
1 & 0 & -2 \\ 3 & 1 & -2 \\ 0 & -1 & -1 
\end{pmatrix*}
\end{equation*}

}


\frame{\frametitle{Example 2: Constructing an Expression for the Inverse}

    Suppose $A$ is an invertible square matrix and $$A^2 + 4A = I$$ Give an expression for $A^{-1}$. 

}



\frame{\frametitle{Example 3: Matrix Completion}

    If possible, fill in the missing elements of the matrices below with numbers so that each of the matrices are singular. If it is not possible to do so, state why. 
    
    \begin{align*}
        \spalignmat{1 0 1; 1, , 1;0 0 1}, \qquad
        \spalignmat{1, , 1; 0, 1 , 1;0 0 1}, \qquad
        \spalignmat{1 0 0; 0,1, 1;0 , , 1}
    \end{align*}

}





\frame{\frametitle{Matrix Completion Problems}

    \begin{itemize}
        \item The previous example is an example of a matrix completion problem (MCP).
        \item MCPs are great questions for recitations, midterms, exams.
        \item the \Emph{Netflix Problem} is another example of an MCP. 
    \end{itemize}
        
    \begin{center}\begin{tikzpicture} \node [mybox](box){\begin{minipage}{0.85\textwidth}

        Given a \Emph{ratings matrix} in which each entry $(i,j)$ represents the rating of movie $j$ by customer $i$ if customer $i$ has watched movie $j$, and is otherwise missing, predict the remaining matrix entries in order to make recommendations to customers on what to watch next.

    \end{minipage}}; \end{tikzpicture}\end{center}        
         
    \vspace{2.1cm}
    
    \center{\small{{\color{Grey}{\small\textit{Students are not expected to be familiar with this material. It's presented to motivate matrix completion.}}}}}


}


\frame{\frametitle{Summary}

    \SummaryLine \vspace{4pt}
    \begin{itemize}\setlength{\itemsep}{8pt}
        \item characterizing the invertibility of a matrix using the Invertible Matrix Theorem
        \item construct and give examples of matrices that are/are not invertible
    \end{itemize}
    \vspace{4pt}
    As we go through the course we will add more equivalent statements to this theorem.
}

