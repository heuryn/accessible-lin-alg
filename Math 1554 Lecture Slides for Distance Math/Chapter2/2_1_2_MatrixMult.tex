\title{Matrix Multiplication}
\subtitle{\SubTitleName}
\institute[]{\Course}
\author{\Instructor}
\maketitle

\tikzstyle{startstop} =[trapezium, trapezium left angle=70, trapezium right angle=110, minimum width=1cm, minimum height=1cm, text centered, fill=Teal!100]

\tikzstyle{io} = [trapezium, trapezium left angle=70, trapezium right angle=110, minimum width=1cm, minimum height=1cm, text centered, fill=DarkRed]

\tikzstyle{process} = [trapezium, trapezium left angle=70, trapezium right angle=110, minimum width=1cm, minimum height=1cm, text centered, fill=Gold!100]

\tikzstyle{arrow} = [thick,->,>=stealth]

\frame{\frametitle{Motivation}

    Suppose we need a linear transform of the form $T(\vec x) = A\vec x$, \onslide<2->{but $A$ is a combination of many transforms. For example: }
    \begin{align*}
        \onslide<3->{A = }\onslide<4->{\underbrace{\begin{pmatrix}
            0&-1 \\ 1&0
        \end{pmatrix}}_{rotation}}\onslide<5->{
        \underbrace{\begin{pmatrix}
            2&0 \\ 0&2
        \end{pmatrix}}_{dilation}}\onslide<6->{
        \underbrace{\begin{pmatrix}
            0&1 \\ 1&0
        \end{pmatrix}}_{reflection}
        \cdots}
    \end{align*}
    \pause 
    \onslide<7->{How can we calculate the product of two (or more) matrices? How does multiplying matrices compare to multiplying numbers? }\onslide<8->{What is matrix multiplication? }

    \vspace{12pt}
    
    \onslide<9->{
    \Emph{Objectives}\\
    \LearningObjectiveStatement \ multiply two matrices together. }    
}


% % First slide - Definition (keeping the original)
% \frame{\frametitle{Matrix Multiplication}
% \vspace{-12pt}
% \begin{center}\begin{tikzpicture} \node [mybox](box){\begin{minipage}{0.95\textwidth}
%     Let $ A $ be an $ m \times n $ matrix, and $ B$ be an $ n \times p$ matrix. \onslide<2->{ The product  $ A B  $ is an $ m \times p$ matrix,} \onslide<3->{equal to} \onslide<3->{$$ A B = A 
%     \begin{pmatrix}
%     \vec b_1 & \cdots & \vec b_p
%     \end{pmatrix} = \onslide<4->{
%     \begin{pmatrix}
%     A \vec b_1 & \cdots & A \vec b_p
%     \end{pmatrix}}
%     $$ }
%     \end{minipage}};
%     \node[fancytitle, right=10pt] at (box.north west) {Definition};
%     \end{tikzpicture}\end{center}
% }

\frame{\frametitle{Definition: Matrix Multiplication}

\vspace{-12pt}
\begin{center}\begin{tikzpicture} \node [mybox](box){\begin{minipage}{0.95\textwidth}

    Let $ A $ be an $ m \times n $ matrix, and $ B$ be an $ n \times p$ matrix. \onslide<2->{ The columns of $B$ are $\vec b_1, \vec b_2 , \cdots ,\vec b_p$. }\onslide<3->{The product  $ A B  $ is an $ m \times p$ matrix,} \onslide<4->{equal to} \onslide<5->{$$ A B = A 
    \begin{pmatrix}
    \vec b_1 & \cdots & \vec b_p
    \end{pmatrix} = 
    \begin{pmatrix}A \vec b_1 & \cdots & A \vec b_p\end{pmatrix}$$} \vspace{-16pt}
    \end{minipage}};
    \node[fancytitle, right=10pt] at (box.north west) {Definition};
    \end{tikzpicture}\end{center}

    \begin{itemize}
        \item<5-> You can think of matrix multiplication $AB$ as \onslide<6->{transforming the columns of $B$ according to a transform defined by $A$. }
        \item<7-> You may encounter other definitions in other textbooks and websites. Our definition is used later on for other theorems and procedures.  
        \item<8-> The above definition calculates the product column-by-column.         
    \end{itemize}
    % \onslide<5->{
    % \Emph{Example} \\
    
    % Compute the product. 
    
    % \begin{equation*}
    % C = AB = 
    % \spalignmat{2 0;1 1}\spalignmat{2 0 0;3 4 0}
    % \end{equation*}
    % }

}

\frame{\frametitle{Example Using the Definition}
    We want to compute
    \begin{equation*}
    C = AB = 
    \spalignmat{2 0;1 1}\spalignmat{2 0 0;3 4 0}
    \end{equation*}
    \onslide<2->{The result is a $2\times 3$ matrix, $C = \begin{pmatrix} \vec c_1 & \vec c_2 & \vec c_3\end{pmatrix}$.}
    \onslide<3->{So }
    \begin{align*}
        \onslide<3->{\vec c_1 &= A\begin{pmatrix} 2\\3\end{pmatrix} } \onslide<6->{= 2\begin{pmatrix} 2\\1\end{pmatrix}+3\begin{pmatrix} 0\\1\end{pmatrix}=}\onslide<7->{\begin{pmatrix} 4\\2\end{pmatrix} + \begin{pmatrix} 0\\3\end{pmatrix}}\onslide<8->{=\begin{pmatrix} 4\\5\end{pmatrix}}\\
        \onslide<4->{\vec c_2 &= A\begin{pmatrix} 0\\4\end{pmatrix} }\onslide<9->{= 0\begin{pmatrix} 2\\1\end{pmatrix}+4\begin{pmatrix} 0\\1\end{pmatrix}=\begin{pmatrix} 0\\0\end{pmatrix} + \begin{pmatrix} 0\\4\end{pmatrix}=\begin{pmatrix} 0\\4\end{pmatrix}}\\
        \onslide<5->{\vec c_3 &= A\begin{pmatrix} 0\\0\end{pmatrix} }\onslide<10->{= 0\begin{pmatrix} 2\\1\end{pmatrix}+0\begin{pmatrix} 0\\1\end{pmatrix}=\begin{pmatrix} 0\\0\end{pmatrix} + \begin{pmatrix} 0\\0\end{pmatrix}=\begin{pmatrix} 0\\0\end{pmatrix}}
    \end{align*}
}

\frame{\frametitle{Example Solution)}

    Final result: \pause
    $$C = AB = \begin{pmatrix} \vec c_1 & \vec c_2 & \vec c_3\end{pmatrix} = \begin{pmatrix} 4 & 0 & 0 \\ 5 & 4 & 0 \end{pmatrix}$$
    
}

\frame{\frametitle{Summary: Matrix Multiplication Definition}
\begin{itemize}
    \item<2-> The definition of matrix multiplication calculates the product column-by-column. 
    \item<3-> The definition also gives us insight into what matrix multiplication is: a transformation of the columns of one of the matrices. 
    \item<4-> You might be wondering: what other ways can we use to multiply matrices? 
    \item<5-> It turns out that not only are there many ways to multiply matrices together! And there is ongoing research on the best way to multiply matrices together! 
    \item<6-> We will introduce the row-column method for multiplying matrices, which calculates the product entry-by-entry. 
\end{itemize}
}

