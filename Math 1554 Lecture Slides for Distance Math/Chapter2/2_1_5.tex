5title{Matrix Transpose and Powers}
\subtitle{\SubTitleName}
\institute[]{\Course}
\author{\Instructor}
\maketitle   


\tikzstyle{startstop} =[trapezium, trapezium left angle=70, trapezium right angle=110, minimum width=1cm, minimum height=1cm, text centered, fill=Teal!30]

\tikzstyle{io} = [trapezium, trapezium left angle=70, trapezium right angle=110, minimum width=1cm, minimum height=1cm, text centered, fill=DarkRed!30]

\tikzstyle{process} = [trapezium, trapezium left angle=70, trapezium right angle=110, minimum width=1cm, minimum height=1cm, text centered, fill=Gold!60]

\tikzstyle{arrow} = [thick,->,>=stealth]

\frame{\frametitle{Topics and Objectives}
    \Emph{Topics} \\
    \TopicStatement
    \begin{itemize}
    
        \item the transpose of a matrix 
        
        \item matrix powers
    
    \end{itemize}
    
    \vspace{0.5cm}
    
    \Emph{Objectives}\\
    
    \LearningObjectiveStatement
    
    \begin{itemize}
    
            \item apply the matrix transpose and matrix powers to solve and analyze matrix equations
    
    \end{itemize}

}








\frame{\frametitle{The Transpose of a Matrix} 

$ A ^{T}$ is the matrix whose columns are the rows of $A$.  \\[12pt]

\Emph{Example} 
\begin{equation*}
\begin{pmatrix}
1 & 2 & 3 & 4  \\ 0 & 1 & 0 & 2 
\end{pmatrix} ^{T} = 
\end{equation*}

\Emph{Properties of the Matrix Transpose}

\begin{enumerate}\setlength\itemsep{1em}
\item  $ (A ^{T}) ^{T} =$ % A $ 
\item  $ (A+B) ^{T} = $ % A ^{T} + B ^{T}$ 
\item  $ (rA) ^{T} = $ %r A ^{T}$ 
\item   $ (AB) ^{T} = $ % B ^{T} A ^{T}$
\end{enumerate}
%% ENUMERATE

}


\frame{\frametitle{Matrix Powers} 

    For $n\times n$ matrix and positive integer $k$, $A^k$ is the product of $k$ copies of $A$. 
    
    $$A^k = A A \ldots A$$

    \Emph{Example}: Compute $C^2$.

    \begin{equation*}
        C = \spalignmat{1 2;0 1}
    \end{equation*}    
    
}



\begin{frame}
\frametitle{Example}

Define 
\begin{equation*}
A = 
\begin{pmatrix}
1 & 0 \\ 0 & 0 
\end{pmatrix} 
, \quad 
B = 
\begin{pmatrix}
1 & 0  & 0 \\ 0 & 0 & 8   
\end{pmatrix} 
, \quad 
C = 
\begin{pmatrix}
1 & 0 & 0 \\ 0 & 2 & 0  \\ 0 & 0 & 2 
\end{pmatrix} 
\end{equation*}

Which of these operations are defined, and what are the dimensions of the result? 

\begin{enumerate} \setlength\itemsep{1.5em}

    \item $A + 3 C^2$ 
    \item $A(AB)^T$ 
    \item $A + A B C B^T$ 

\end{enumerate}

\end{frame}

\frame{\frametitle{Summary}

    \SummaryLine \vspace{4pt}
    \begin{itemize}\setlength{\itemsep}{8pt}
            \item use of the matrix transpose, matrix powers, to solve and analyze matrix equations
    \end{itemize}
    For example, we can determine whether a particular expression involving matrices is defined and what the dimensions of the product will be. 
}

