
\title{The Row-Column Method for Matrix Multiplication}
\subtitle{\SubTitleName}
\institute[]{\Course}
\author{\Instructor}
\maketitle

\tikzstyle{startstop} =[trapezium, trapezium left angle=70, trapezium right angle=110, minimum width=1cm, minimum height=1cm, text centered, fill=Teal!100]

\tikzstyle{io} = [trapezium, trapezium left angle=70, trapezium right angle=110, minimum width=1cm, minimum height=1cm, text centered, fill=DarkRed]

\tikzstyle{process} = [trapezium, trapezium left angle=70, trapezium right angle=110, minimum width=1cm, minimum height=1cm, text centered, fill=Gold!100]

\tikzstyle{arrow} = [thick,->,>=stealth]

\frame{\frametitle{Motivation}

    \begin{itemize}
        \item <2-> The matrix multiplication definition calculates matrix products column-by-column. 
        \item <3-> The row-column method is commonly used for calculating matrix products by hand. 
        \item <4-> The row-column method calculates the product entry-by-entry. 
    \end{itemize} 

    \vspace{12pt}
    
    \onslide<5->{
    \Emph{Objectives}\\
    \LearningObjectiveStatement \ multiply matrices the row-column method. }    
}




\frame{\frametitle{Matrix Multiplication Using the Row-Column Method}
    Example: use the row-column method to compute
    \onslide<2->{
    \begin{equation*}
    C = A B = 
    \begin{pmatrix} 2 & 0 \\ 1 & 1 \end{pmatrix}
    \begin{pmatrix} 2 & 0 & 0 \\ 3 & 4 & 0 \end{pmatrix}
    \end{equation*}
    }
    
    \onslide<3->{
    \begin{itemize}
        \item<3-> The result $C$ will be the same matrix we calculated using the matrix multiplication definition. 
        \item<4-> We'll compute each entry using the row-column method.
        \item<5-> Highlighted rows and columns will show which numbers we're using.
    \end{itemize}
    }
}

% Computing first entry with highlighting
\frame{\frametitle{Computing $c_{11}$ (First Entry)}

    \begin{align}
        C = \begin{pmatrix}c_{11} & c_{12} & c_{13}\\c_{21} & c_{22} & c_{23} \end{pmatrix}=AB = \begin{pmatrix} \rowcolor{blue!10} 2 & 0 \\ 1 & 1 \end{pmatrix}
    \begin{pmatrix} \multicolumn{1}{>{\columncolor{green!10}}c}{2} & 0 & 0 \\ \multicolumn{1}{>{\columncolor{green!10}}c}{3} & 4 & 0 \end{pmatrix}
    \end{align}
    
    \pause
    And: 
    $$c_{11} = (2)(2) + (0)(3) = 4$$
    
    \pause
    
    \vspace{8pt}
    Result so far:
    
    \pause
    $$C = \begin{pmatrix} \mathbf{4} & ? & ? \\ ? & ? & ? \end{pmatrix}$$
}

% Computing second entry
\frame{\frametitle{Computing $c_{12}$}

    \begin{align}
    \begin{pmatrix} \rowcolor{blue!10} 2 & 0 \\ 1 & 1 \end{pmatrix}   \begin{pmatrix} 2 & \multicolumn{1}{>{\columncolor{green!10}}c}{0} & 0 \\ 3 & \multicolumn{1}{>{\columncolor{green!10}}c}{4} & 0 \end{pmatrix}
    \end{align}
    
    \pause
    And: 
    $$c_{12} = (2)(0) + (0)(4) = 0$$
    
    \vspace{8pt}
    \pause
    Result so far:
    $$C = \begin{pmatrix} 4 & \mathbf{0} & ? \\ ? & ? & ? \end{pmatrix}$$

}

% Computing third entry
\frame{\frametitle{Computing $c_{21}$}
    \begin{center}
    $\begin{pmatrix} 2 & 0 \\ \rowcolor{blue!10} 1 & 1 \end{pmatrix}$
    $\begin{pmatrix} \multicolumn{1}{>{\columncolor{green!10}}c}{2} & 0 & 0 \\ \multicolumn{1}{>{\columncolor{green!10}}c}{3} & 4 & 0 \end{pmatrix}$
    
    \vspace{8pt}
    $c_{21} = (1)(2) + (1)(3) = 5$
    
    \vspace{8pt}
    Result so far:
    $C = \begin{pmatrix} 4 & 0 & ? \\ \mathbf{5} & ? & ? \end{pmatrix}$
    \end{center}
}

% Computing fourth entry
\frame{\frametitle{Computing $c_{22}$ (Fourth Entry)}
    \begin{center}
    $\begin{pmatrix} 2 & 0 \\ \rowcolor{blue!10} 1 & 1 \end{pmatrix}$
    $\begin{pmatrix} 2 & \multicolumn{1}{>{\columncolor{green!10}}c}{0} & 0 \\ 3 & \multicolumn{1}{>{\columncolor{green!10}}c}{4} & 0 \end{pmatrix}$
    
    \vspace{8pt}
    $c_{22} = (1)(0) + (1)(4) = 4$
    
    \vspace{8pt}
    Result so far:
    $C = \begin{pmatrix} 4 & 0 & ? \\ 5 & \mathbf{4} & ? \end{pmatrix}$
    \end{center}
}

% Computing fifth and sixth entries
\frame{\frametitle{Computing Final Entries ($c_{13}$ and $c_{23}$)}
    \begin{center}
    For $c_{13}$:
    $\begin{pmatrix} \rowcolor{blue!10} 2 & 0 \\ 1 & 1 \end{pmatrix}$
    $\begin{pmatrix} 2 & 0 & \multicolumn{1}{>{\columncolor{green!10}}c}{0} \\ 3 & 4 & \multicolumn{1}{>{\columncolor{green!10}}c}{0} \end{pmatrix}$
    
    $c_{13} = (2)(0) + (0)(0) = 0$
    
    \vspace{8pt}
    For $c_{23}$:
    $\begin{pmatrix} 2 & 0 \\ \rowcolor{blue!10} 1 & 1 \end{pmatrix}$
    $\begin{pmatrix} 2 & 0 & \multicolumn{1}{>{\columncolor{green!10}}c}{0} \\ 3 & 4 & \multicolumn{1}{>{\columncolor{green!10}}c}{0} \end{pmatrix}$
    
    $c_{23} = (1)(0) + (1)(0) = 0$
    \end{center}

    \pause 
    Final result:
    $$C = AB = \begin{pmatrix} 4 & 0 & 0 \\ 5 & 4 & 0 \end{pmatrix}$$
    The same result obtained with the matrix multiplication definition.
    
}

% % Complete result
% \frame{\frametitle{Complete Matrix Multiplication}
%     \begin{center}
%     Final result:
    
%     \vspace{8pt}
%     \onslide<1->{
%     $C = AB = \begin{pmatrix} 4 & 0 & 0 \\ 5 & 4 & 0 \end{pmatrix}$
%     }
    
%     \vspace{12pt}
%     \onslide<2->{
%     \begin{tikzpicture} \node [mybox](box){\begin{minipage}{0.85\textwidth}
%     Each entry $c_{ij}$ is computed by:
%     \begin{itemize}
%         \item Taking row $i$ from matrix $A$ (highlighted in blue)
%         \item Taking column $j$ from matrix $B$ (highlighted in green)
%         \item Computing their dot product for entry $(i,j)$ in matrix $C$
%     \end{itemize}
%     \end{minipage}};
%     \node[fancytitle, right=10pt] at (box.north west) {Key Insight};
%     \end{tikzpicture}
%     }
%     \end{center}
% }

\frame{\frametitle{Theorem: The Row-Column Method}

\vspace{-12pt}
\begin{center}\begin{tikzpicture} \node [mybox](box){\begin{minipage}{0.95\textwidth}

    Let $ A $ be an $ m \times n $ matrix, and $ B$ be an $ n \times p$ matrix. \onslide<2->{ The entry in row $i$} \onslide<3->{and column $j$} \onslide<4->{of $AB$, is } \onslide<5->{the sum of the products of corresponding entries of row $i$ of $A$, and column $j$ of $B$. } \onslide<6->{That is, }
    $$\onslide<7->{(AB)_{ij} = } \onslide<8->{a_{i1}b_{1j} + }\onslide<9->{a_{i2}b_{2j} +}\onslide<10->{ a_{i3}b_{3j} + \cdots} \onslide<11->{+ a_{in}b_{nj}}$$ \vspace{-12pt}
    \end{minipage}};
    \node[fancytitle, right=10pt] at (box.north west) {Theorem};
    \end{tikzpicture}\end{center}

    \onslide<12->{\Emph{Brief explanation:} } \onslide<13->{Let $B = \begin{pmatrix}
        \vec b_1 & \vec b_2 & \cdots & \vec b_p\end{pmatrix}$. }\onslide<14->{Then by the definition of matrix multiplication, column $j$ of $AB$ is $A\vec b_j$. }\onslide<15->{ And entry $i$ of $A\vec b_j$ is found using the row-column rule.}

}


% Final summary slide
\frame{\frametitle{Summary}
    \begin{itemize}\setlength{\itemsep}{8pt}
        \item<2-> We have two ways to compute matrix multiplication: 
        \begin{itemize}
            \item {\normalsize the definition}
            \item {\normalsize the row-column method}
        \end{itemize}
        \item<3-> Often the row-column method is what we use for quick hand-calculations. 
        \item<4-> We use the definition when introducing new algorithms.
        \item<5-> There are many other ways of computing matrix products, but for now this is all we need. 
    \end{itemize}
}