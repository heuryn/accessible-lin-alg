\title{The Inverse of an $n\times n$ Matrix}
\subtitle{\SubTitleName}
\institute[]{\Course}
\author{\Instructor}
\maketitle  



\frame{\frametitle{Topics and Objectives}
\Emph{Topics} \\
\TopicStatement
\begin{itemize}
    \item an algorithm for computing the inverse of a square matrix
    % \item Elementary matrices and their role in calculating the matrix inverse.
\end{itemize}

\vspace{0.5cm}

\Emph{Objectives}\\

\LearningObjectiveStatement

\begin{itemize}
        \item compute the inverse of an $ n \times n $ matrix, and use it to solve linear systems
        % \item Construct elementary matrices.
 \end{itemize}

\vspace{0.25cm} 


}






\frame{\frametitle{The Matrix Inverse}

    Recall the following theorem. 
    \begin{center}\begin{tikzpicture} \node [mybox](box){\begin{minipage}{0.85\textwidth}
    
    $A \in \R^{n\times n}$ has an inverse if and only if for all $ \vec b \in \mathbb R ^{n}$, $ A \vec x= \vec b$ has a unique solution.  And, in this case, $ \vec  x = A ^{-1} \vec b$. 
    
    \end{minipage}};
    \node[fancytitle, right=10pt] at (box.north west) {Theorem};
    \end{tikzpicture}\end{center}
    
    This theorem gives us a method for solving a linear systems with $n$ equations and $n$ variables. But how do we construct the inverse of an $n\times n$ matrix? 


}










\frame{\frametitle{An Algorithm for Computing $A^{-1}$}

Suppose $A \in \R^{n\times n}$. We can use the following algorithm to compute $A^{-1}$. 
\begin{enumerate}
    \item Row reduce the augmented matrix $(A\,|\,I_n)$ to RREF.
    \item If reduction has form  $(I_n\,|\,B)$ then $A$ is invertible and $B=A^{-1}$. Otherwise, $A$ is not invertible.
\end{enumerate}

\vspace{12pt}
\Emph{Example} \\
Compute the inverse of $A = \begin{pmatrix*}[r]
0 & 1 & 2 \\ 1 & 0 & 3 \\ 0 & 0 & 1
\end{pmatrix*}$.

}



\begin{frame}
\frametitle{Why Does Our Algorithm Produce $A^{-1}$?}
Suppose $A$ is a $3\times3$ matrix and $A^{-1} = (\vec x_1 \ \vec x_2 \ \vec x_3)$. The first column of $A^{-1}$ is
\begin{align*}
    \vec x_1 &= A^{-1} \vec e_1 
\end{align*}
This implies: $$A\vec x_1 = \vec e_1, \quad \text{or} \quad (A \, | \, \vec e_1)$$
Thus:
\begin{itemize}
    \item If we row reduce to RREF, we obtain the first column of the inverse, $\vec x_1$. 
    \item Each column of $A^{-1}$ is found by reducing $A \vec x_i = \vec e_i$. 
\end{itemize}

\textit{}.
\end{frame}

\begin{frame}
\frametitle{Why Does This Work?}
We can think of our algorithm as simultaneously solving $n$ linear systems:
\begin{align*}
A \vec x_1 &= \vec e_1\\
A \vec x_2 &= \vec e_2 \\
 & \vdots \\
A \vec x_n &= \vec e_n
\end{align*}
Each column of $A^{-1}$ is $A^{-1} \vec e_i = \vec x_i$. \\[12pt]
\textit{Another perspective on constructing $A^{-1}$ uses elementary matrices}.
\end{frame}


\frame{\frametitle{Summary}

    \SummaryLine \vspace{4pt}
    \begin{itemize}\setlength{\itemsep}{8pt}
            \item a method for constructing the inverse of an $ n \times n $ matrix, $A^{-1}$, that could be used to solve a linear system
    \end{itemize}
    Our algorithm will have limitations but the concept of a matrix inverse is something that is widely used in applications of linear algebra, even if it is not used in practice. 
}








