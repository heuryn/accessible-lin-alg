
\title{Column Space and Null Space}
\subtitle{\SubTitleName}
\institute[]{\Course}
\author{\Instructor}
\maketitle  



\frame{\frametitle{Topics and Objectives}
\Emph{Topics} \\
\TopicStatement
\begin{itemize}
    \item the column space and null space of a matrix
\end{itemize}

\vspace{0.5cm}

\Emph{Objectives}\\

\LearningObjectiveStatement

\begin{itemize}
    \item determine whether a vector is a column or null space of a matrix, or identify a vector in that subspace
    \item construct a matrix whose column and/or null space is given
    % \item construct a basis for a subspace (for example, a basis for Col(A))
\end{itemize}


}


\frame{\frametitle{The Column Space and the Null Space of a Matrix}

    \Emph{Recall}: for $ \vec v_1 ,\dotsc, \vec v_p \in \mathbb R ^{n}$, that
$ \operatorname {Span} \{\vec v_1 ,\dotsc, \vec v_p\}$ is: the set of all possible linear combinations of the vectors $ \vec v_j$. 

    \vspace{12pt} 
    
    
    This is a \Emph{subspace}, spanned by $ \vec v_1 ,\dotsc, \vec v_p$.  

    \pause 

    % ~~ ~~ Highlight Box ~~ ~~
    \begin{center}\begin{tikzpicture} \node [mybox](box){\begin{minipage}{0.90\textwidth}\vspace{2pt}

    Given an $ m \times n $ matrix $ A = \begin{bmatrix}
    \vec a_1 & \cdots & \vec a _{n}
    \end{bmatrix}$ \vspace{2pt}
    \begin{itemize}
        \item  The \Emph{column space of $ A$}, $ \operatorname {Col} A $, is the subspace of $ \mathbb R ^{m}$ spanned by $ \vec a_1 ,\dotsc, \vec a_n$.  \vspace{2pt}
        \item The \Emph{null space of $A$}, $ \operatorname {Null} A $, is the subspace of $\mathbb R^n$ spanned by the set of all vectors $ \vec x$ that solve $ A \vec x= \vec 0$. 
    \end{itemize}
    \end{minipage}};\node[fancytitle, right=10pt] at (box.north west) {Definition};
    \end{tikzpicture}\end{center}

}






\frame{\frametitle{Example: Column Space}
Is $\vec b$ in the column space of $A$? 
$$A = \spalignmat{1 -3 -4;-4 6 -2;-3 7 6} 
% \sim \spalignmat{1 -3 -4 ;0 -6 -18; 0 0 0}
, \quad \vec b = \spalignmat{ 3 ; 3; -4}
$$
}

\frame{\frametitle{Example: Null Space}
Using the matrix on the previous slide: is $\vec v$ in the null space of $A$?  $$A = \spalignmat{1 -3 -4;-4 6 -2;-3 7 6} , \quad \vec v = \spalignmat{-5\lambda ; -3\lambda; \lambda }, \quad \lambda \in \mathbb R $$
}
 
 
 
\frame{\frametitle{Example: Column and Null Space}
Give an example of a vector in the column space of $A$, and a vector in the null space of $A$. $$A = \spalignmat{1 0 0 4;0 1 3 0;0 0 0 0}$$ 
}

\frame{\frametitle{Example Construction}
Give an example of a matrix whose column space is spanned by $\spalignmat{1;1}$ and whose null space is spanned by $\spalignmat{2;1}$. 
}
 
 



\frame{\frametitle{Summary}

    \SummaryLine \vspace{4pt}
    \begin{itemize}\setlength{\itemsep}{8pt}
        \item determining whether a vector is a column or null space of a matrix
        \item identifying vectors in the column and null space of a matrix
        \item constructing a matrix whose column and/or null space is given   
    \end{itemize}
    

}