\title{The Basis of a Subspace}
\subtitle{\SubTitleName}
\institute[]{\Course}
\author{\Instructor}
\maketitle  



\frame{\frametitle{Topics and Objectives}
\Emph{Topics} \\
\TopicStatement
\begin{itemize}
    % \item subspaces, Column space, and Null spaces 
    \item a basis for a subspace
\end{itemize}

\vspace{0.5cm}

\Emph{Objectives}\\

\LearningObjectiveStatement

\begin{itemize}
    % \item determine whether a set is a subspace
    % \item determine whether a vector is in a particular subspace, or find a vector in that subspace
    \item construct a basis for a subspace (for example, a basis for Col(A))
\end{itemize}

}


 
\frame{\frametitle{Basis}

    % ~~ ~~ Highlight Box ~~ ~~
    \begin{center}\begin{tikzpicture} \node [mybox](box){\begin{minipage}{0.80\textwidth}\vspace{2pt}

    A \Emph{basis} for a subspace $ H$ is a set of linearly independent vectors in $H$ that span $ H$. 

    \end{minipage}};\node[fancytitle, right=10pt] at (box.north west) {Definition};
    \end{tikzpicture}\end{center}
    
    \pause

    \Emph{Example} \\
    The set $ H = \{  \vec x \in \mathbb R^4 \; \large| \; x_1 - 3x_2 - 5x_3 + 7x_4=0\}$ is a subspace. 
    \begin{enumerate}[a)]
        \item $H$ is a null space for what matrix $A$?
        \item Construct a basis for $H$. 
    \end{enumerate}


}

\frame{\frametitle{Example}

    Construct a basis for Null$A$ and a basis for Col$A$.
    \begin{equation*} A=
    \begin{pmatrix*}[r]
    -3 & 6 & -1  & 0 
    \\
    1 & -2 & 2  & 0 
    \\
    2 & -4 & 5  & 0 
    \end{pmatrix*}
    \sim 
    \begin{pmatrix*}[r]
    1 & -2 & 0  & 0 
    \\
    0 & 0& 1  & 0 
    \\
    0 & 0 & 0  & 0 
    \end{pmatrix*}
    \end{equation*}
}



\frame{\frametitle{Summary}

    \SummaryLine \vspace{4pt}
    \begin{itemize}\setlength{\itemsep}{8pt}
        \item constructing a basis for a subspace
        \item constructing a basis for the column and/or null space of a matrix
    \end{itemize}
    

}