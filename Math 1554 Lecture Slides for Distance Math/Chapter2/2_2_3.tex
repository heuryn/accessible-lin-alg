\title{Elementary Matrices}
\subtitle{\SubTitleName}
\institute[]{\Course}
\author{\Instructor}
\maketitle  



\frame{\frametitle{Topics and Objectives}
\Emph{Topics} \\
\TopicStatement
\begin{itemize}
    \item the inverse of a matrix, its algebraic properties, and its relation to solving systems of linear equations
    \item elementary matrices and their role in calculating the matrix inverse
\end{itemize}

\vspace{0.5cm}

\Emph{Objectives}\\

\LearningObjectiveStatement

\begin{itemize}
        \item apply the formal definition of an inverse, and its algebraic properties, to solve and analyze linear systems
        % \item Compute the inverse of an $ n \times n $ matrix, and use it to solve linear systems.
        \item construct elementary matrices and characterize row operations with them
 \end{itemize}

\vspace{0.25cm} 


}









\frame{\frametitle{Properties of the Matrix Inverse}

$ A$ and $B$ are invertible $ n \times n $ matrices. 

\begin{itemize}
\item $ (A ^{-1} ) ^{-1} = A $ 
\item $ (AB) ^{-1} = B ^{-1} A ^{-1} $  (Non-commutative!) 
\item $ (A ^{T} ) ^{-1} = (A ^{-1} ) ^{T}$ 
\end{itemize}

\vspace{12pt} 

\Emph{Example} \\
True or false: $(ABC)^{-1} = C^{-1}B^{-1}A^{-1}$.

}








\begin{frame}
\frametitle{Elementary Matrices}

An elementary matrix, $E$, is one that differs by $I_n$ by one row operation. 

Recall our elementary row operations:  
\begin{itemize} 
    \item swap rows
    \item multiply a row by a non-zero scalar
    \item add a multiple of one row to another
\end{itemize}
We can represent each operation by a matrix multiplication with an \Emph{elementary matrix}.
\end{frame}

\begin{frame}
\frametitle{Example}
Suppose $$E \begin{pmatrix} 1&1&1\\-2&1&0\\0&0&1\end{pmatrix}= \begin{pmatrix} 1&1&1\\0&3&2\\0&0&1\end{pmatrix}$$ By inspection, what is $E$? How does it compare to $I_3$?
\end{frame}

\begin{frame}
\frametitle{Theorem}

Returning to understanding why our algorithm works, we apply a sequence of row operations to $A$ to obtain $I_n$:
\begin{align*}
    (E_k \cdots E_3E_2E_1 ) A &= I_n
\end{align*}
Thus, $E_k \cdots E_3E_2E_1 $ is the inverse matrix we seek. 

\vspace{12pt}

Our algorithm for calculating the inverse of a matrix is the result of the following theorem. 

\begin{center}\begin{tikzpicture} \node [mybox](box){\begin{minipage}{0.85\textwidth}\vspace{2pt}

Matrix $A$ is invertible if and only if it is row equivalent to the identity. In this case, the any sequence of elementary row operations that transforms $ A$ into $ I$, applied to $ I$, generates $ A ^{-1} $.  

\end{minipage}};
\node[fancytitle, right=10pt] at (box.north west) {Theorem};
\end{tikzpicture}\end{center}



\vspace{12pt} 



\end{frame}




\begin{frame}\frametitle{Using The Inverse to Solve a Linear System}

    \begin{itemize}
        \item We could use $A^{-1}$ to solve a linear system,  $$A \vec x = \vec b$$ We would calculate $A^{-1}$ and then: $\vec x = A^{-1}\vec b$.
    
        \item As our textbook points out, $A^{-1}$ is seldom used: computing it can take a very long time, and is prone to numerical error.
        \item So why did we learn how to compute $A^{-1}$? Later on in this course, we use elementary matrices and properties of $A^{-1}$ to derive results.
        \item A recurring theme of this course: just because we \Emph{can} do something a certain way, does not meant that we \Emph{should}.
    \end{itemize}
\end{frame}


\frame{\frametitle{Summary}

    \SummaryLine \vspace{4pt}
    \begin{itemize}\setlength{\itemsep}{8pt}
            \item elementary matrices and their relationship to row operations
            \item properties of the matrix inverse
    \end{itemize}
    \vspace{4pt}
    Elementary matrices are used a few times throughout this course to describe the processes behind algorithms. 
}
