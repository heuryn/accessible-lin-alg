\title{The Inverse of a $2\times 2$ Matrix}
\subtitle{\SubTitleName}
\institute[]{\Course}
\author{\Instructor}
\maketitle  








\frame{\frametitle{Topics and Objectives}
\TopicStatement
\begin{itemize}
    \item the inverse of a $2\times 2$ matrix
\end{itemize}

\vspace{0.5cm}

\Emph{Objectives}\\

\LearningObjectiveStatement

\begin{itemize}
        % \item Apply the formal definition of an inverse, and its algebraic properties, to solve and analyze linear systems. 
        \item compute the inverse of a $ 2 \times 2 $ matrix and use it to solve a linear system
        % \item Construct elementary matrices.
 \end{itemize}

\vspace{0.25cm} 


}


\frame{\frametitle{An Algorithm with Limitations}

\textit{"Your scientists were so preoccupied with whether or not they could, \\ they didn't stop to think if they should."  \\  - Spielberg and Crichton, Jurassic Park, 1993 film} \\ \vspace{12pt}
 The algorithm we introduce in this section \Emph{could} be used to compute an inverse of an $n\times n$ matrix. At the end of this section of the course, we will discuss some of the problems with our algorithm and why it can be difficult to compute a matrix inverse.  

    
}




\frame{\frametitle{The Matrix Inverse}

    \begin{center}\begin{tikzpicture} \node [mybox](box){\begin{minipage}{0.75\textwidth}

        $A \in \R^{n\times n}$ is \Emph{invertible} (or \Emph{non-singular}) if there is a $C \in \R^{n\times n}$ so that $$AC = CA = I_n.$$  If there is, we write $C= A ^{-1}$.  

    \end{minipage}};\node[fancytitle, right=10pt] at (box.north west) {Definition};
    \end{tikzpicture}\end{center}
    A matrix that is not invertible is \Emph{singular}.

}







\frame{\frametitle{The Inverse of a $2\times2$ Matrix}


There is a formula for computing the inverse of a $2\times2$ matrix. % that you may have encountered in high school. 

\begin{center}\begin{tikzpicture} \node [mybox](box){\begin{minipage}{0.90\textwidth}

The  $ 2 \times 2 $ matrix  $  \begin{pmatrix*}[r]
a & b \\ c & d 
\end{pmatrix*}$ is non-singular if and only if $ ad - bc \neq 0$, and 
\begin{equation*}
 \begin{pmatrix*}[r]
a & b \\ c & d 
\end{pmatrix*} ^{-1} = \frac 1 {ad-bc} 
 \begin{pmatrix*}[r]
d & -b \\ -c & a  
\end{pmatrix*}
\end{equation*}

\end{minipage}};
\node[fancytitle, right=10pt] at (box.north west) {Theorem};
\end{tikzpicture}\end{center}

\Emph{Example}

State the inverse of the matrix $
\begin{pmatrix*}[r]
2 & 5  \\ -3 & -7
\end{pmatrix*}
$.

}



\frame{\frametitle{Solving a Linear System}

Use a matrix inverse to solve the linear system.
\begin{align*}
3 x_1  + 4 x_2 & = 7 \\ 5x_1 + 6 x_2 & = 7
\end{align*}


}



\frame{\frametitle{Summary}

    \SummaryLine \vspace{4pt}
    \begin{itemize}\setlength{\itemsep}{8pt}
            \item use of the inverse of a $ 2 \times 2 $ matrix to solve a linear system
    \end{itemize}
    In the next set of videos we will explore a method to compute the inverse of an $n\times n$ matrix and some of its limitations. 
}



