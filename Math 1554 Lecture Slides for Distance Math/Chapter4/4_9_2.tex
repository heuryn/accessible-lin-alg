\title{Markov Chain Convergence}
\subtitle{\SubTitleName}
\institute[]{\Course}
\author{\Instructor}
\maketitle   



\frame{\frametitle{Topics and Objectives}
\Emph{Topics} \\
\TopicStatement
\begin{itemize}

    \item regular stochastic matrices and their relationship to convergence of Markov chains

\end{itemize}

\vspace{0.5cm}

\Emph{Objectives}\\

\LearningObjectiveStatement

\begin{itemize}

    % \item construct stochastic matrices and probability vectors
    
    % \item identify steady-states of Markov chains
    
    \item model and solve real-world problems using Markov chains (e.g. - find a steady-state vector for a Markov chain)
    
    \item determine whether a stochastic matrix is regular 
    
\end{itemize}

\vspace{0.25cm} 

}




\begin{frame}\frametitle{Convergence}

    We often want to know what happens to a process, 
    
    \begin{align}
        \vec x_{k+1} = P\vec x_k, \quad k = 0, 1, 2, \ldots 
    \end{align}
    
    as $k \rightarrow \infty$. 
    
    \vspace{12pt} 
    We may want to know, for example, if the sequence generated by (1),
    $$\vec x_1, \vec x_2, \vec x_3, \ldots$$
    will converge to a steady-state, and if so, what those steady-state vectors are. 
    
\end{frame}



\begin{frame}\frametitle{Regular Stochastic Matrices}

    \begin{center}\begin{tikzpicture} \node [mybox](box){\begin{minipage}{0.85\textwidth}\vspace{4pt}
    
    A stochastic matrix $P$ is \Emph{regular} if there is some $k$ such that $P^k$ only contains strictly positive entries.
    
    \end{minipage}};
    \node[fancytitle, right=10pt] at (box.north west) {Definition};
    \end{tikzpicture}\end{center}
    
    \vspace{12pt}
    \pause 
    
    This matrix is regular stochastic:
    \begin{align*}
        A &= \spalignmat{0.1 0.7;0.9 0.3} = \frac{1}{10} \spalignmat{1 7;9 3} 
    \end{align*}
    
\end{frame}


\begin{frame}\frametitle{Another Example of a Regular Stochastic Matrix}

    Another example of a regular stochastic matrix:
    \begin{align*}
        B & = \frac{1}{10}\spalignmat{0 1 2;2 8 7;8 1 1} 
    \end{align*}
    
    \pause 
    
    Note that
    \begin{align*}
        B^2 & = \frac{1}{100}\spalignmat{18 10 9;72 73 67;10 17 24} 
    \end{align*}
    
    \pause 
    Because $B^k$ has strictly positive entries for $k=2$, $B$ is regular stochastic. 
    
    \vspace{6pt}
    \pause
    It can be very difficult to determine whether a matrix is regular stochastic. 
\end{frame}



\begin{frame}\frametitle{Convergence and Regular Stochastic Matrices}

    \begin{center}\begin{tikzpicture} \node [mybox](box){\begin{minipage}{0.85\textwidth}\vspace{4pt}
    
    If $P$ is a regular stochastic matrix, then $P$ has a unique steady-state vector $\vec q$, and $\vec x_{k+1} = P\vec x_k$ converges to $\vec q$ as $k \rightarrow \infty$. 
    
    \end{minipage}};
    \node[fancytitle, right=10pt] at (box.north west) {Theorem};
    \end{tikzpicture}\end{center}

\end{frame}


\begin{frame}\frametitle{Example: Car Rental Company}

    A car rental company has 3 rental locations, A, B, and C. Cars can be returned at any location. The table below gives the pattern of rental and returns for a given week.
    
    \begin{table}[]
    \centering
    \label{my-label}
    \begin{tabular}{lllll}
                      &  & \multicolumn{3}{l}{rented from} \\\hline
                      &  & A & B & C \\\hline
    \multirow{3}{*}{returned to} & A & .8 & .1  & .2  \\
                      & B & .2 & .6 & .3     \\
                      & C & .0 & .3 & .5   \\\hline
    \end{tabular}
    \end{table}
        
     There are 1000 cars at each location today. 
     \begin{enumerate}[a)]
        \item Construct a stochastic matrix, $P$, for this problem.
        \item What happens to the distribution of cars after a long time? You may assume that $P$ is regular. 
     \end{enumerate}

\end{frame}




%% FRAME
\begin{frame}\frametitle{Solution to Part (a): Set of Equations}
    \begin{table}[]
    \begin{tabular}{lllll}
                      &  & \multicolumn{3}{l}{rented from} \\\hline
                      &  & A & B & C \\\hline
    \multirow{3}{*}{returned to} & A & .8 & .1  & .2  \\
                      & B & .2 & .6 & .3     \\
                      & C & .0 & .3 & .5   \\\hline
    \end{tabular}
    \end{table}
    
    \pause 
    
    If $x_{A,k}, x_{B,k}, x_{C,k}$ are the number of cars in week $k$ at locations $A,B,C$ respectively, then after one week,
    \begin{align*}
        \onslide<2->{x_{A,1} & = 0.8x_{A,0} + 0.1x_{B,0} + 0.2x_{C,0} \\ }
        \onslide<3->{x_{B,1} & = 0.2x_{A,0} + 0.6x_{B,0} + 0.3x_{C,0}  \\ }
        \onslide<4->{x_{C,1} & = 0.0x_{A,0} + 0.3x_{B,0} + 0.5x_{C,0} }
    \end{align*}
    
\end{frame}

\begin{frame}\frametitle{Solution to Part (a): Matrix $P$}

    Our set of equations can be represented with a matrix equation, \pause
        \begin{align*}
        x_{A,1} & = 0.8x_{A,0} + 0.1x_{B,0} + 0.2x_{C,0} \\ 
        x_{B,1} & = 0.2x_{A,0} + 0.6x_{B,0} + 0.3x_{C,0} \quad \Rightarrow \quad \vec x_1 = P\vec x_0 \\
        x_{C,1} & = 0.0x_{A,0} + 0.3x_{B,0} + 0.5x_{C,0} 
    \end{align*}
    where
    \pause 
    $$P = \begin{pmatrix}
.8 & .1 & .2 \\ .2 & .6 & .3 \\ .0 & .3 & .5 
\end{pmatrix}
$$
\end{frame}




%% FRAME
\begin{frame}\frametitle{Solution to Part (a): Another Perspective}
    Another approach to determining $P$ involving a graph. 
    
    \vspace{12pt}
    
    \begin{tikzpicture}
    \begin{scope}[->,>=stealth',shorten >=1pt,auto,node distance=3.5cm,
      thick,main node/.style={circle,fill=DarkBlue!10,draw,font=\sffamily\bfseries}]
      \node[main node] (1) {A};
      \node[main node] (2) [right of=1] {B};
        \node[main node] (3) [below right of=1] {C};
      \path[every node/.style={font=\sffamily\small}]
        (1) edge [bend right] node[below] {.2} (2)
            edge [loop left] node {.8} (1)
        (2) edge node [above] {.1} (1)
            edge [loop right] node {.6} (2)
            edge [left] node {.3}  (3) 
         (3) edge node [bend left] {.2} (1)
            edge [loop below] node {.5} (3)
            edge [bend right] node[right] {.3}  (2);
    \end{scope}
    \draw (7,-1) node {
    $ \displaystyle 
    P = 
    \begin{pmatrix}
    .8 & .1 & .2 \\ .2 & .6 & .3 \\ .0 & .3 & .5 
    \end{pmatrix}
    $ }; 
        \end{tikzpicture}    

\end{frame}

\begin{frame}\frametitle{Solution to Part (b)}

    Part b: \textit{What happens to the distribution of cars after a long time? You may assume that $P$ is regular. }
    
    
    \vspace{12pt}
    \onslide<2->{$P$ is regular $\Rightarrow$ can assume that $P$ has a unique steady-state vector $\vec q$, and $\vec x_{k+1} = P\vec x_k$ converges to $\vec q$ as $k \rightarrow \infty$. }
    
    \vspace{12pt}
    \onslide<3->{The steady-state vector, $\vec q$ is given by: }
    \begin{align*}
        \onslide<4->{P\vec q &= \vec q \\}
        \onslide<5->{P\vec q - \vec q &= \vec 0 \\}
        \onslide<6->{(P - I)\vec q &= \vec 0 }
    \end{align*}
    \onslide<7->{$\vec q$ is a probability vector in Null$(P-I)$. We need to calculate $P-I$ and $\vec q \ldots $}
\end{frame}

\frame{
$$P-I = \begin{pmatrix}
    .8 -1 & .1 & .2 \\ .2 & .6-1 & .3 \\ .0 & .3 & .5 -1
    \end{pmatrix} = \begin{pmatrix}
    -.2 & .1 & .2 \\ .2 & -.4 & .3 \\ .0 & .3 & -.5
    \end{pmatrix}
    $$
    \pause
$\vec q$ is a vector in the null space of the above matrix. We apply the usual process for finding a vector in the null space of a matrix. 
\pause
    \begin{align*}
        \spalignaugmat{-.2 .1 .2 0;.2 -.4 .3 0 ; 0 .3 -.5 0} \sim \spalignaugmat{-2 1 2 0;0 -3 5 0 ; 0 3 -5 0} \sim \spalignaugmat{6 0 -11 0;0 -3 5 0 ; 0 0 0 0}
    \end{align*}
    $x_3$ is free. If we set $x_3 = 6$, then 
    \pause 
    \begin{align*}
        6x_1 -11x_3 = 0 &\Rightarrow x_1 = 11 \\
        -3x_2 +5x_3 = 0 &\Rightarrow x_2 = 10
    \end{align*}
    Therefore ... 
}    
\frame{    
    A vector in the null space of $P-I$ is 
    \pause
    $$\spalignmat{11;10;6}$$
    \pause
    A probability vector in the null space is
    $$\vec q = \frac{1}{27} \spalignmat{11;10;6}$$
    This is our steady-state vector. 
    
    \vspace{12pt}
    \pause
    No matter what the initial distribution of cars happens to be, after a long period of time, the distribution of cars is given by $\vec q$. 
}

% \begin{frame}\frametitle{Convergence in $\mathbb R^2$}
%     The stochastic vectors in the plane are  the line segment below, and a stochastic matrix maps stochastic vectors to themselves. Iterates $ P ^{k} \vec x_0$ converge to the steady state. 
    
%     \vspace{12pt} 
    
%     \begin{tikzpicture}[scale=0.9]
%     \draw[->] (-0.5,0) -- (4,0);  \draw[->] (0,-0.5) -- (0,4); 
%     \draw (3,0) node[below] {1} -- (0,3) node[left] {1};  
%     \filldraw  (2,1) circle (.1em) node (S) {}; 
%     \draw[<<-] (2,1) node[below,left] {$\color{DarkRed} \vec x _{\infty }$}  --  (4,2) node [above] {Steady State Vector}; 
%     \foreach \x/\y/\k in { 3/0/0 , 1/2/1, 2.5/0.5/2, 1.5/1.5/3} \filldraw (\x,\y) circle (0.1em) node[above] {$ \vec x_{\k}$}; 
%     \draw (8,1) node {  $ P ^{k} \longrightarrow \begin{bmatrix}
%     \vec x _{\infty } & \vec x _{\infty }
%     \end{bmatrix}$}; 
%     \end{tikzpicture}
% \end{frame} 

% \begin{frame}\frametitle{Convergence in $\mathbb R^3$}
% The Stochastic vectors in $ \mathbb R ^{3}$, are vectors $ \begin{bmatrix}
% s \\ t \\ 1- s-t 
% \end{bmatrix}$, where $ 0 \leq s , t \leq 1$ and $ s+t \leq 1$.  $ P$ \Emph{contracts} stochastic vectors to $\vec x _{\infty }$. 


% \begin{center}
% \begin{tikzpicture}[scale=0.9]
% \draw (0,3.5) node[above] {$ (1,0,0)$} --  (3,0) node[below] {$ (0,1,0)$} -- 
%  (-3,0) node[below] {$ (0,0,1)$} --  (0,3.5);  
% \filldraw (1,1) circle (.15em) node[above] {$\color{DarkRed} \vec x _{\infty }$}; 
% \end{tikzpicture}
% \end{center}
% \end{frame}
    
    
\frame{\frametitle{Summary}

    \SummaryLine \vspace{4pt}
    \begin{itemize}\setlength{\itemsep}{8pt}
        \item regular stochastic matrices and their relationship to convergence of Markov chains
    \end{itemize}
    % Later in our course we will take another look at Markov chains and their relationship to eigenvalue problems and other applications. 
    
}

\frame{

}