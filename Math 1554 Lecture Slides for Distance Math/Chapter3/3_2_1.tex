\title{Determinants and Row Operations}
\subtitle{\SubTitleName}
\institute[]{\Course}
\author{\Instructor}
\maketitle   


\frame{\frametitle{Topics and Objectives}
\Emph{Topics} \\
\TopicStatement
\begin{itemize}

    \item the relationships between row reductions, the invertibility of a matrix, and determinants
    
\end{itemize}

\vspace{0.5cm}

\Emph{Objectives}\\

\LearningObjectiveStatement

\begin{itemize}

    \item apply properties of determinants (related to row reductions, transpose, and matrix products) to compute determinants
    \item use determinants to determine whether a square matrix is invertible

\end{itemize}

\vspace{0.25cm} 

}


\frame{\frametitle{Understanding Relationships Between Ideas}

\textit{``A problem isn't finished just because you've found the right answer."} \\ - Y\={o}ko Ogawa   \\ \vspace{12pt} \pause 
 We have a method for computing determinants, but without some of the strategies we explore in this section, the algorithm can be very inefficient.   

}


\begin{frame}\frametitle{How Can we Compute Determinants More Efficiently?} 

\begin{itemize}
    \item<1-> We saw how determinants are difficult or impossible to compute with a cofactor expansion for large $N$.
    \item<2-> Row operations give us a more efficient way to compute determinants.
\end{itemize}

\end{frame}


\begin{frame}\frametitle{Row Operations} 

%%%%%%%%%%%%%%%%%%%%%%%%%%%%%% THEOREM THEOREM THEOREM 
\begin{center}\begin{tikzpicture} \node [mybox](box){\begin{minipage}{0.80\textwidth} \vspace{4pt}
Let $ A$ be a square matrix. 

%%  ENUMERATE
\begin{enumerate}
\item<1-> If a multiple of a row of $ A$ is added to another row to produce $ B$, then 
$ \operatorname {det} B = \operatorname {det} A $. 

\item<2-> If two rows are interchanged to produce $ B$, then $  \operatorname {det} B = -\operatorname {det} A $. 

\item<3-> If one row of $ A$ is multiplied by a scalar $ k$ to produce $ B$, then 
$  \operatorname {det} B = k\operatorname {det} A $. 
\end{enumerate}
%% ENUMERATE

 \end{minipage}}; \node[fancytitle, right=10pt] at (box.north west) {Theorem: Row Operations and the Determinant}; \end{tikzpicture}\end{center}
 %%%%%%%%%%%%%%%%%%%%%%%%%%%%%% THEOREM THEOREM THEOREM



 
\end{frame}


\begin{frame}
\frametitle{Using Row Operations to Compute a $3\times3$ Determinant}  
Compute $ \det A = \begin{vmatrix*}[r]
1 & -4 & 2 \\ -2 & 8 & -9 \\ -1 & 7 & 0
\end{vmatrix*}$.

\vspace {2cm} 



\end{frame}

\begin{frame}\frametitle{Invertibility}
    Important practical implication:  if $ A$ is reduced to echelon form,
 by $\textcolor{DarkRed}{r}$ interchanges of rows and columns, then 
 \begin{equation*}
\lvert  A\rvert = 
\begin{cases}
(-1) ^{\textcolor{DarkRed}{r}} \times \textup{(product of pivots)},  & \textup{when $ A$ is invertible} 
\\
0, & \textup{when $ A$ is singular} 
\end{cases}
\end{equation*}


\end{frame}



% \begin{frame}\frametitle{Using Row Operations to Compute a $4\times4$ Determinant} 

%     Compute the determinant 
%     \begin{equation*}
%     \begin{vmatrix*}[r]
%         0 & 1 & 2 & -1 \\ 2 & 5 & -7 & 3 \\ 0 & 3 & 6 &  2 \\ -2 & -5 & 4 & 2 
%     \end{vmatrix*}
%     \end{equation*}
    
% \end{frame}



\frame{\frametitle{Summary}

    \SummaryLine \vspace{4pt}
    \begin{itemize}\setlength{\itemsep}{8pt}
        \item relationships between row reductions, the invertibility of a matrix, and determinants
    \end{itemize}
    

}


