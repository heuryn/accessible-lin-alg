\title{Parametric Vector Forms for Inhomogeneous Systems} 
\subtitle{\SubTitleName}
\institute[]{\Course}
\author{\Instructor}
\maketitle   


\begin{frame}
\frametitle{Motivating Questions}

     
    \onslide<2->{
    \begin{center}\begin{tikzpicture} \node [mybox](box){\begin{minipage}{0.75\textwidth} \vspace{2pt}
                \onslide<2->{How can we express the set of solutions to a consistent and \textbf{inhomogeneous} system in a concise form?} \onslide<3->{How does the solution set compare to the homogeneous case?  }
    \end{minipage}};
    \end{tikzpicture}\end{center}
    }
    \onslide<4->{

    \vspace{6pt}
    
    \LO\\

    % \LearningObjectiveStatement

    \begin{itemize}
        % \item express the solution set of a linear equation in parametric vector form
        \item Obtain a solution of an \textbf{inhomogeneous} linear system in parametric vector form \onslide<5->{and give a geometric interpretation of the solution set.}
        % \item Characterize homogeneous linear systems using the concepts of free variables, span, pivots, linear combinations, and echelon forms.
    \end{itemize}

    }
\end{frame}

\begin{frame}
\frametitle{Homogeneous System in $\mathbb{R}^3$}
Consider the system: 
\onslide<2->{
\begin{align*}
    x_1 + x_2 &= 0 \\
    x_2 + x_3 &= 0
\end{align*}}\onslide<3->{Each equation is a plane,}\onslide<4->{ the solution set is the line where they intersect.}
\onslide<5->{Earlier we found the parametric vector form for the solution set:}\onslide<5->{
\begin{align*}
    \vec{x} = x_3 \begin{pmatrix} 1 \\ -1 \\ 1 \end{pmatrix}
\end{align*}}
\onslide<6->{All points in the solution set have the form $(t,-t,t)$, with $t=x_3$.}
\end{frame}


\begin{frame}
\frametitle{Inhomogeneous System}
Now consider the \textbf{inhomogeneous} system:
\begin{align*}
    x_1 + x_2 &= 3k \\
    x_2 + x_3 &= k
\end{align*}
\onslide<2->{
Then:
\begin{itemize}
    \item <3-> The system is homogeneous when $k=0$.
    \item <4-> How does the solution set vary as we change $k$? 
    \item <5-> What is the relationship between the solution sets of the \textbf{homogeneous} and \textbf{inhomogeneous} systems? 
\end{itemize}
}
\end{frame}



\begin{frame}
\frametitle{Inhomogeneous System}
Our system:
\begin{align*}
    x_1 + x_2 &= 3k \\
    x_2 + x_3 &= k
\end{align*}
\onslide<2->{
Expressing as an augmented matrix and row reduce:}
\begin{align*}
    \onslide<3->{\begin{amatrix}{3} 1 & 1 & 0 & 3k \\\ 0 & 1 & 1 & k \end{amatrix} }
    \onslide<4->{& \sim \begin{amatrix}{3} 1 & 0 & -1 & 2k \\\ 0 & 1 & 1 & k \end{amatrix} } \onslide<5->{\ \text{using } R_1-R_2 \to R_1}
\end{align*}
\onslide<6->{ 
Solving for the basic variables:}
\begin{align*}
    \onslide<7->{x_1 +0x_2-x_3 = 2k } \onslide<8->{\ \Rightarrow \ x_1 &= 2k + x_3} \\
    \onslide<9->{0x_1 + x_2 + x_3 = k } \onslide<10->{\ \Rightarrow \ x_2 &= k - x_3}
\end{align*}
\end{frame}

\begin{frame}
\frametitle{Solving for Basic Variables}
Expressions for the basic variables:
\begin{align*}
    x_1 &= 2k + x_3 \\
    x_2 &= k - x_3 
\end{align*}
\onslide<3->{Note that $x_3$ is free. } \onslide<4->{Expressing in parametric vector form:}
\begin{align*}
    \onslide<5->{\vec{x} 
        = \begin{pmatrix} x_1\\x_2\\x_3 \end{pmatrix} 
        = }
        \onslide<6->{\begin{pmatrix} \onslide<7->{2k+x_3}\\\onslide<8->{k-x_3}\\\onslide<9->{0+x_3} \end{pmatrix} }
        \onslide<10->{= \begin{pmatrix} 2k \\ k \\ 0 \end{pmatrix} + x_3 } \onslide<10->{\begin{pmatrix} \onslide<11->{1} \\ \onslide<12->{-1} \\ \onslide<13->{1} \end{pmatrix}}
\end{align*}
\onslide<12->{This is the parametric vector form for the solution set.}
\end{frame}


\begin{frame}
\frametitle{Interpreting the Solution Set}
\onslide<2->{Our solution set:}
\begin{align*}
    \onslide<3->{\vec{x} 
        = \begin{pmatrix} x_1\\x_2\\x_3 \end{pmatrix} 
         = }
         \onslide<4->{\begin{pmatrix} 2k \\ k \\ 0 \end{pmatrix}}
        \onslide<4->{+ x_3 \begin{pmatrix} 1 \\ -1 \\ 1 \end{pmatrix}}
        \onslide<5->{=\vec p + \vec x_h}
\end{align*}
\onslide<6->{Note that:}
\begin{itemize}
    \item<7->$\vec p$ is a constant vector, and is called a \Emph{particular solution} 
    \item<8->$\vec x_h$ is the set solutions to the homogeneous problem $A\vec x = \vec 0$
    \item<9-> The solution set $\vec x = \vec p + \vec x_h$ are a shifted version of the solutions to $A\vec x = \vec 0$. 
\end{itemize}
    
\end{frame}


\begin{frame}
    \frametitle{Geometric Interpretation of the Solution Set}
    We found:
\begin{align*}
    \onslide<2->{\vec{x} 
        = \begin{pmatrix} x_1\\x_2\\x_3 \end{pmatrix} 
        = }
        \onslide<2->{ \begin{pmatrix} 2k \\ k \\ 0 \end{pmatrix}} \onslide<3->{+ x_3 }\onslide<4->{\begin{pmatrix} 1 \\ -1 \\ 1 \end{pmatrix}}
\end{align*}    
    \onslide<5->{The solution set:}
    \begin{itemize}
        \item<6-> is a translated version of the homogeneous solution set, and
        \item<7-> the solution set does not include the origin. 
    \end{itemize}

    \vspace{4pt}
    \onslide<8->{
    \begin{center}\begin{tikzpicture} \node [mybox](box){\begin{minipage}{0.9\textwidth}\vspace{0pt}
        The solution set is a line in $\mathbb R^3$ that \Emph{does not} pass through the origin. 
    \end{minipage}};
    % \node[fancytitle, right=10pt] at (box.north west) {Theorem};
    \end{tikzpicture}\end{center}
    }
\end{frame}



\begin{frame}
    \frametitle{The Solution Set of $A\vec x = \vec b$}
    \vspace{-24pt}
    \begin{center}\begin{tikzpicture} \node [mybox](box){\begin{minipage}{0.98\textwidth}\vspace{0pt}
        \onslide<2->{Suppose $A\vec x = \vec b$ is consistent, and $\vec x = \vec p$ is a solution.} \onslide<3->{Then the solution set of $A\vec x = \vec b$ is the set of all vectors of the form } \onslide<4->{$$\vec x = \vec p + \vec x_h$$ where } \onslide<5->{$\vec p$ is any particular solution, and $\vec x_h$ is the set of all solutions of $A\vec x = \vec 0$.}
    \end{minipage}};
    \node[fancytitle, right=10pt] at (box.north west) {Theorem};
    \end{tikzpicture}\end{center}

    \onslide<6->{Note:}
    \begin{itemize}
        \item<7-> The theorem only applies to consistent systems. 
        \item<8-> $A\vec x = \vec 0$ is always consistent.
        % \item<9-> If $\vec x = \vec p + \vec x_h$ is a solution, $\onslide<10->{A\vec x = } \onslide<11->{ A(\vec p + \vec x_h) = } \onslide<12->{A\vec p + A\vec x_h =}\onslide<13->{ A\vec p + \vec 0 = } \onslide<14->{\vec b}$ \onslide<14->{as required. }
    \end{itemize}
\end{frame}







\frame{\frametitle{Summary}

\SummaryLine \vspace{4pt}
\begin{itemize}\setlength{\itemsep}{8pt}
    \item expressing the solution set of an inhomogeneous linear system in parametric vector form
    \item the geometric relationship between the solution to $A\vec x = \vec b$ and $A\vec x = \vec 0$
\end{itemize}

}



% ADDITIONAL EXAMPLES

% \begin{frame}
% \frametitle{Parametric Vector Form for the Non-homogeneous Case}
%     Write the solution as a sum of vectors. 
%     \pause
%     \begin{align*}
%         2x_1 + x_2 &= 4 \\
%         4x_1 + 2x_2 &= 8
%     \end{align*}
% \end{frame}

% \begin{frame}
% \frametitle{Row Reduce}
%     Expressing the system as an augmented matrix and row reducing yields:\pause 
%     \begin{align*}
%         \begin{pmatrix} 2&1&4\\4&2&8\end{pmatrix} 
%         &\sim \begin{pmatrix} 2&1&4\\0&0&0\end{pmatrix}, \ R_2 - 2R_1 \to R_2
%     \end{align*}
%     \pause 
%     This gives us one equation: $2x_1 + x_2 = 4$. \pause
    
%     Therefore $x_2$ is free, and $x_1 = 2 - \frac{1}{2}x_2$.
%     Expressing as a vector equation: 
%     $$\begin{pmatrix} \vec x = \begin{pmatrix} x_1 \\ x_2 \end{pmatrix}\end{pmatrix} = \begin{pmatrix} 2-x_2/2 \\x_2 \end{pmatrix}$$
% \end{frame}

% \begin{frame}
% \frametitle{Parametric Vector Form}
%     \onslide<2->{We found that $x_2$ is free and $x_1 = 2 - \frac{1}{2}x_2$} 
%     \onslide<3->{Our goal is to now express the solution set as a sum of vectors.} 
%     \onslide<4->{Note that solutions to the system will be vectors in $\mathbb R^2$. So if $\vec x$ is a solution, then}
%     $$\onslide<5->{ \vec x = \begin{pmatrix} x_1\\x_2\end{pmatrix}}$$
% \end{frame}

% \begin{frame}
% \frametitle{Parametric Vector Form for the Non-homogeneous Case}

%     Write the solution as a sum of vectors. In other words, express the solution in parametric vector form. Give a geometric interpretation of the solution.
%     \begin{align*}
%         x_1 + 3x_2 + x_3 &=4 \\
%         2x_1 -x_2 - 5x_3 &= 1 \\
%         x_1 - 2x_3 &=1
%     \end{align*}

%     \textit{Note that the left-hand side is the same as a previous example}.  

% \end{frame}


% \begin{frame}
% \frametitle{Row Reduce}

%     Expressing the system as an augmented matrix and row reducing yields:
%     \begin{align*}
%         \begin{pmatrix} 1&3&1&4\\2&-1&-5&1\\1&0&-2&1\end{pmatrix} 
%         &\sim \begin{pmatrix} 1&3&1&4\\0&-7&-7&-7\\1&0&-2&1\end{pmatrix}, \ R_2 - 2R_1 \to R_2 \\
%         &\sim \begin{pmatrix} 1&3&1&4\\0&1&1&1\\0&-3&-3&-3\end{pmatrix} \\
%         &\sim \begin{pmatrix} 1&0&-2&1\\0&1&1&1\\0&0&0&0\end{pmatrix} 
%     \end{align*}
%     Therefore $x_3$ and $x_4$ are free, $x_1 = 2x_3-x_4$, and $x_2 = -x_3-x_4$.

% \end{frame}


% \begin{frame}
% \frametitle{Parametric Vector Form}

%     \onslide<2->{We found that $x_3$ and $x_4$ are free, $x_1 = 2x_3-x_4$, and $x_2 = -x_3-x_4$} \onslide<3->{Our goal is to now express the solution set as a sum of vectors. } \onslide<4->{Note that solutions to the system will be vectors in $\mathbb R^4$. So if $\vec x$ is a solution, then}
%     $$\onslide<5->{\vec x = \begin{pmatrix} x_1\\x_2\\x_3\\x_4\end{pmatrix}}$$

% \end{frame}
