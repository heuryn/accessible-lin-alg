\title{Existence and Uniqueness} 
\subtitle{\SubTitleName}
\institute[]{\Course}
\author{\Instructor}
\maketitle   




\frame{\frametitle{Topics and Learning Objectives}
\Emph{Topics} \\
\TopicStatement


\begin{itemize}
    \item consistency, existence, uniqueness
    % \item row reduction algorithm
    \item pivots, and basic and free variables 
    %\item Parametrized solution sets 
    % \item Echelon forms, existence and uniqueness 
\end{itemize}

\vspace{0.5cm}

\LO\\

\LearningObjectiveStatement

\begin{itemize}
    % \item Characterize a linear system in terms of the number of leading entries, free variables, pivots, pivot columns, pivot positions.
    \item determine whether a linear system is consistent from its echelon form
    \item apply the row reduction algorithm to compute the coefficients of a polynomial
\end{itemize}

}



\frame{\frametitle{Basic and Free Variables}
Consider the augmented matrix
\begin{align*} \begin{amatrix}{1} A & \vec b \ \end{amatrix} = &
\begin{amatrix}{5}
1 & 3 & 0 & 7 & 0 & 4 
\\
0 & 0 & 1 & 4 & 0 & 5 
\\
0 & 0 & 0 & 0 & 1 & 6 
\end{amatrix}
\end{align*}
The leading one's are in first, third, and fifth columns.
\begin{itemize}
    \item<2-> The pivot columns of $A$ are the first, third, and fifth columns
    \item<3-> The corresponding variables of the system $A\vec x = \vec b$ are $x_1$, $x_3$, and $x_5$. Variables that correspond to a pivot are \Emph{basic variables}.
    \item<4-> Variables that are not basic are \Emph{free variables}. They can take any value. 
    \item<5-> The free variables are $x_2$ and $x_4$. Any choice of the free variables leads to a solution of the system.  
\end{itemize}
\vspace{4pt}
}



\frame{\frametitle{Notes on Basic and Free Variables}


\begin{itemize}
    \item <1-> Note that a matrix, on its own, does not have basic variables or free variables. Systems have variables. 
    \item <2-> If $A$ has $n$ columns, then the linear system $$\begin{amatrix}{1} A & \vec b \ \end{amatrix}$$ must have $n$ variables. One variable for each column of the matrix. 
    \item <3-> There are two types of variables: basic and free. And a variable cannot be both free and basic at the same time. 
    \begin{align*}
    n 
    &= \text{number of columns of } A \\
    &= (\text{number of basic variables}) + (\text{number of free variables}) 
    \end{align*}
\end{itemize}

}




\frame{\frametitle{Existence and Uniqueness}

\begin{center}\begin{tikzpicture} \node [mybox](box){\begin{minipage}{0.90\textwidth}\vspace{4pt}

A linear system is consistent if and only if (exactly when) the last column of the \Emph{augmented} matrix does not have a pivot. This is the same as saying that the RREF of the augmented matrix does \Emph{not} have a row of the form
$$\spalignmat{0 , 0, 0,  \cdots, 0 | 1}$$
Moreover, if a linear system is consistent, then it has
\begin{enumerate}
    \item a unique solution if and only if there are no free variables, and
    \item infinitely many solutions that are parameterized by free variables. 
\end{enumerate}
\end{minipage}};
\node[fancytitle, right=10pt] at (box.north west) {Theorem};
\end{tikzpicture}\end{center}

}




\frame{\frametitle{Example: Existence and Uniqueness}

    If possible, determine the coefficients of the polynomial $y(t) = a_0t+a_1t^2$ that passes through the points that are given in the form $(t,y)$. 
    \begin{enumerate}[a)]\setlength{\itemsep}{12pt}
        \item $L(-1,0)$ and $M(1,1)$
        \item $P(2,0)$, $Q(1,1)$, and $R(0,2)$
    \end{enumerate}


}


\frame{\frametitle{Summary: Fundamental Questions}

In this video we explored the following concepts. \vspace{4pt}
\begin{itemize}\setlength{\itemsep}{8pt}
    \item augmented matrices and consistent systems
    \item pivots, and basic and free variables 
    \item fundamental questions that we will revisit throughout the course regarding consistency, existence, uniqueness
\end{itemize}

}
