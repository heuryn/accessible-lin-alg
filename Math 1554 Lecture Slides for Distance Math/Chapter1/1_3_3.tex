\title{Span} 
\subtitle{\SubTitleName}
\institute[]{\Course}
\author{\Instructor}
\maketitle   




\frame{\frametitle{Topics and Learning Objectives}

\Emph{Topics} \\
\TopicStatement

\begin{itemize}
    % \item  Vectors in $ \mathbb R ^{n}$, and their basic properties 
    
    % \item linear combinations of vectors
    \item the span of a set of vectors

\end{itemize}

\vspace{0.5cm}

\LO\\

\LearningObjectiveStatement

\begin{itemize}
    % \item  Apply geometric and algebraic properties of vectors in $ \mathbb R ^{n}$ to compute vector additions and scalar multiplications.
    \item  characterize a set of vectors in terms of \Emph{linear combinations} and their \Emph{span}, and how they are related to each other geometrically
\end{itemize}



}











\frame{\frametitle{Span} 

    \begin{center}\begin{tikzpicture} \node [mybox](box){\begin{minipage}{0.80\textwidth}\vspace{4pt}

    Given vectors $ \vec v_1, \vec v_2 ,\dotsc, \vec v_p \in \mathbb R ^{n} $, and scalars $ c_1 , c_2, \dotsc, c_p$. The set of all linear combinations of $ \vec v_1, \vec v_2 ,\dotsc, \vec v_p $ is called the \Emph{span} of $ \vec v_1, \vec v_2 ,\dotsc, \vec v_p $.  

    \end{minipage}};
    \node[fancytitle, right=10pt] at (box.north west) {Definition};
    \end{tikzpicture}\end{center}    
}





\frame{\frametitle{Span Example}

Is $ \vec y$ in the span of vectors $\vec v_1$ and $\vec v_2$?  \\[6pt]

$ \vec v_1 = \spalignmat{ 1 ; -2 ; -3 }$, 
$ \vec v_2  = \spalignmat{ 2 ; 5 ; 6 }$,  and 
$ \vec y =  \spalignmat{ 7 ; 4 ; 15 }$.  \\ 
}




\frame{\frametitle{The Span of Two Vectors in $ \mathbb R^{3}$}

    In the previous example, did we find that $\vec y$ is in the span of $\vec v_1$ and $\vec v_2$? \vspace{12pt} 

    \Emph{In general:} Any two non-parallel vectors in $\R^3$ span a plane that passes through the origin. Any vector in that plane is also in the span of the two vectors.
    \begin{center}
%        \begin{tikzpicture}
            %\begin{axis}
             %   \addplot3[mesh,draw=gray,samples=100] {x+y};
                %coordinates { (0,0,0) (0,0.5,0) (0,1,0) (0.5,0,0.5) (0.5,0.5,0.5) (0.5,1,0.5)(1,0,1) (1,0.5,1) (1,1,1)};
            %\end{axis}
            %\draw[thick, -stealth,] (2.3,2) -- (4.5,3.7) node[anchor = west]{$\vec x$};
            %\draw[thick, -stealth,] (1.1,1.9) -- (5.8,3.8) node[anchor = west]{$\vec y$};
        %\end{tikzpicture}


      \tdplotsetmaincoords{74}{100}
        \begin{tikzpicture}[scale=2.4,
            axis/.style={->,black,-stealth}, 
            vectorC/.style={-stealth,black,very thick},
            vectorV/.style={-stealth,Teal,very thick},
            dsline/.style={black,dashed},
            perpline/.style={black, thin},
            tdplot_main_coords
            ]
        
            \coordinate (O) at (0,0,0);
            \coordinate (X) at (2,0,0);
            \coordinate (Y) at (0,1,0);        
            \coordinate (Z) at (0,0,1);        
            \coordinate (X1) at (1,1,0);        
            \coordinate (X2) at (1,0,1);        
            \coordinate (X3) at (2,1,1);        
        
            % plane
            \filldraw[draw=Teal!40,fill=Teal!08,]          
            (0,0,0) -- (1,1,0) -- (2,1,1) -- (1,0,1)
            -- cycle;
            
            % draw coordinate vectors
            \draw[vectorC] (O) -- (X) node[above]{};
            \draw[vectorC] (O) -- (Y) node[below]{};
            \draw[vectorC] (O) -- (Z) node[below]{};
            
            % draw coordinate vectors
            \draw[vectorV] (O) -- (X1) node[above]{};
            \draw[vectorV] (O) -- (X2) node[below]{};
        
            % other labels
            \draw (O) node[below]{$\vec 0$};
            
        \end{tikzpicture}
        
    \end{center}
        
}







\frame{\frametitle{Summary}

We explored the following concepts in this video. \vspace{4pt}
\begin{itemize}\setlength{\itemsep}{8pt}
    \item characterizing a set of vectors in terms of \Emph{linear combinations}, their \Emph{span}, and how they are related to each other geometrically
\end{itemize}
\pause
}
