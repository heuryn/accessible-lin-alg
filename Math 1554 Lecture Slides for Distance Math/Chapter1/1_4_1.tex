\title{The Matrix-Vector Product} 
\subtitle{\SubTitleName}
\institute[]{\Course}
\author{\Instructor}
\maketitle   






\frame{\frametitle{Topics and Learning Objectives}

    \Emph{Topics} \\
    \TopicStatement
    
    \begin{itemize}
        \item matrix notation for systems of equations
        \item the matrix product $ A \vec x$
    \end{itemize}

    \vspace{0.5cm}
    
    \LO\\
    
    \LearningObjectiveStatement
    
    \begin{itemize}
        \item compute matrix-vector products
        \item express linear systems as vector equations and matrix equations
        % \item characterize linear systems and sets of vectors using the concepts of span, linear combinations, and pivots
    \end{itemize}


}



\frame{\frametitle{Multiple Representations} 


    \textit{``Mathematics is the art of giving the same name to different things." } \\ - H. Poincar\'{e} 
    
    
    \vspace{24pt}

    In this section we introduce another way of expressing a linear system that we will use throughout this course.

} 





\frame{\frametitle{Notation for Dimensions of Vectors and Matrices}

    \vspace{-2pt}
    \begin{center}
        \begin{tabular}{ m{2cm} m{11cm} }
        \Emph{symbol} & \Emph{meaning}  \\
        \hline 
        \vspace{2pt} $\in$ & 
        \vspace{2pt} belongs to
        \\
        \vspace{2pt} $\mathbb R^n$  & 
        \vspace{2pt} the set of vectors with $n$ real-valued elements
        \\        
        \vspace{2pt} $\mathbb R^{m\times n}$  & 
        \vspace{2pt} the set of real-valued matrices with $m$ rows and $n$ columns
        \\                
        \end{tabular}
    \end{center}
    
    \vspace{6pt}
    \Emph{Example} \\
    The notation $\vec x \in \mathbb R^5$ means that $\vec x$ is a vector with five real-valued elements. 
}


\frame{\frametitle{Matrix-Vector Product as a Linear Combination}

    \begin{center}\begin{tikzpicture} \node [mybox](box){\begin{minipage}{0.85\textwidth}\vspace{4pt}

    If $ A \in \mathbb R^{m \times n}$ has columns $ \vec a_1 ,\dotsc, \vec a_n$ and $\vec x \in \mathbb R ^{n}$, then the \Emph{matrix vector product} $A \vec x$ is a linear combination of the columns of $A$.
    \begin{equation*}
    A \vec x 
    = 
    \begin{pmatrix*}
    \vert & \vert & \cdots & \vert  
    \\
    \vec a_1 & \vec a_2 & \cdots & \vec a_n
    \\
    \vert & \vert & \cdots & \vert  
    \end{pmatrix*} 
    \begin{pmatrix*}
    x_1 \\ x_2 \\ \vdots \\ x_n
    \end{pmatrix*} 
    = x_1 \vec a_1 + x_2 \vec a_2 + \cdots + x_n \vec a_n
    \end{equation*}
    Note that $ A \vec x$ is in the span of the columns of $ A$. 
    \end{minipage}};
    \node[fancytitle, right=10pt] at (box.north west) {Definition};
    \end{tikzpicture}\end{center}    
}

\frame{\frametitle{Linear Combination Examples}

    Suppose $A = \spalignmat{1 0 ;0 -3}$ and $\vec x = \spalignmat{2;3}$
    \begin{enumerate}
        \item The following product can be written as a linear combination of vectors:
        \vspace{12pt}
        $$A \vec x = \phantom { ******************************************************}$$
        \vspace{12pt}
        
        \item Is $\vec b = \spalignmat{2;9}$ in the span of the columns of $A$? 
    \end{enumerate}


}

\frame{\frametitle{Summary}

We explored the following concepts in this video. \vspace{4pt}
\begin{itemize}\setlength{\itemsep}{8pt}
        \item computing matrix-vector products
        \item expressing linear systems as vector equations and matrix equations
        % \item characterize linear systems and sets of vectors using the concepts of span, linear combinations, and pivots
\end{itemize}

}
