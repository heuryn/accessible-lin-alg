\title{Linear Independence Theorems} 
\subtitle{\SubTitleName}
\institute[]{\Course}
\author{\Instructor}
\maketitle   



\frame{\frametitle{Topics and Learning Objectives}

    \Emph{Topics} \\
    \TopicStatement
    \begin{itemize}
        \item linear independence
        \item geometric interpretation of linearly independent vectors
    \end{itemize}
    
    \vspace{0.5cm}

    \LO\\
    
    \LearningObjectiveStatement

    \begin{itemize}
        \item characterize a set of vectors and linear systems using the concept of linear independence
        \item construct dependence relations between linearly dependent vectors
    \end{itemize}

}


\begin{frame}
\frametitle{A Motivating Question}

What is the smallest number of vectors needed in a parametric solution to a linear system?

\end{frame}



\begin{frame}
\frametitle{Recall: Linear Independence}

A set of vectors $\{\vec v_1,\ldots,\vec v_k\}$ in $\R^n$ are \Emph{linearly independent} if 
\begin{align*}
    c_1 \vec v_1+c_2 \vec v_2+\cdots+c_k \vec v_k= \vec 0
\end{align*}
has only the \Emph{trivial} solution. It is \Emph{linearly dependent} otherwise.

\vspace{0.5cm} 

In other words, $\{\vec v_1,\ldots,\vec v_k\}$ are linearly dependent if there are real numbers $c_1,c_2,\ldots ,c_k$ \Emph{not all zero} so that
\begin{align*}
 c_1 \vec v_1+c_2 \vec v_2+\cdots+c_k \vec v_k= \vec 0
\end{align*}

\end{frame}





\begin{frame}
\frametitle{Example: Two Dependent Vectors}

    Suppose $\vec v_1, \vec v_2 \in \R^n$. When is the set $\{\vec v_1, \vec v_2\}$ linearly dependent? Provide a geometric interpretation.
    
    \pause 
    
    \vspace{12pt}
    
    \Emph{Solution} \\
    From our definition of linear dependence, if $\vec v_1, \vec v_2$ are dependent, \onslide<3->{then there exists a $c_1$ and a $c_2$, not \textbf{both} zero, so that $$c_1\vec v_1 + c_2\vec v_2 = \vec 0$$} 
    
\end{frame}





\begin{frame}
\frametitle{Example: Two Dependent Vectors}
    We consider two cases:
    \begin{itemize}
        \item<2->[1)] If $\vec v_1$ and/or $\vec v_2$ is the zero vector, then the vectors are dependent. If for example $\vec v_1 = \vec 0$, then $c_1\vec v_1 + c_2\vec v_2 = \vec 0$ is satisfied for $c_2 = 0$ and any $c_1$. 
        \item<3-> [2)] If $\vec v_1\ne \vec 0$ and $\vec v_2 \ne \vec 0$, then $\vec v_2 = -\frac{c_1}{c_2}\vec v_1$, so $\vec v_1$ and $\vec v_2$ are multiples of each other. The vectors are parallel (one vector is in the span of the other). 
    \end{itemize}

\end{frame}

\begin{frame}
\frametitle{Example: Two Dependent Vectors (continued)}

    Thus, two vectors in $\mathbb R^n$ are dependent when either or both of the following occur. 
    \begin{itemize}
        \item <2-> One or both vectors are the zero vector. 
        \item <3-> One vector is a multiple of the other. 
    \end{itemize}

\end{frame}








\begin{frame}
\frametitle{Linear Independence Theorems}

    \begin{enumerate}
        \item[1)] \Emph{More Vectors Than Elements:} Suppose $\vec v_1,\ldots ,\vec v_k$ are vectors in $\R^n$. If $k>n$, then $\{\vec v_1,\ldots ,\vec v_k\}$ is linearly dependent.
        
        \vspace{12pt}
        \pause 
        
        \Emph{Wny?} Every column of the matrix \[A = (\vec v_1,\ldots ,\vec v_k ) \] would have to be pivotal for the vectors to be independent. \pause But $A$ has \textbf{more columns than rows}, so every column cannot be pivotal. The vectors must be linearly dependent. 
    \end{enumerate}
        
\end{frame}








\begin{frame}
\frametitle{Linear Independence Theorems}

    \begin{enumerate}
        \item[2)] \Emph{Set Contains Zero Vector:} If any one or more of $\vec v_1,\ldots , \vec v_k$ is $\vec 0$, then $\{ \vec v_1, \ldots, \vec v_k\}$ is linearly dependent.
        
        \vspace{12pt}
        \pause 
        
        \Emph{Wny?} Every column of the matrix \[A = (\vec v_1,\ldots ,\vec v_k ) \] would have to be pivotal for the vectors to be independent. \pause But $A$ \textbf{has a zero column}, so every column cannot be pivotal. The vectors must be linearly dependent.         
    \end{enumerate}

\end{frame}


% \begin{frame}
% \frametitle{Linear Independence Theorems}

%     \begin{enumerate}
%         \item[3)] \Emph{Row Operations:} The linear dependence relations between the columns o $\vec v_1,\ldots , \vec v_k$ is $\vec 0$, then $\{ \vec v_1, \ldots, \vec v_k\}$ is linearly dependent.
        
%         \vspace{12pt}
%         \pause 
        
%         \Emph{Wny?} Every column of the matrix \[A = (\vec v_1,\ldots ,\vec v_k ) \] would have to be pivotal for the vectors to be independent. \pause But $A$ \textbf{has a zero column}, so every column cannot be pivotal. The vectors must be linearly dependent.         
%     \end{enumerate}

% \end{frame}




\begin{frame}
\frametitle{Application of our Linear Independence Theorems}

    By inspection, which matrices have linearly independent columns? 
    \begin{enumerate}
        \item $A = \spalignmat{1 0;2 0}$ \onslide<2->{\quad \textit{zero column $\Rightarrow$ dependent}}
        \item $B = \spalignmat{1 0 1;0 1 1}$ \onslide<3->{\quad \textit{more columns than rows $\Rightarrow$ dependent}}
        \item $C = \spalignmat{1 0 1;2 1 3;3 1 4}$ \onslide<4->{\quad \textit{last column is the sum of the first two $\Rightarrow$ dependent}}
        \item $D = \spalignmat{1 0 1;0 1 1;0 0 1}$ \onslide<5->{\quad \textit{every column is pivotal $\Rightarrow$ linearly independent}}
    \end{enumerate}

\end{frame}



\frame{\frametitle{Summary}

\SummaryLine \vspace{4pt}
\begin{itemize}\setlength{\itemsep}{8pt}
        \item characterizing a set of vectors and linear systems using the concept of linear independence
        \item constructing dependence relations between linearly dependent vectors
\end{itemize}

}
