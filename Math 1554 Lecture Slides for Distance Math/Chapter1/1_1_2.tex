\title{Consistent Systems} 
\subtitle{\SubTitleName}
\institute[]{\Course}
\author{\Instructor}
\maketitle   





\frame{\frametitle{Topics and Learning Objectives}
\Emph{Topics} \\
\TopicStatement

\begin{itemize}
    % \item Systems of Linear Equations 
    % \item Matrix Notation 
    % \item Elementary Row Operations 
    \item augmented matrices
    \item fundamental questions of existence and uniqueness of solutions 
    \item row equivalence
\end{itemize}

\vspace{0.5cm}

\LO\\

\LearningObjectiveStatement

\begin{itemize}
    \item express a set of linear equations as an augmented matrix
    \item characterize a linear system in terms of the number of solutions, and whether the system is consistent or inconsistent
    % \item Apply elementary row operations to solve linear systems of equations.
\end{itemize}

}



\frame{\frametitle{Augmented Matrices} 

It is redundant to write $x_1,x_2, \ldots$ again and again. So we rewrite systems using matrices. \pause For example,
\begin{equation*}
\begin{array}{cccc}
    x_1 & - 2x_2 &+ x_3 &=0  \\
 & 2x_2 &-8x_3 &= 7 
%  5x_1 &  & -5x_3 & =10
\end{array}
\end{equation*}
\pause
can be written as the \Emph{augmented matrix},

\begin{equation*}
\begin{amatrix}{3}
 1 & -2 & 1 & 0 \\
 0 & 2 & -8 & 7 
%  5 & 0 & -5 & 10
\end{amatrix}
\end{equation*}
\pause

The vertical line reminds us that the first three columns are the coefficients to our variables $x_1$, $x_2$, and $x_3$. \pause Row operations can be applied to rows of augmented matrices as though they were coefficients in a system. 


}




\frame{\frametitle{Consistent Systems and Row Equivalence} 

    \vspace{-6pt}

    \begin{center}\begin{tikzpicture} \node [mybox](box){\begin{minipage}{0.95\textwidth}\vspace{4pt}
    
        A linear system is \Emph{consistent} if it has at least one solution. 
        
    \end{minipage}};\node[fancytitle, right=10pt] at (box.north west) {\textbf{Definition: Consistent}};\end{tikzpicture}\end{center}
    
    \vspace{-6pt}
    \pause
    
    \begin{center}\begin{tikzpicture} \node [mybox](box){\begin{minipage}{0.95\textwidth}\vspace{4pt}
    
        Two matrices are \Emph{row equivalent} if a sequence of row operations transforms one matrix into the other.  
        
    \end{minipage}};\node[fancytitle, right=10pt] at (box.north west) {\textbf{Definition: Row Equivalence}};\end{tikzpicture}\end{center}
    
    \vspace{6pt}
    \pause 
    Note: if the augmented matrices of two linear systems are row equivalent, then the systems have the same solution set.

}




\frame{\frametitle{Example for Consistent Systems and Row Equivalence} 


    Suppose $$A=\spalignmat{1 1;0 1}, \quad B=\spalignmat{1 0;0 1}, \quad C=\spalignmat{1 1;0 0}$$
    \pause
    \begin{enumerate}
        \item Are $A$ and $B$ row equivalent? Are $A$ and $C$ row equivalent? 
    \end{enumerate}
} 

\frame{\frametitle{Example for Consistent Systems and Row Equivalence} 


    Suppose $$A=\spalignmat{1 1;0 1}, \quad B=\spalignmat{1 0;0 1}, \quad C=\spalignmat{1 1;0 0}$$
    \pause
    \begin{enumerate}
        \item[2.] Do the augmented matrices $\spalignaugmat{1 1 1;0 1 1}$ and $\spalignaugmat{1 1 1;0 0 1}$ correspond to consistent systems? 
    \end{enumerate}
} 



\frame{\frametitle{Summary: Fundamental Questions}

In this video we explored the following concepts. \vspace{4pt}
\begin{itemize}\setlength{\itemsep}{8pt}
    \item Augmented matrices, row equivalence, and consistent systems. 
    \item Fundamental questions that we revisit many times throughout our course:\vspace{4pt}
    \begin{enumerate}\setlength{\itemsep}{8pt}
    \item  Does a given linear system have a solution? In other words, is it consistent? 
    \item  If it is consistent, is the solution unique? 
    \end{enumerate}
\end{itemize}

}

\frame{}
    