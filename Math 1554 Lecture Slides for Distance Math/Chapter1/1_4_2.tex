\title{Existence of Solutions} 
\subtitle{\SubTitleName}
\institute[]{\Course}
\author{\Instructor}
\maketitle   






\frame{\frametitle{Topics and Learning Objectives}

    \Emph{Topics} \\
    \TopicStatement
    
    \begin{itemize}
        \item solution sets
    \end{itemize}

    \vspace{0.5cm}
    
    \LO\\
    
    \LearningObjectiveStatement
    
    \begin{itemize}
        % \item compute matrix-vector products
        \item express linear systems as vector equations and matrix equations
        \item characterize solution sets of linear systems using the concepts of span, linear combinations
    \end{itemize}


}



\frame{\frametitle{Equivalent Solution Sets}


Note that if  $ A$ is a $ m \times n$ matrix with columns $ \vec a_1 ,\dotsc, \vec a_n$, and $\vec x \in \mathbb R ^{n}$ and $ \vec b \in \mathbb R ^{m}$, then the solutions to  
\begin{equation*}
A \vec x = \vec b 
\end{equation*}
has the same set of solutions as the vector equation 
\begin{equation*}
x_1 \vec a_1 + \cdots + x_n \vec a_n = \vec b 
\end{equation*}
which as the same set of solutions as the set of linear equations with the augmented matrix 
\begin{equation*}
\begin{bmatrix*}
\vec a_1 & \vec a_2  & \cdots & \vec a_n & \vec b
\end{bmatrix*}
\end{equation*}


}


\frame{\frametitle{Linear Combinations and the Existence of Solutions}

    \begin{center}\begin{tikzpicture} \node [mybox](box){\begin{minipage}{0.80\textwidth}\vspace{2pt}

        The equation $ A \vec x = \vec b $ has a solution if and only if $ \vec b$ is a linear combination of the columns of $ A$. 
        
    \end{minipage}};
    \node[fancytitle, right=10pt] at (box.north west) {Theorem};
    \end{tikzpicture}\end{center}    

    This follows directly from our definition of $A\vec x$ being a linear combination of the columns of $A$. 
}


\frame{\frametitle{Using Linear Combinations to Characterize a System}

\Emph{Example}\\
For what vectors $ \vec b = \spalignmat{b_1;b_2;b_3}$ 
does the equation have a solution? 
\begin{equation*}
    \spalignmat{1 3 4;2 8 4;0 1 -2}\vec x = \vec b 
\end{equation*}
}





\frame{\frametitle{Multiple Representations of Linear Systems}

    We now have four \Emph{equivalent} ways of representing a linear system.
    
    \begin{enumerate}\setlength{\itemsep}{12pt}
        \item A list of equations: $ 2x_1 + 3x_2 = 7, \quad  x_1 - x_2 = 5 $
        \item An augmented matrix:
        $
        \begin{amatrix}{2}
         2 & 3 & 7 \\
         1 & -1 & 5 
         \end{amatrix}
        $
        
        \item A vector equation:
          $ x_1\spalignmat{2 ; 1} + x_2\spalignmat{3; -1} = \spalignmat{7; 5} $
        
        \item A matrix equation: $ \spalignmat{ 2 3; 1 -1}\spalignmat{x_1 ; x_2} = \spalignmat{7; 5} $
    \end{enumerate}
    
    Each representation gives us a different way to think about linear systems.
    
}

\frame{\frametitle{Summary}

\SummaryLine \vspace{4pt}
\begin{itemize}\setlength{\itemsep}{8pt}
        \item computing matrix-vector products
        \item expressing linear systems as vector equations and matrix equations
        \item characterize linear systems and sets of vectors using the concepts of span, linear combinations, and pivots
\end{itemize}

}
