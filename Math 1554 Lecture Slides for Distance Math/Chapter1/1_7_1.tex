\title{A Definition of Linear Independence} 
\subtitle{\SubTitleName}
\institute[]{\Course}
\author{\Instructor}
\maketitle   




\frame{\frametitle{Topics and Learning Objectives}

    \Emph{Topics} \\
    \TopicStatement
    \begin{itemize}
        \item linear independence
        \item geometric interpretation of linearly independent vectors
    \end{itemize}
    
    \vspace{0.5cm}

    \LO\\
    
    \LearningObjectiveStatement

    \begin{itemize}
        \item characterize a set of vectors and linear systems using the concept of linear independence
        % \item Construct dependence relations between linearly dependent vectors.
    \end{itemize}

}


\begin{frame}
\frametitle{A Motivating Question}

What is the smallest number of vectors needed in a parametric solution to a linear system?

\end{frame}



\begin{frame}
\frametitle{Linear Independence}

A set of vectors $\{\vec v_1,\ldots,\vec v_k\}$ in $\R^n$ are \Emph{linearly independent} if 
\begin{align*}
    c_1 \vec v_1+c_2 \vec v_2+\cdots+c_k \vec v_k= \vec 0
\end{align*}
has only the \Emph{trivial} solution. It is \Emph{linearly dependent} otherwise.

\vspace{0.5cm} 

In other words, $\{\vec v_1,\ldots,\vec v_k\}$ are linearly dependent if there are real numbers $c_1,c_2,\ldots ,c_k$ \Emph{not all zero} so that
\begin{align*}
 c_1 \vec v_1+c_2 \vec v_2+\cdots+c_k \vec v_k= \vec 0
\end{align*}

\end{frame}









\begin{frame}\frametitle{How to Establish Linear Independence}
Consider the vectors: $$\vec v_1, \vec v_2, \ldots  \vec v_k$$
To determine whether the vectors are linearly independent, we can set the linear combination to the zero vector: 
\begin{equation*}
    c_1 \vec v_1+c_2 \vec v_2+\cdots+c_k \vec v_k = 
    \begin{pmatrix} 
      \vec v_1 & \vec v_2 & \cdots & \vec v_k
    \end{pmatrix}
    \begin{pmatrix}
    c_1 \\ c_2 \\ \vdots \\ c_n
    \end{pmatrix} 
    = V \vec c  \stackrel {\color{DarkRed}??}= \vec 0
\end{equation*}

\bigskip 

Linear independence: there is \Emph{NO} non-zero solution $\vec c$  

\bigskip 

Linear dependence: there is a non-zero solution $\vec c$.  

\end{frame}




\begin{frame}
\frametitle{Example: Determine Whether Set is Independent}

For what values of $h$, if any, is the set of vectors linearly independent?
\begin{align*}
    \begin{pmatrix} 1\\1\\ h \end{pmatrix}, 
    \begin{pmatrix} 1 \\ h \\1\end{pmatrix}, 
    \begin{pmatrix} h\\1\\1\end{pmatrix}
\end{align*}

\end{frame}

\frame{}

\frame{\frametitle{Summary}

\SummaryLine \vspace{4pt}
\begin{itemize}\setlength{\itemsep}{8pt}
        \item characterizing a set of vectors using the concept of linear independence
\end{itemize}

}





% \begin{frame}
% \frametitle{Example: Two Vectors and Linear Dependence}

%     Suppose $\vec v_1, \vec v_2 \in \R^n$. When is the set $\{\vec v_1, \vec v_2\}$ linearly dependent? Provide a geometric interpretation.

% \end{frame}

% \begin{frame}
% \frametitle{Two Linear Independence Theorems}

%     \begin{enumerate}
%         \item \Emph{More Vectors Than Elements:} Suppose $\vec v_1,\ldots ,\vec v_k$ are vectors in $\R^n$. If $k>n$, then $\{\vec v_1,\ldots ,\vec v_k\}$ is linearly dependent.
%         \vfill
%         \item \Emph{Set Contains Zero Vector:} If any one or more of $\vec v_1,\ldots , \vec v_k$ is $\vec 0$, then $\{ \vec v_1, \ldots, \vec v_k\}$ is linearly dependent.
%     \end{enumerate}

    


% \end{frame}
