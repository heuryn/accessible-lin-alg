\title{Homogeneous Systems} 
\subtitle{\SubTitleName}
\institute[]{\Course}
\author{\Instructor}
\maketitle   




\frame{\frametitle{Topics and Learning Objectives}

    \Emph{Topics} \\
    \TopicStatement
    \begin{itemize}
        \item homogeneous systems
        \item parametric \Emph{vector} forms of solutions to linear systems
    \end{itemize}

    \vspace{0.5cm}

    \LO\\
    
    \LearningObjectiveStatement

    \begin{itemize}
        % \item Express the solution set of a linear system in parametric vector form.
        % \item Provide a geometric interpretation to the solution set of a linear system.
        \item characterize homogeneous linear systems using the concepts of free variables, span, pivots, linear combinations, and echelon forms
    \end{itemize}
}




\begin{frame}
\frametitle{Homogeneous Systems}

    \begin{center}\begin{tikzpicture} \node [mybox](box){\begin{minipage}{0.90\textwidth}\vspace{4pt}

        Linear systems of the form $A\vec x = \vec 0$ are \Emph{homogeneous}. \\[12pt]
        Linear systems of the form $A\vec x = \vec b, \ \vec b \ne \vec 0$, are \Emph{inhomogeneous}.
        
    \end{minipage}};
    \node[fancytitle, right=10pt] at (box.north west) {Definition};
    \end{tikzpicture}\end{center}    
    


    \vspace{12pt} 
    Because homogeneous systems always have the \Emph{trivial solution}, $\vec x= \vec 0$, the interesting question is whether they have \Emph{non-trivial} solutions.  \\[12pt]

\end{frame}





\begin{frame}
\frametitle{Homogeneous Systems}

    \begin{center}\begin{tikzpicture} \node [mybox](box){\begin{minipage}{0.60\textwidth} 
        \vspace{-6pt}
        \begin{align*} 
        A\vec x &= \vec 0 \text{ has a nontrivial solution} \\
        & \iff  \text{there is a free variable} \\
        & \iff  A \text{ has a column with no pivot.}
        \end{align*}
    \end{minipage}}; \node[fancytitle, right=10pt] at (box.north west) {Observation}; \end{tikzpicture}\end{center}

\end{frame}






\begin{frame}\frametitle{Example: a Homogeneous System}

    Identify the free variables, and the solution set, of the system.
    \begin{align*}
        x_1 + 3x_2 + x_3 &=0 \\
        2x_1 -x_2 - 5x_3 &= 0 \\
        x_1 - 2x_3 &=0
    \end{align*}

\end{frame}




\frame{\frametitle{Summary}

\SummaryLine \vspace{4pt}
\begin{itemize}\setlength{\itemsep}{8pt}
    \item characterizing homogeneous and inhomogeneous systems 
    \item relationships between free variables, pivots, and solutions
    \item identifying free variables of homogeneous systems
\end{itemize}

}



% \begin{frame}\frametitle{Parametric Forms, Homogeneous Case}

%     In the example on the previous slide we expressed the solution to a system using a vector equation. This is a \Emph{parametric form} of the solution. \\[12pt]

%     In general, suppose the free variables for $A \vec x= \vec 0$ are $x_k,\ldots,x_n$. Then all solutions to $A \vec x= \vec 0$ can be written as
%     \begin{align*}
%         \vec x = x_k \vec v_k + x_{k+1} \vec v_{k+1}+\cdots + x_n \vec v_n
%     \end{align*}
%     for some $\vec v_k,\ldots,\vec v_n$. This is the \Emph{parametric form} of the solution.

% \end{frame}





% \begin{frame}
% \frametitle{Example 2 (non-homogeneous system)}

%     Write the parametric vector form of the solution, and give a geometric interpretation of the solution.
%     \begin{align*}
%         x_1 + 3x_2 + x_3 &=9 \\
%         2x_1 -x_2 - 5x_3 &= 11 \\
%         x_1 - 2x_3 &=6
%     \end{align*}

%     (Note that the left-hand side is the same as Example 1).  


% %We need a particular solution to the non-homogeneous equation. You can get this by row-reducing the augmented matrix, or by guessing.  Here is a particular solution: 
% %\begin{equation*}
% %\begin{bmatrix*}[r]
% %1 & 1 & -3 \\ 1 & -1 & 1 \\ -1 & 5 & -7 
% %\end{bmatrix*}
% %\begin{bmatrix*}[r]
% %-2 \\- 1 \\ 1  
% %\end{bmatrix*} = 
% %\begin{bmatrix*}[r]
% %0 \\ 1 \\ -10 
% %\end{bmatrix*}
% %\end{equation*}
% %(Continued on next slide) 

% \end{frame}


% %\begin{frame}
% %Then, you add to this (or any other) particular solution the parametric form of your homogeneous solution: 
% %\begin{equation*}
% %\begin{bmatrix*}[r]
% %1 & 1 & -3 \\ 1 & -1 & 1 \\ -1 & 5 & -7 
% %\end{bmatrix*}
% %\begin{bmatrix*}[r]
% %2 \\ 5 \\ 3
% %\end{bmatrix*} = \vec 0 
% %\end{equation*}
% %And the matrix only has one free variable, call it $ x_3$, so our parameterized solution to the non-homogeneous solution is 
% %\begin{equation*}
% %\begin{bmatrix*}[r]
% %-1 \\- 1 \\ 1  
% %\end{bmatrix*} + x_3 \begin{bmatrix*}[r]
% %2 \\ 5 \\ 3
% %\end{bmatrix*} 
% %\end{equation*}
% %\end{frame}
