\title{Geometric Interpretations of Linear Transforms} 
\subtitle{\SubTitleName}
\institute[]{\Course}
\author{\Instructor}
\maketitle   


\frame{\frametitle{Topics and Learning Objectives}
    \Emph{Topics} \\
    \TopicStatement
    \begin{itemize}
        \item geometric interpretations of a linear transform
    \end{itemize}
    
    \vspace{0.5cm}

    \LO\\
    
    \LearningObjectiveStatement

    \begin{itemize}
        \item construct and interpret linear transformations in $\mathbb R^n$ (for example, interpret a linear transform as a projection, or as a shear)
        % \item Characterize linear transforms using the concepts of domain, codomain, image, and range

    \end{itemize}
    
    \vspace{0.25cm} 

% \Emph{Motivating Question} \\


}






\begin{frame}
\frametitle{Linear Transformations}

A function $T:\R^n \rightarrow \R^m$ is \Emph{linear} if

\begin{itemize}
    \item $T(\vec u+ \vec v) = T(\vec u) + T(\vec v)$ for all $\vec u, \vec v$ in $\R^n$.
    \item $T(c \vec v) = cT(\vec v)$ for all $\vec v \in \R^n$, and $c$ in $\R$.
\end{itemize}


So if $T$ is linear, then $$ T(c_1\vec v_1+\cdots+c_k\vec v_k) =  c_1T(\vec v_1)+\cdots+c_kT(\vec v_k)  $$

This is called the \Emph{principle of superposition}. % The idea is that if we know $T(\vec e_1),\dots,T(\vec e_n)$, then we know every $T(\vec v)$. 

 
\vspace{0.5cm} 

\Emph{Fact}: Every matrix transformation $T_A$ is linear.

% Next time: Every linear function $\R^n \to \R^m$ is a matrix transformation. 
\end{frame}




\begin{frame}
\frametitle{Geometric Interpretations of Transforms in $\mathbb R^2$}

Suppose $T$ is the linear transformation $T(\vec x) = A\vec x$. Give a short geometric interpretation of what  $T(\vec x)$ does to vectors in $\R^2$. 

\begin{enumerate}
    \item[1)] $A=\begin{pmatrix} 0 & 1 \\ 1 & 0 \end{pmatrix}$ \vspace{1cm}

    \item[2)] $A=\begin{pmatrix} 1 & 0 \\ 0 & 0 \end{pmatrix}$ \vspace{1cm}

    \item[3)] $A=\begin{pmatrix} k & 0 \\ 0 & k \end{pmatrix}$ for $k \in \R$ \vspace{1cm}
    
\end{enumerate}

\end{frame}



\begin{frame}
\frametitle{Geometric Interpretations of Transforms in $\mathbb R^3$}

What does $T_A$ do to vectors in $\R^3$?

\begin{enumerate}
    \item[a)] $A=\begin{pmatrix} 1&0&0 \\ 0&1&0 \\ 0&0&0 \end{pmatrix}$ \vspace{2cm}

    \item[b)] $A=\begin{pmatrix} 1&0&0\\0&-1&0\\0&0&1 \end{pmatrix}$ \vspace{1cm}

\end{enumerate}


\end{frame}







\begin{frame}
\frametitle{Constructing the Matrix of the Transformation}
A linear transformation  $ T \;:\;  \mathbb R ^2 \mapsto \mathbb R ^{3} $  satisfies 
\begin{equation*}
T \left( \begin{pmatrix}
 1 \\ 0
\end{pmatrix}\right) = \begin{pmatrix}
5 \\ -7 \\ 2  
\end{pmatrix},
\qquad 
T \left( \begin{pmatrix*}
  0 \\ 1
 \end{pmatrix*} \right) = \begin{pmatrix*}[r]
-3 \\ 8 \\ 0  
\end{pmatrix*}
\end{equation*}
What is the matrix, $A$, so that $ T = Ax$? 

\end{frame}


\frame{\frametitle{Summary}

    \SummaryLine \vspace{4pt}
    \begin{itemize}\setlength{\itemsep}{8pt}
            \item constructing linear transformations in $\mathbb R^2$ and $\mathbb R^3$ and geometric interpretations for them
    \end{itemize}
    We will need to go into more detail on linear transformations and their relationships to linear systems. 
}


