\title{Echelon Form and RREF} 
\subtitle{\SubTitleName}
\institute[]{\Course}
\author{\Instructor}
\maketitle   


\frame{\frametitle{Topics and Learning Objectives}
\Emph{Topics} \\
\TopicStatement

\begin{itemize}
\item echelon form and row reduced echelon form
% \item Row reduction algorithm
% \item Pivots, and basic and free variables 
%\item Parametrized solution sets 
% \item Echelon forms, existence and uniqueness 
\end{itemize}

\vspace{0.5cm}

\LO\\

\LearningObjectiveStatement

\begin{itemize}
    \item identify whether a matrix is in echelon form or in row reduced echelon form (RREF)
    \item give examples of matrices in echelon form or in RREF
    % \item Characterize a linear system in terms of the number of leading entries, free variables, pivots, pivot columns, pivot positions.
    % \item Apply the row reduction algorithm to reduce a linear system to echelon form, or reduced echelon form.
    % \item Apply the row reduction algorithm to compute the coefficients of a polynomial.    
\end{itemize}

}



\frame{\frametitle{Motivation: Identifying a Solution to a Linear System}

    This matrix below in a form referred to as \Emph{row reduced echelon form}. 
    \begin{equation*}
    \begin{amatrix}{2}
     1 & 0 & 3 \\
     0 & 1 & 7 
    %  5 & 0 & -5 & 10
    \end{amatrix}
    \end{equation*}
    By inspection, what is the solution to the linear system? 
    
}



\frame{\frametitle{Definition: Echelon Form}

%%%%%%%%%%%%%%%%%%%%%%%%%%%%%%  DEFINITION DEFINITION DEFINITION
%\begin{definition}
A rectangular matrix is in \Emph{echelon form} if 
%%  ENUMERATE
\begin{enumerate}
\item<2-> All zero rows (if any are present) are at the bottom. 
\item<3-> The first non-zero entry (or \Emph{leading entry}) of a row is to the right of any leading entries in the row above it (if any).
\item<4-> All entries below a leading entry (if any) are zero.  
\end{enumerate}

\onslide<4->{
\Emph{Examples}\\
Matrix $A$ is in echelon form. $B$ is not in echelon form. 

$$A = \spalignmat{2 0 1 1;0 0 5 3;0 0 0 0}, \quad B = \spalignmat{0 0 3;0 0 2}$$
}
}

\frame{\frametitle{Definition: Echelon Form}

A matrix in echelon form is in \Emph{row reduced echelon form} (RREF) if  
%%  ENUMERATE
\begin{enumerate}
\item All leading entries, if any, are equal to 1. 
\item Leading entries are the only nonzero entry in their respective column. 
\end{enumerate}

\onslide<2->{
\Emph{Examples}\\
Matrix $A$ is in RREF. $B$ is not in RREF. 

$$A = \spalignmat{1 0 0 1;0 0 1 3;0 0 0 0}, \quad B = \spalignmat{1 0 6 1;0 0 1 3;0 0 0 0}$$
}

}

\frame{\frametitle{Example of a Matrix in Echelon Form} 

\begin{equation*}
\blacksquare = \textup{non-zero number}, \qquad \ast = \textup{any number}
\end{equation*}


\begin{equation*}
\begin{pmatrix*}
0 & \blacksquare  & \ast  & \ast & \ast & \ast   & \ast & \ast & \ast  & \ast 
\\
0 & 0 & 0   & \blacksquare& \ast & \ast   & \ast & \ast & \ast  & \ast 
\\
0 & 0 & 0 & 0 & 0 & 0 & 0 &  \blacksquare & \ast  & \ast 
\\
0  & 0  & 0  & 0 &  0  & 0    &  0  &  0 & \blacksquare  & \ast 
\\
0  & 0  & 0   & 0 &  0  & 0    &  0  &  0 &  0 &0 
\end{pmatrix*}
\end{equation*}


}




\begin{frame}
\frametitle{Example}

Which of the following are in RREF? 
\begin{align*}
    a) \quad & \begin{pmatrix} 1 & 0\\0 & 2 \end{pmatrix} 
    && d) \quad \begin{pmatrix} 0 & 6 & 3 & 0 \end{pmatrix} \\ \\
    b) \quad & \begin{pmatrix} 0 & 0 \\ 0 & 0 \end{pmatrix} 
    && e) \quad \begin{pmatrix} 1 & 17 & 0\\ 0 & 0 & 1\end{pmatrix} \\ \\
    c) \quad & \begin{pmatrix} 0 \\ 1 \\ 0 \\ 0 \end{pmatrix} 
\end{align*}

\end{frame}


\frame{\frametitle{Summary: Echelon and RREF}

In this video we explored the following concepts. \vspace{4pt}
\begin{itemize}\setlength{\itemsep}{8pt}
    \item echelon and row reduced echelon forms
    \item identifying whether a matrix is in echelon or in RREF 
\end{itemize}

}

    
    


