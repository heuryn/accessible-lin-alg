\title{The Row Reduction Algorithm} 
\subtitle{\SubTitleName}
\institute[]{{\Course}}
\author{\Instructor}
\maketitle   




\frame{\frametitle{Topics and Learning Objectives}
\Emph{Topics} \\
\TopicStatement


\begin{itemize}
\item row reduction algorithm
\item pivots and pivot columns %, and basic and free variables 
%\item Parametrized solution sets 
% \item Echelon forms, existence and uniqueness 
\end{itemize}

\vspace{0.5cm}

\LO\\

\LearningObjectiveStatement

\begin{itemize}
    \item characterize a linear system in terms of the number of leading entries, pivots, pivot columns, pivot positions
    \item apply the row reduction algorithm to reduce a linear system to echelon form, or to RREF
\end{itemize}

}




\frame{\frametitle{Definition: Pivot Position, Pivot Column}

    A \Emph{pivot position} in a matrix $ A$ is a location in $ A$ that corresponds to a leading 1 in the row reduced echelon form of $ A$.  \\[12pt]
    \pause 
    A \Emph{pivot column} is a column of $ A$ that contains a pivot position. \\[12pt]
    \pause 
    \Emph{Example}: Express the matrix in RREF and identify the pivot columns.
    
    \begin{equation*}
    \left(
    \begin{array}{rrrr} 
        0 & -3 & -6 & 9
        \\
        -1 & -2 & -1 & 3 
        \\
        -2 & -3 & 0 & 3 
    \end{array}\right)
    \end{equation*}
}






\frame{\frametitle{Row Reduction Algorithm} 

    The algorithm we used in the previous example produces a matrix in RREF. Its steps can be stated as follows.
    \pause
    {\color{Teal}
        \setlength{\extrarowheight}{0.2cm}
        \begin{center}
        \hspace{-.9cm}\begin{tabular}{ p{1.4cm} p{11cm} }
            Step 1: & Swap the first row with a lower one so the leftmost nonzero entry is in the first row \\ 
            Step 2: & Scale the 1st row so that its leading entry is equal to 1 \\
            Step 3: & Use row replacement so all entries above and below this leading entry (if any) are equal to zero \\
            % Step 2 & Swap the 2nd row with a lower one so that the leftmost nonzero entry below 1st row is in the 2nd row
    
        \end{tabular}
        \end{center}
        \setlength{\extrarowheight}{0.0cm}
        \pause 
        Then repeat these steps for row 2, then for row 3, and so on, for the remaining rows of the matrix. 
    }

}


\frame{\frametitle{Notes on the Row Reduction Algorithm} 

    \begin{itemize}
        \item <1-> There are many algorithms for reducing a matrix to echelon form, or to RREF.
        \item <2-> If we only need to count pivots, we do not need RREF. Echelon form is sufficient. 
    \end{itemize}

}


\frame{\frametitle{Summary: Fundamental Questions}

In this video we explored the following concepts. \vspace{4pt}
\begin{itemize}\setlength{\itemsep}{8pt}
    \item pivot, pivot columns, pivot positions
    \item the row reduction algorithm
\end{itemize}

}


