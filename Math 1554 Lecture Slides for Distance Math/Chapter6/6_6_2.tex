\title{Mean-Deviation Form}
\subtitle{\SubTitleName}
\institute[]{\Course}
\author{\Instructor}
\maketitle   


\begin{frame}\frametitle{Topics and Objectives}
    \Emph{Topics} \\
    %\TopicStatement
    \begin{itemize}
    
        \item mean-deviation form
        
    \end{itemize}
    
    \vspace{0.5cm}
    
    \Emph{Learning Objectives}\\
    
    \begin{itemize}
    
        \item apply mean-deviation form to a linear model of the form $y = c_0 + c_1 x$
      
    \end{itemize}
    
    \vspace{0.25cm} 
 
\end{frame}
 
 
 

\begin{frame}
\frametitle{Mean-Deviation Form}

    \begin{itemize}
        \item <1-> Linear models are commonly used in science and engineering model relationships between two or more quantities $\Rightarrow$ identifying ways to make that process more efficient is valuable.
        \item <2-> A common practice when using a model of the form $y = c_0 + c_1x$ is to compute the average, $\bar x$, of the $x-$values, and introduce a new variable $x_{\ast} = x - \bar x$. 
        \item <3-> The new data are in \Emph{mean-deviation form}.  
    \end{itemize}
    \onslide<4->{In the next few slides we will see how this approach can be useful. }
    
\end{frame}

\begin{frame}
\frametitle{Example: Linear Model of the Form $y = c_0 + c_1x$}

    Suppose we are given the data below. 
    \begin{center}
    \begin{tabular}{c|cccc} 
    $ x_i$ & 2 & 5 & 7 & 8 
    \\ \hline 
    $ y_i$ & 1 & 1 & 4 & 3 
    \end{tabular}
    \end{center}
    
    If we were to fit a linear model to this data of the form $ y= c_0 + c_1 x$, we could construct the system
        \pause

    \begin{equation*}
    \begin{pmatrix}
    1 & 2 \\ 1 & 5 \\ 1 & 7 \\ 1 & 8 
    \end{pmatrix} \begin{pmatrix}
    c _0 \\ c _1 
    \end{pmatrix} = \begin{pmatrix}
    1 \\ 1 \\ 4 \\ 3 
    \end{pmatrix}
    \end{equation*}
    This is a least-squares problem of the form $A \vec x = \vec y$. 
    
\end{frame}

\begin{frame}{Solution using Normal Equations}
    If we use the normal equations directly, we have 
    \begin{align*}
    A ^{T} A = \begin{pmatrix} 4 & 22 \\ 22 & 142 \end{pmatrix} , \qquad 
    A ^{T} \vec y = \begin{pmatrix} 9 \\ 59 \end{pmatrix}
    \end{align*}
    
    The least-squares solution is given by solving
    \begin{equation*}
    \begin{pmatrix}4 &22 \\  22 & 142 \end{pmatrix} \begin{pmatrix} c_0 \\ c_1 \end{pmatrix} = \begin{pmatrix} 9 \\ 59 \end{pmatrix}
    \end{equation*}
    \pause 
    After solving this linear system, we obtain $c_0 = -5/21$ and $c_1 = 19/42$. 
    \begin{equation*}
        y = c _0 + c _1 x = \frac{-5}{21} + \frac{19}{42} x 
    \end{equation*}
    \pause 
    If we were to use mean-deviation form, how would our approach be different? 
\end{frame}


\begin{frame}
\frametitle{Example: Linear Model of the Form $y = c_0 + c_1x$}

    \begin{center}
    \begin{tabular}{c|cccc} 
    $ x_i$ & 2 & 5 & 7 & 8 
    \\ \hline 
    $ y_i$ & 1 & 1 & 4 & 3 
    \end{tabular}
    \end{center}
    The average value of $x_i$ is $\bar x = 5.5$. Subtracting $\bar x$ from the $x-$values, our new linear model becomes 
    $$ y = c_0 + c_1 x_*$$
    and our linear system becomes \pause
    \begin{equation*}
    \begin{pmatrix}
    1 & -3.5 \\ 1 & -0.5 \\ 1 & 1.5 \\ 1 & 2.5
    \end{pmatrix} \begin{pmatrix}
    c _0 \\ c _1 
    \end{pmatrix} = A \vec x = \begin{pmatrix}
    1 \\ 1 \\ 4 \\ 3 
    \end{pmatrix}
    =\vec b
    \end{equation*}
    \pause 
    What property does the columns of $A$ now have?
    
\end{frame}


\begin{frame}{The Normal Equations with Mean-Deviation Form}
    If we use the normal equations again, we now have 
    \begin{align*}
    A ^{T} A = \begin{pmatrix} 4 & 0 \\ 0 & 21 \end{pmatrix} , \qquad 
    A ^{T} \vec y = \begin{pmatrix} 9 \\ 9.5 \end{pmatrix}
    \end{align*}
    \pause By using mean-deviation form, $A$ has orthogonal columns, so $A^TA$ is\pause diagonal. \pause Upon solving the normal equations, we obtain (by inspection) $c_0 = 9/4$ and $c_1 = 19/42$. 
    \begin{equation*}
        y = c _0 + c _1 x_* = \frac{9}{4} + \frac{19}{42} (x - \bar x) =  \frac{-5}{21} +  \frac{19}{42} x
    \end{equation*}
    \pause 
    
\end{frame}





 \frame{\frametitle{Summary}

    \SummaryLine \vspace{4pt}
    \begin{itemize}\setlength{\itemsep}{8pt}

        \item constructing a linear model of the form $y = c_0 + c_1 x$ by using a change of variable, $x_* = x - \bar x$ (mean-deviation form). 

    \end{itemize}
    
    \vspace{16pt}
    \pause 
    
    
}
