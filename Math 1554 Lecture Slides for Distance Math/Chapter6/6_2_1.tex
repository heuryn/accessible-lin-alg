\title{Orthogonal Bases}
\subtitle{\SubTitleName}
\institute[]{\Course}
\author{\Instructor}
\maketitle   
  

 
\begin{frame}\frametitle{Topics and Objectives}
\Emph{Topics} \\
%\TopicStatement
\begin{itemize}

    \item orthogonal sets of vectors

    \item orthogonal and orthonormal bases 
    

\end{itemize}

\vspace{0.5cm}

\Emph{Learning Objectives}\\

\LearningObjectiveStatement

\begin{itemize}
    % \item compute orthogonal projections and distances
    % \item express a vector as a linear combination of orthogonal vectors
    % \item characterize bases for subspaces of $\mathbb R^n$, and
    \item determine whether a basis is orthogonal or whether it is orthonormal
    \item construct and give examples of orthogonal and orthonormal bases
\end{itemize}


\end{frame}


\begin{frame}{Orthogonal Vector Sets}

    % ~~ ~~ Highlight Box ~~ ~~
    \begin{center}\begin{tikzpicture} \node [mybox](box){\begin{minipage}{0.85\textwidth}\vspace{2pt}

        A set of vectors $ \{\vec u_1 ,\dotsc, \vec u_p\}$ are an \Emph{orthogonal set} of vectors if for each $ j\neq k$, $ \vec u_j \perp \vec u_k$.  

    \end{minipage}};
    \node[fancytitle, right=10pt] at (box.north west) {Definition};
    \end{tikzpicture}\end{center}
    % ~~ ~~ Highlight Box ~~ ~~
        
    \vspace{4pt} 
    \pause 
    
    \Emph{Example:} Fill in the missing entries to make $ \{\vec u_1 ,\vec u_2, \vec u_3\}$ a set of non-zero orthogonal vectors.
    \begin{equation*}
        \vec u_1 = \begin{pmatrix} 4 \\ 0 \\ 1 \end{pmatrix}, \quad 
        \vec u_2 = \begin{pmatrix}-2 \\ 0 \\    \phantom{-1 - 1 }\end{pmatrix}, \quad 
        \vec u_3 = \begin{pmatrix} 0 \\  \phantom{-1 - 1 } \\ \phantom{-1 - 1 }\end{pmatrix}
    \end{equation*}
\end{frame}


\begin{frame}{Orthogonal Vector Sets}
    \Emph{Example:} Fill in the missing entries to make $ \{\vec u_1 ,\vec u_2, \vec u_3\}$ a set of non-zero orthogonal vectors.
    \begin{equation*}
        \vec u_1 = \begin{pmatrix} 4 \\ 0 \\ 1 \end{pmatrix}, \quad 
        \vec u_2 = \begin{pmatrix}-2 \\ 0 \\    \phantom{-1 - 1 }\end{pmatrix}, \quad 
        \vec u_3 = \begin{pmatrix} 0 \\  \phantom{-1 - 1 } \\ \phantom{-1 - 1 }\end{pmatrix}
    \end{equation*}
    
    \Emph{Solution}\\ \pause
    We need $\vec u_1\cdot \vec u_2 = 0$\pause, so we need $\vec u_2 = \begin{pmatrix} -2\\0 \\ 8 \end{pmatrix}$. 
    
    \pause 
    
    \vspace{12pt}
    
    We also need $\vec u_3\cdot \vec u_1 = \vec u_3 \cdot \vec u_2 = 0$, so we can set $\vec u_3 = \begin{pmatrix} 0 \\ 1 \\ 0 \end{pmatrix}$.

\end{frame}



\begin{frame}{Orthogonality and Linear Independence}
\begin{center}\begin{tikzpicture} \node [mybox](box){\begin{minipage}{0.85\textwidth} \vspace{4pt}
    Let  $S = \{\vec u_1 ,\dotsc, \vec u_p\}$ be an \Emph{orthogonal set} of vectors.  %Then, for scalars $ c_1 ,\dotsc, c_p$, 
    % \begin{equation*}
    % \bigl\lVert c_1 \vec u_1 + \cdots + c_p \vec u_p  \bigr\rVert ^2 
    % = c_1 ^2 \lVert \vec u_1 \rVert ^2 + \cdots + c_p ^2 \lVert \vec u_p\rVert ^2 . 
    % \end{equation*}    In particular, 
    If $ \vec u_i$ are non-zero, the then $S$ is a set of \Emph{linearly independent} vectors. 
    \end{minipage}};
    \node[fancytitle, right=10pt] at (box.north west) {Theorem};
    \end{tikzpicture}\end{center}
    \vspace{6pt}
    \pause
    \Emph{Proof}: Suppose $ \{\vec u_1 ,\dotsc, \vec u_p\}$ are a set of non-zero orthogonal vectors, \pause and $$\sum_{i=1}^ p c_i\vec u_i = c_1 \vec u_1 + c_2 \vec u_2  + \ldots + c_p \vec u_p = \vec 0$$ for scalars $c_1, c_2, \ldots , c_p$, then $\ldots$

\end{frame}


\begin{frame}{Orthogonality and Linear Independence}

    Then 
    \begin{align*}
        0 &= \vec u_1 \cdot \vec 0 \\
        \onslide<2->{&= \vec u_1 \cdot \sum_{i=1}^ p c_i\vec u_i} \\
        \onslide<3->{&= \sum_{i=1}^ p c_i \vec u_1 \cdot \vec u_i} \\
        \onslide<4->{&= c_1 \vec u_1 \cdot \vec u_1} 
    \end{align*}
    \onslide<5->{But $\vec u_1$ is not the zero vector, so $c_1=0$. Likewise, $c_2, c_3 , \ldots c_p$ must also be zero, which means that $S$ is a set of linearly independent vectors.  }
\end{frame}





\begin{frame}{Orthogonal Bases}

    \vspace{-12pt}
    
    \begin{center}\begin{tikzpicture} \node [mybox](box){\begin{minipage}{0.85\textwidth}\vspace{4pt}
        Let  $ \{\vec u_1 ,\dotsc, \vec u_p\}$ be an orthogonal  basis for a subspace $ W$ of $ \mathbb R ^{n}$. Then, for any vector $ \vec w\in W$, 
        \begin{equation*}
            \vec w= c_1  \vec u_1   + \cdots + c_p   \vec u_p  . 
        \end{equation*}
        Above, the scalars are $ \displaystyle c_ q = \frac { \vec w \, \cdot \, \vec u_q } { \vec u _{q} \cdot \, \vec u_q }$. 
    \end{minipage}};
    \node[fancytitle, right=10pt] at (box.north west) {Theorem (Expansion in Orthogonal Basis) };
    \end{tikzpicture}\end{center}
    \pause
    Obtaining the coefficients, $ \displaystyle c_ q$, with the above formula is generally more efficient than row reduction.  \pause However, we can only apply this theorem when we have an orthogonal basis for $W$. 
\end{frame}










\begin{frame}{Example: Orthogonal Basis}  
    \vspace{-6pt}
    Suppose $W$ is the subspace of $\mathbb R ^{3}$ that is orthogonal to $\vec x$. 
    $$\vec x = \spalignmat{1;1;1}, \quad \vec u = \spalignmat{1;-2;1}, \quad \vec v = \spalignmat{-1;0;1}, \quad \vec s = \spalignmat{3;-4;1}$$
    \begin{enumerate}
        \item Confirm that an orthogonal basis for $W$ is given by $\vec u$ and $\vec v$.
        \item Assume $\vec s \in W$. Compute the expansion of $\vec s$ in the basis for $W$.
    \end{enumerate}
\end{frame}


\begin{frame}{Solution to Part 1}  
    \vspace{-6pt}
    Suppose $W$ is the subspace of $\mathbb R ^{3}$ that is orthogonal to $\vec x$. 
    $$\vec x = \spalignmat{1;1;1}, \quad \vec u = \spalignmat{1;-2;1}, \quad \vec v = \spalignmat{-1;0;1}, \quad \vec s = \spalignmat{3;-4;1}$$ \vspace{-12pt}
    \begin{enumerate}
        \item Confirm that an orthogonal basis for $W$ is given by $\vec u$ and $\vec v$.
        % \item Assume $\vec s \in W$. Compute the expansion of $\vec s$ in the basis for $W$.
    \end{enumerate}
    
    \pause 
    \Emph{Solution to Part 1}\\ \pause 
    For $\vec u$ and $\vec v$ to form an orthogonal basis for $W$, (a) $\vec u$ and $\vec v$ must be in $W$, and (b) they must be orthogonal.  \pause 
    \begin{enumerate}[a)]
        \item $\vec x \cdot \vec u = 1 -2 +1 = 0$ and $\vec x \cdot \vec v = -1 + 0 + 1 = 0 $ so $\vec u$ and $\vec v$ are in $W$. 
        \item $\vec u \cdot \vec v = -1 + 0 -1 = 0$, so $\vec u$ and $\vec v$ are orthogonal.
    \end{enumerate}
    \pause 
    Thus, $\vec u$ and $\vec v$ form an orthogonal basis for $W$. 
\end{frame}



\begin{frame}{Solution to Part 2}  
    \vspace{-6pt}
    Suppose $W$ is the subspace of $\mathbb R ^{3}$ that is orthogonal to $\vec x$. \pause 
    $$\vec x = \spalignmat{1;1;1}, \quad \vec u = \spalignmat{1;-2;1}, \quad \vec v = \spalignmat{-1;0;1}, \quad \vec s = \spalignmat{3;-4;1}$$ \vspace{-12pt} \pause 
    \begin{enumerate}
        % \item Confirm that an orthogonal basis for $W$ is given by $\vec u$ and $\vec v$.
        \item[2.] Assume $\vec s \in W$. Compute the expansion of $\vec s$ in the basis for $W$.
    \end{enumerate}
    
    \vspace{12pt}
    \Emph{Solution to Part 2}\\ \pause 
    Given that $\vec s$ is in $W$, and we have an orthogonal basis for $W$, we can use our theorem to write: \pause 
    $$\vec s = \frac{\vec s \cdot \vec u}{\vec u \cdot \vec u}\vec u + \frac{\vec s \cdot \vec v}{\vec v \cdot \vec v}\vec v=\frac{12}{6}\vec u + \frac{-2}{2}\vec v = 2\vec u - \vec v$$ \pause 
    Therefore, $\vec s = 2\vec u - \vec v$. 
\end{frame}


\begin{frame}{Definition}
    \begin{center}\begin{tikzpicture} \node [mybox](box){\begin{minipage}{0.85\textwidth}\vspace{4pt}
        An \Emph{orthonormal basis} for a subspace $ W$ is an orthogonal basis $ \{\vec u_1 ,\dotsc, \vec u_p\}$ 
        in which every vector $ \vec u_q$ has unit length.  In this case, for each $ \vec w\in W$,  
        \begin{gather*}
        \vec w = (\vec w \cdot \vec u_1) \vec u_1 + \cdots +  (\vec w \cdot \vec u_p ) \vec u_p
        \\
        \lVert \vec w \rVert  =  \sqrt{(\vec w \cdot \vec u_1) ^2 + \cdots +  (\vec w \cdot \vec u_p) ^2} 
    \end{gather*}
    \end{minipage}};
    \node[fancytitle, right=10pt] at (box.north west) {Definition  (Orthonormal Basis)};
    \end{tikzpicture}\end{center}
\end{frame}




\begin{frame}{Example: Orthonormal Basis}  
    \vspace{-6pt}
    $ W $ is a subspace of $ \mathbb R ^{3}$ that is perpendicular to $x$.  Calculate the missing coefficients in the orthonormal basis for $W$, which is $\{u, v\}$. 
    \begin{equation*}
    x = \spalignmat{1;1;1}, \qquad 
    u = 
    \frac{1}{\sqrt{\quad}}
    \begin{pmatrix}
        1 \\ 0 \\ \phantom {\int\int} 
    \end{pmatrix} ,
    \qquad 
    v= 
    \frac{1}{\sqrt{\quad}}
    \begin{pmatrix}
        \phantom {\int\int}  \\ \phantom {\int\int}  \\ \phantom {\int\int} 
    \end{pmatrix} 
    \end{equation*}
    \Emph{Solution for $u$}\\    
    \pause For $u$ to be in $W$ we need $x\cdot u =0$. \pause 
    $$\text{Set } u=\frac{1}{\sqrt a}\begin{pmatrix} 1\\0\\b \end{pmatrix} \quad \Rightarrow \quad \pause x\cdot u = \frac{1}{\sqrt{a}}(1+0+b)=0 \quad \Rightarrow \quad b = -1$$
    \pause For $u$ to have unit length, we need $a=2$. 
\end{frame}



\begin{frame}{Example: Orthonormal Basis}  
    \vspace{-12pt}
    % $ W $ is a subspace of $ \mathbb R ^{3}$ that is perpendicular to $x$.  Calculate the missing coefficients in the orthonormal basis for $W$, which is $\{u, v\}$. 
    \begin{equation*}
    x = \spalignmat{1;1;1}, \qquad 
    u = 
    \frac{1}{\sqrt{2}}
    \begin{pmatrix}
        1 \\ 0 \\ -1
    \end{pmatrix} ,
    \qquad 
    v= 
    \frac{1}{\sqrt{\quad}}
    \begin{pmatrix}
        \phantom {\int\int}  \\ \phantom {\int\int}  \\ \phantom {\int\int} 
    \end{pmatrix} 
    \end{equation*}
    \pause 
    \Emph{Solution for $v$}\\
    For $u$ and $v$ to form an orthonormal basis, we need $u\cdot v=0$. \pause $$ \text{Set } v=\frac{1}{\sqrt k}\begin{pmatrix} c_1\\c_2\\c_3 \end{pmatrix} \quad \Rightarrow u\cdot v \text{ implies } c_1=c_3$$ If $v$ is in $W$ we need $x\cdot v =0$. \pause 
    $$x\cdot v = \frac{1}{\sqrt{k}}(c_1+c_2+c_3)=0 \quad \Rightarrow \quad c_2 = -2c_1$$
    \pause Choosing $c_1 = 1$, then $c_2 = -2$ and $c_3=1$. \pause For $\|v\| = 1$, we need $k=6$. 
\end{frame}


\begin{frame}{Example: Orthonormal Basis}  
    Our vectors are:
    \begin{equation*}
    x = \spalignmat{1;1;1}, \qquad 
    u = 
    \frac{1}{\sqrt{2}}
    \begin{pmatrix}
        1 \\ 0 \\ -1
    \end{pmatrix} ,
    \qquad 
    v= 
    \frac{1}{\sqrt{6}}
    \begin{pmatrix}
        1 \\-2\\1
    \end{pmatrix} 
    \end{equation*}
    \pause As required, $u$ and $v$ are orthonormal, \pause and they form a basis for $W$, which is the set of vectors orthogonal to $x$. 
\end{frame}



\begin{frame}{Orthogonal Bases}

    Do not forget that bases are not unique. 
    
    \vspace{12pt} 
    
    \pause 
    
    For example, any vector $\vec w \in \mathbb R^3$ can be written as a linear combination of $\{\vec e_1,\vec e_2,\vec e_3\}$, or any other orthogonal basis for $\mathbb R^3$.

    \pause 
\begin{columns}
\begin{column}{.45\textwidth}

    \begin{center}
    \tdplotsetmaincoords{70}{140}
    \begin{tikzpicture}[scale=1.5, axis/.style={-,gray}, 
        vector/.style={-stealth,DarkBlue,very thick},
        tdplot_main_coords
        ]
        
        \coordinate (O) at (0,0,0);

        % draw axes
        \draw[axis] (-1.5,0,0) -- (1.5,0,0) node[anchor=north east]{};
        \draw[axis] (0,-1.5,0) -- (0,1.5,0) node[anchor=north east]{};
        \draw[axis] (0,0,-1.0) -- (0,0,1.5) node[anchor=north west]{};       
        
        % draw bases
        \draw[vector] (O) -- (1,0,0) node[anchor=north]{$\vec e_1$};
        \draw[vector] (O) -- (0,1,0) node[anchor=north east]{$\vec e_2$};
        \draw[vector] (O) -- (0,0,1) node[anchor=north west]{$\vec e_3$};

    \draw (0,0,-1.5) node[above] {{\small an orthogonal basis for $\mathbb R^3$}};
    \end{tikzpicture}
    \end{center}

\end{column}\begin{column}{.45\textwidth}
    
    \begin{center}
    \tdplotsetmaincoords{70}{140}
    \begin{tikzpicture}[scale=1.5, axis/.style={-,gray}, 
        vector/.style={-stealth,DarkBlue,very thick},
        tdplot_main_coords
        ]
        
        \coordinate (O) at (0,0,0);

        % draw axes
        \draw[axis] (-1.5,0,0) -- (1.5,0,0) node[anchor=north east]{};
        \draw[axis] (0,-1.5,0) -- (0,1.5,0) node[anchor=north east]{};
        \draw[axis] (0,0,-1.0) -- (0,0,1.5) node[anchor=north west]{};       
        
        % draw bases
        \draw[vector] (O) -- (1,0,0) node[anchor=north]{$\vec e_1$};
        \draw[vector] (O) -- (0,-1,0) node[anchor= south]{$-\vec e_2$};
        \draw[vector] (O) -- (0,0,1) node[anchor=north west]{$\vec e_3$};
        \draw (0,0,-1.5) node[above] {{\small another orthogonal basis for $\mathbb R^3$}};
    \end{tikzpicture}
    \end{center}

\end{column}
\end{columns}

\end{frame}

\frame{\frametitle{Summary}

    \SummaryLine \vspace{4pt}
    \begin{itemize}\setlength{\itemsep}{8pt}

    \item orthogonal sets of vectors

    \item orthogonal and orthonormal bases 
    
    \end{itemize}
    
    \vspace{8pt}
    
    \pause 
}

