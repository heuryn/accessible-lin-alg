\title{The QR Factorization}
\subtitle{\SubTitleName}
\institute[]{\Course}
\author{\Instructor}
\maketitle   
  


\begin{frame}\frametitle{Topics and Objectives}
    \Emph{Topics} \\
    %\TopicStatement
    \begin{itemize}
        % \item an introduction to the Gram Schmidt Process for constructing an orthogonal basis for a subspace
        \item  The QR factorization of matrices and its properties
    \end{itemize}
    
    \vspace{0.5cm}
    
    \Emph{Learning Objectives}\\
    
    %\LearningObjectiveStatement
    
    \begin{itemize}
    % \item  apply the Gram Schmidt Process to construct an orthogonal basis for a subspace spanned by two vectors in $\mathbb R^n$
    
    \item  Construct the QR factorization of a matrix.
      
    \end{itemize}
    
    \vspace{0.25cm} 
 
\end{frame}



\begin{frame}{Orthonormal Bases} 

    Earlier in this course we defined an orthonormal basis. 
    
    \pause 
    
    % ~~ ~~ Highlight Box ~~ ~~
    \begin{center}\begin{tikzpicture} \node [mybox](box){\begin{minipage}{0.85\textwidth}\vspace{2pt}

        A set of vectors form an \Emph{orthonormal basis} if the vectors are mutually orthogonal and have unit length.  

    \end{minipage}};
    \node[fancytitle, right=10pt] at (box.north west) {Definition};
    \end{tikzpicture}\end{center}
    % ~~ ~~ Highlight Box ~~ ~~
    \pause 
    \Emph{Example} \\
    The two vectors below form an orthogonal basis for a subspace $W$. Obtain an orthonormal basis for $W$. 
    \begin{equation*}
        \vec v_1 =\begin{pmatrix} 3 \\ 2 \\ 0 \end{pmatrix}, 
        \quad 
        \vec v_2 =\begin{pmatrix} -2 \\ 3 \\ 1 \end{pmatrix}. 
    \end{equation*}
\end{frame}



\begin{frame}{QR Factorization}
\begin{center}\begin{tikzpicture} \node [mybox](box){\begin{minipage}{0.95\textwidth}
\vspace{2pt}

    Any  $ m \times n $ matrix $A$ with linearly independent columns has the \Emph{QR factorization}
    \begin{equation*}
        A = Q R 
    \end{equation*}
    where 

    \begin{itemize}
        \item $ Q$ is $ m \times n$, its columns are an orthonormal basis for $ \operatorname {Col} A$. 
        \item $ R$ is $ n \times n$, upper triangular, with positive entries on its diagonal % and 
    \end{itemize}

    \end{minipage}};
    \node[fancytitle, right=10pt] at (box.north west) {Theorem};
    \end{tikzpicture}\end{center}
    
%     In the interest of time:
%     \begin{itemize}
%     	\item we will not consider the case where $A$ has linearly dependent columns
% 	\item students are not expected to know the conditions for which $A$ has a QR factorization
%     \end{itemize}
\end{frame}


\begin{frame}{Notes on the QR Factorization}

    \begin{itemize}
        \item We are not considering the case when $A$ has linearly dependent columns.
        
        \item<2->$Q$ can be obtained using a Gram-Schmidt process.
        
        \item<3-> To obtain $R$, we can use $R = Q^TA$, because
        \begin{align*}
            \onslide<4->{A &= QR \\}
            \onslide<5->{Q^TA &= Q^TQ R, \quad \text{but } Q^TQ = I \\}
            \onslide<6->{Q^T A &= R}
        \end{align*}
        \item<7-> The length of the $ j^{th}$ column of $R$ is equal to the length of the $j^{th}$ column of $A$.
        
    \end{itemize}
\end{frame}

\begin{frame}{Example} 

    Construct the $ QR$ decomposition for $A = \begin{pmatrix}
        3 & -2  \\ 2 & 3  \\ 0 & 1 
        \end{pmatrix}$.
    \pause
    
    \Emph{Solution}\\
    The columns of $Q$ form an orthonormal basis for $\Col A$, but the columns of $A$ are already orthogonal (if they were not, we could use Gram-Schmidt). 
    
    \vspace{12pt}
    \pause 
    Construct $Q$ by dividing each column by its respective length. 
    $$Q = \spalignmat{3/\sqrt{13} -2/\sqrt{14};2/\sqrt{13} 3/\sqrt{14};0 1/\sqrt{14}}$$
    Next we construct $R$. 
    
\end{frame}

\begin{frame}{Example} 

    As we saw, $R = Q^TA$. \pause 
    $$R = Q^TA = \spalignmat{3/\sqrt{13} 2/\sqrt{13} 0 ;-2/\sqrt{14} 3/\sqrt{14} 1/\sqrt{14}} \begin{pmatrix}
        3 & -2  \\ 2 & 3  \\ 0 & 1 
    \end{pmatrix} = \begin{pmatrix} 13/\sqrt{13} & 0\\0 &14/\sqrt{14}\end{pmatrix}$$
    \pause 
    Thus, 
    $$A = QR = \spalignmat{3/\sqrt{13} -2/\sqrt{14};2/\sqrt{13} 3/\sqrt{14};0 1/\sqrt{14}} \begin{pmatrix} 13/\sqrt{13} & 0\\0 &14/\sqrt{14}\end{pmatrix} $$
\end{frame}





 \frame{\frametitle{Summary}

    \SummaryLine \vspace{4pt}
    \begin{itemize}\setlength{\itemsep}{8pt}

    \item the QR factorization and how to compute it
    \item properties of matrices $Q$ and $R$ in the QR factorization

    \end{itemize}
    
    \vspace{16pt}
    \pause 
    
    
}

