\title{An Introduction to the Gram-Schmidt Process}
\subtitle{\SubTitleName}
\institute[]{\Course}
\author{\Instructor}
\maketitle   
  



 
\begin{frame}\frametitle{Topics and Objectives}
    \Emph{Topics} \\
    %\TopicStatement
    \begin{itemize}
        \item an introduction to the Gram-Schmidt Process for constructing an orthogonal basis for a subspace
        % \item  The $QR$ decomposition of matrices and its properties
    \end{itemize}
    
    \vspace{0.5cm}
    
    \Emph{Learning Objectives}\\
    
    %\LearningObjectiveStatement
    
    \begin{itemize}
    \item  apply the Gram-Schmidt Process to construct an orthogonal basis for a subspace spanned by two vectors in $\mathbb R^n$
    
    % \item  Compute the $ QR $ factorization of a matrix.
      
    \end{itemize}
    
    \vspace{0.25cm} 
 
\end{frame}



\begin{frame}{Motivating Questions}

Suppose $\vec x_1 = \spalignmat{1;1;0}$ and $\vec x_2 = \spalignmat{0;2;1}$.

\begin{center}
    \tdplotsetmaincoords{70}{110}
    \begin{tikzpicture}[scale=1.2,
        axis/.style={->,black,-stealth}, 
        vectorV/.style={-stealth,DarkBlue,very thick},
        vectorX/.style={-stealth,DarkBlue, thick},
        perpline/.style={DarkBlue, thin},
        tdplot_main_coords
        ]
        
        \coordinate (O) at (0,0,0);
        \coordinate (X1) at (1,1,0);        
        \coordinate (X2) at (0,2,1);        

        \draw[axis] (-2,0,0) -- (2,0,0) node[right]{$x$};
        \draw[axis] (0,-1,0) -- (0,2.5,0) node[right]{$y$};
        \draw[axis] (0,0,-1) -- (0,0,1.5) node[right]{$z$};

        % draw x vectors
        \draw[vectorX] (O) -- (X1) node[below]{$\vec x_1$};
        \draw[vectorX] (O) -- (X2) node[right]{$\vec x_2$};

        % % draw q vectors
        % \draw[vectorV] (O) -- (.7,.7,0) node[below]{$\vec q_1$};
        % \draw[vectorV] (O) -- (0,1,0) node[above]{$\vec q_2$};

        % % draw some of the perpendicular symbols
        % \draw[perpline] (0.2,0.2,0.0) -- (0.2,0.0,0.0);
        % \draw[perpline] (0.2,0.2,0.0) -- (0.0,0.2,0.0);        
        
         
    \end{tikzpicture}
    
    \pause $\vec x_1$ and $\vec x_2$ are linearly independent \pause $\Rightarrow$ they form a basis for $W = \Span\{\vec x_1, \vec x_2\}$
    
    \vspace{4pt}
    \pause but $\vec x_1$ and $\vec x_2$ do not give an \Emph{orthogonal} basis for $W$
    
    \vspace{4pt}
    \pause how might we construct an orthogonal basis for $W$? 
    
\end{center}


\end{frame}




\begin{frame}{Example}  

    Let $ W $ be the subspace of $ \mathbb R ^{3}$ spanned by $ \vec x_1$ and $ \vec x_2$. Construct an orthogonal basis for $ W$. 
    $$\vec x_1 = \spalignmat{1;1;0}, \quad \vec x_2 = \spalignmat{0;2;1}$$

    \pause 
\end{frame}


\begin{frame}\frametitle{The Gram-Schmidt Process with Two Vectors} 
    Suppose we are given a set of vectors $ \{\vec x_1 ,\vec x_2\}$ in $ \mathbb R ^{n}$. \onslide<2->{We can construct an orthogonal basis, $\{\vec v_1, \vec v_2\}$ for the space that they span, $W$, with the following process.}
    \begin{align*}
        \onslide<3->{\vec v_1 & = \vec x_1}
        \\
        \onslide<4->{\vec v_2 & = \vec x_2 - \frac {\vec x_2 \cdot \vec v_1} {\vec v_1 \cdot \vec v_1} \vec v_1}
    \end{align*}
    \onslide<5->{We can show that if $ \{\vec x_1 ,\vec x_2\}$ are independent, then $ \{\vec v_1 ,\vec v_2\} $ is an orthogonal basis for $W$:}
    \begin{itemize}
        \item<6-> $\vec v_1$ and $\vec v_2$ are in $W$
        \item<7-> $\vec v_1$ and $\vec v_2$ span $W$
        \item<8-> $\vec v_1$ and $\vec v_2$ are orthogonal: $\vec v_1 \cdot \vec v_2 = 0$ 
    \end{itemize}
    
\end{frame}


\begin{frame}\frametitle{The Gram-Schmidt Process with Two Vectors} 
    
    What happens if $ \{\vec x_1 ,\vec x_2\}$ are non-zero vectors, but are linearly dependent? Does $\vec v_2$ have a particular value? 
    
\end{frame}




 \frame{\frametitle{Summary}

    \SummaryLine \vspace{4pt}
    \begin{itemize}\setlength{\itemsep}{8pt}

    \item an introduction to the Gram-Schmidt Process for constructing an orthogonal basis for a subspace

    \end{itemize}
    
    \vspace{16pt}
    \pause 
    
    We used this process to construct an orthogonal basis for a subspace spanned by \Emph{two} vectors in $\mathbb R^n$.

}
