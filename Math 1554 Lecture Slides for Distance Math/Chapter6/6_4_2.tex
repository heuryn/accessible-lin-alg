\title{The Gram-Schmidt Process}
\subtitle{\SubTitleName}
\institute[]{\Course}
\author{\Instructor}
\maketitle   
  


\begin{frame}\frametitle{Topics and Objectives}
    \Emph{Topics} \\
    %\TopicStatement
    \begin{itemize}
        \item an introduction to the Gram-Schmidt Process for constructing an orthogonal basis for a subspace
        % \item  The $QR$ decomposition of matrices and its properties
    \end{itemize}
    
    \vspace{0.5cm}
    
    \Emph{Learning Objectives}\\
    
    %\LearningObjectiveStatement
    
    \begin{itemize}
    \item  apply the Gram-Schmidt Process to construct an orthogonal basis for a subspace spanned by $p$ vectors in $\mathbb R^n$
    
    % \item  Compute the $ QR $ factorization of a matrix.
      
    \end{itemize}
    
    \vspace{0.25cm} 
 
\end{frame}






\begin{frame}{Example}
    The vectors below span a subspace $W$ of $\mathbb R^{4}$. Construct an orthogonal basis for $W$. 
    \begin{equation*}
        \vec x_1 = \begin{pmatrix}
        1 \\ 1\\ 1\\ 0 
        \end{pmatrix}, \quad 
        \vec x_2 = \begin{pmatrix}
        0 \\ 1\\ 2 \\ 1 
        \end{pmatrix}, \quad 
        \vec x_3 = \begin{pmatrix}
        0 \\ 1 \\ -1 \\ 0 
    \end{pmatrix}. 
    \end{equation*}


\end{frame}

\begin{frame}\frametitle{The Gram-Schmidt Process} 
\onslide<1->{Suppose $ \{\vec x_1 ,\dotsc, \vec x_p\}$ are a basis for a subspace $ W$ of $ \mathbb R ^{n}$.  }
\begin{align*}
    \onslide<2->{W_1 &= \Span\{\vec v_1\}  && \vec v_1 = \vec x_1 \\}
    \onslide<3->{W_2 &= \Span\{\vec v_1, \vec v_2\}  && \vec v_2 = \vec x_2 - \proj_{W_1}\vec x_2 \\}
    \onslide<4->{W_3 &= \Span\{\vec v_1, \vec v_2,\vec v_3\} && \vec v_3 = \vec x_3 - \proj_{W_2}\vec x_3 \\
    & \vdots && \vdots \\}
    \onslide<5->{W_p &= \Span\{\vec v_1 \ldots \vec v_p\} && \vec v_p = \vec x_p - \proj_{W_{p-1}}\vec x_p}
\end{align*}
\onslide<6->{Then, $ \{\vec v_1 ,\dotsc, \vec v_p\} $ is an \Emph{orthogonal} basis for $W$. }
\end{frame}


\begin{frame}\frametitle{The Gram-Schmidt Process} 
The Gram-Schmidt process can also be written this way. 
\begin{align*}
\vec v_1 & = \vec x_1 
\\
 \vec v_2 & = \vec x_2 - \frac {\vec x_2 \cdot \vec v_1} {\vec v_1 \cdot \vec v_1} \vec v_1 
\\
\vec v_3 & = \vec x_3 - \frac {\vec x_3 \cdot \vec v_1} {\vec v_1 \cdot \vec v_1} \vec v_1
    - \frac {\vec x_3 \cdot \vec v_2} {\vec v_2 \cdot \vec v_2} \vec v_2
\\
& \vdots 
\\
\vec v_p & = 
\vec x_p - \frac {\vec x_p \cdot \vec v_1} {\vec v_1 \cdot \vec v_1} \vec v_1
   - \cdots  - \frac {\vec x_p \cdot \vec v_ {p-1}} {\vec v_{p-1} \cdot \vec v_{p-1}} \vec v_{p-1}
\end{align*}

\end{frame}



 \frame{\frametitle{Summary}

    \SummaryLine \vspace{4pt}
    \begin{itemize}\setlength{\itemsep}{8pt}

    \item the Gram-Schmidt Process for constructing an orthogonal basis for a subspace

    \end{itemize}
    
    \vspace{16pt}
    \pause 
    
    We used this process to construct an orthogonal basis for a subspace spanned by $p$ vectors in $\mathbb R^n$.

    
}





% \begin{frame}{Geometric Interpretation}

%     \onslide<1->{Suppose $\vec x_1, \vec x_2, \vec x_3$ are linearly independent vectors in $\mathbb R^n$. We wish to construct an orthogonal basis for the space that they span.} 
    
% \begin{center}
%     \tdplotsetmaincoords{70}{40}
%     \begin{tikzpicture}[scale=1.3,
%         axis/.style={->,black,-stealth}, 
%         vectorP/.style={-stealth,black, thick},
%         vectorV/.style={-stealth,DarkBlue,very thick},
%         vectorX/.style={-stealth,DarkRed,very thick},
%         dsline/.style={black,dashed},
%         perpline/.style={DarkBlue, thin},
%         tdplot_main_coords
%         ]
        
%         \coordinate (O) at (0,0,0);
%         \coordinate (Y) at (1,0,1);
%         \coordinate (H) at (1,0,0);        
%         \coordinate (X1) at (3,1,0);        
%         \coordinate (X2) at (1,2.6,0);        
%         \coordinate (X3) at (1.9,1,1.8);        
%         \coordinate (P) at (1.9,1,0);        
%         \coordinate (W) at (.5,-1.8,0);        
        
%         % % plane
%         % \filldraw[
%         % draw=DarkBlue!20,fill=DarkBlue!05,]          
%         %     (-1,-2,0) 
%         %     -- (3,-2,0)
%         %     -- (3,4,0)
%         %     -- (-1,4,0)
%         %     -- cycle;

%         \draw[axis] (-2,0,0) -- (3.5,0,0) ;
%         \draw[axis] (0,-1,0) -- (0,3.5,0) ;
%         \draw[axis] (0,0,-1) -- (0,0,1.5) ;

%         % GIVEN
%         \onslide<2-2>{
%         \draw[vectorX] (O) -- (X1) node[below]{$\vec x_1$};
%         }
%         \onslide<2->{        
%         \draw[vectorX] (O) -- (X2) node[right]{$\vec x_2$};
%         \draw[vectorX] (O) -- (X3) node[right]{$\vec x_3$};
%         }
        
%         % W1
%         \onslide<3->{
%         \draw[dsline,DarkBlue] (-3,0,0) -- (4,0,0) node[below]{$W_1$};
%         \draw[vectorV] (O) -- (X1) node[below]{$\vec x_1=\vec v_1$};
%         }

        
%         % W2 
%         \onslide<4->{
%         \draw[dsline,black] (X2) -- (X1);
%         \draw[vectorV] (O) -- (0,2.6,0) node[above]{$\vec v_2$};
%         \draw[DarkBlue] (W) node[above]{$W_2$};        
%         }
        
%         % W3
%         \onslide<6->{
%         \draw[dsline,black] (P) -- (X3);
%         \draw[vectorP] (O) -- (P) node[right]{proj$_{W_2}\vec x_3$};    
%         \draw[vectorV] (O) -- (0,0,1.8) node[above]{$\vec v_3$};        
%         }
        

        
%         % % draw some of the perpendicular symbols
%         % \draw[perpline] (0.0,0.0,0.2) -- (0.2,0.0,0.2);
%         % \draw[perpline] (0.2,0.0,0.2) -- (0.2,0.0,0.0);
        
%         % \draw[perpline] (0.0,0.0,0.2) -- (0.0,0.2,0.2);
%         % \draw[perpline] (0.0,0.2,0.2) -- (0.0,0.2,0.0);
        
%     \end{tikzpicture}
    
%     \vspace{12pt} 
    
%     We construct vectors $\color{DarkBlue} \vec v_1, \vec v_2, \vec v_3$, which form our \Emph{orthogonal} basis. 
    
%     \end{center}
% \end{frame}

