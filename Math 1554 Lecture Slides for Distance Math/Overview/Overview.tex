\documentclass[xcolor=table,11pt,notes=hide,t,handout]{beamer} 

% FOOTER SPECIFICATIONS
\defbeamertemplate*{footline}{}
{
  \leavevmode
  \hbox{
  \begin{beamercolorbox}[wd=0.45\paperwidth,ht=2.95ex,dp=2ex,left]{}
    \hspace*{2ex} Slide \insertframenumber{}  
  \end{beamercolorbox}}
    \hbox{
  \begin{beamercolorbox}[wd=0.5\paperwidth,ht=2.95ex,dp=2ex,right]{}
    \hspace*{2ex} 
  \end{beamercolorbox}}
  \vskip0pt
}

\usefonttheme[onlymath]{serif} %makes math characters serif, but also increases size of non-math characters



% CUSTOMIZED COMMANDS
% LOAD PACKAGES
\usepackage{amsmath} % allows for align env and other things
\usepackage{graphicx} % allows for graphics
\usepackage{textcomp} % allows for single apostrophe
%\usepackage{enumitem} % allows for alpha lettering in enumerated lists
\usepackage{amsfonts} % for real numbers symbol and other sets
\usepackage[mathscr]{euscript} % Euler script font
\usepackage{wrapfig} % to allow text wrapping
\usepackage{mathtools}
\usepackage{multirow}
\usepackage{pgfplots} % for surfaces (chapter 7), 2D graphics, etc
\usepackage{tikz-3dplot} 
\pgfplotsset{compat=1.9}
\usepackage{colortbl}
\usepackage{array}
% ~ ~ ~ ~ ~ ~ ~ ~ ~ ~ ~ ~ ~ ~ ~ ~ ~ ~ ~ ~ ~ ~ ~ ~ ~ ~ ~ 
% FONT
\usefonttheme[onlymath]{serif} %makes math characters serif, but also increases size of non-math characters
% ~ ~ ~ ~ ~ ~ ~ ~ ~ ~ ~ ~ ~ ~ ~ ~ ~ ~ ~ ~ ~ ~ ~ ~ ~ ~ ~ 
% CUSTOMIZED EMPHASIS
\definecolor{Black}{rgb}{0.0,0.0,0.0} % 
\definecolor{Teal}{rgb}{0.1,0.2,0.4}% 
\definecolor{Grey}{rgb}{0.4,0.4,0.4} % 
\newcommand{\Emph}[1]{{\color{Teal}\textbf{#1}}} % not needed
\DeclareTextFontCommand{\emph}{\bfseries\em}
% ~ ~ ~ ~ ~ ~ ~ ~ ~ ~ ~ ~ ~ ~ ~ ~ ~ ~ ~ ~ ~ ~ ~ ~ ~ ~ ~ 
% HEADER STYLE 
% THIS IS THE COLOR OF HEADER BAR ON TOP OF LECTURE SLIDES
\definecolor{HeaderBlue}{rgb}{0.051,0.051,0.051} % 
% Header Font Style
\setbeamerfont{frametitle}{size=\large\bfseries\color{Teal}}
% HEADER SPACING
\setbeamertemplate{headline}{\vskip3pt} % SPACE ABOVE HEADER
\addtobeamertemplate{frametitle}{}{\vspace*{0.3cm}} % SPACE AFTER HEADER
% ~ ~ ~ ~ ~ ~ ~ ~ ~ ~ ~ ~ ~ ~ ~ ~ ~ ~ ~ ~ ~ ~ ~ ~ ~ ~ ~ 
% COLORS FOR DIAGRAMS
\definecolor{DarkBlue}{rgb}{0.0,0.0,0.6} % 
\definecolor{DarkGreen}{rgb}{0.0,0.3,0.0} % 
\definecolor{DarkRed}{rgb}{0.6,0.0,0.0} % 
\definecolor{LightBlue}{rgb}{0.0,0.5,1.0} % 
\definecolor{Orange}{rgb}{1.0,0.5,0.0} % 
\definecolor{Gold}{rgb}{0.8,0.6,0.1} % 
\definecolor{LightGreen}{rgb}{0.3,0.9,0.3} % 
% ~ ~ ~ ~ ~ ~ ~ ~ ~ ~ ~ ~ ~ ~ ~ ~ ~ ~ ~ ~ ~ ~ ~ ~ ~ ~ ~ 
% FORMATTING OF BULLET AND ENUMERATED LISTS
\setbeamertemplate{itemize item}{$\bullet$}
\setbeamertemplate{itemize subitem}{$\bullet$}
\setbeamercolor{itemize item}{fg=Teal}
\setbeamercolor{itemize subitem}{fg=Teal}
% \setbeamercolor{itemize subitem}{fg=gray}
\setbeamercolor{enumerate item}{fg=Teal}
\setbeamercolor{enumerate subitem}{fg=gray}
% ~ ~ ~ ~ ~ ~ ~ ~ ~ ~ ~ ~ ~ ~ ~ ~ ~ ~ ~ ~ ~ ~ ~ ~ ~ ~ ~ 
% Topic and Learning Objective Statement
\newcommand{\LO}{\Emph{\color{Teal}Learning Objectives}}
\newcommand{\TopicStatement}{We will explore the following concepts in this video.}
\newcommand{\LearningObjectiveStatement}{After watching this video you should be able to:}
\newcommand{\LearningObjectiveStatementModule}{After completing the learning activities in this module students should be able to do the following.}
\newcommand{\LearningObjectiveStatementCourse}{After completing the learning activities in this course students should be able to do the following.}

\newcommand{\SummaryLine}{We explored the following concepts in this video.}
% ~ ~ ~ ~ ~ ~ ~ ~ ~ ~ ~ ~ ~ ~ ~ ~ ~ ~ ~ ~ ~ ~ ~ ~ ~ ~ ~ 
% COLORS IN LECTURE TITLE SLIDES 
%(the slide that appears at start of each lecture)
\setbeamercolor{frametitle}{fg=Black}
\setbeamercolor{slidetitle}{fg=Black}
\setbeamercolor*{title}{fg=black!80} % color of title slide title
% ~ ~ ~ ~ ~ ~ ~ ~ ~ ~ ~ ~ ~ ~ ~ ~ ~ ~ ~ ~ ~ ~ ~ ~ ~ ~ ~ 
% TITLE SLIDES 
\newcommand{\Course}{\textbf{Linear Algebra}}
\newcommand{\Instructor}{{\large {\color{Gold}{Greg Mayer, Ph.D.}}}\\[2pt] Academic Professional\\{\small {\color{gray} School of Mathematics}}}
\setbeamercolor{title}{fg=black}
\setbeamercolor{subtitle}{fg=Teal}
\setbeamertemplate{title page}[default][left,colsep=-12bp,rounded=true]
% ~ ~ ~ ~ ~ ~ ~ ~ ~ ~ ~ ~ ~ ~ ~ ~ ~ ~ ~ ~ ~ ~ ~ ~ ~ ~ ~ 
% COLORED CIRCLES FOR DIAGRAMS
\newcommand{\RedCircle}[2][black,fill=DarkRed]{\tikz[baseline=-0.5ex]\draw[#1,radius=#2] (0,0) circle ;}%
\newcommand{\BlueCircle}[2][black,fill=LightBlue]{\tikz[baseline=-0.5ex]\draw[#1,radius=#2] (0,0) circle ;}%
\newcommand{\GreenCircle}[2][black,fill=LightGreen]{\tikz[baseline=-0.5ex]\draw[#1,radius=#2] (0,0) circle ;}%
\newcommand{\OrangeCircle}[2][black,fill=Orange]{\tikz[baseline=-0.5ex]\draw[#1,radius=#2] (0,0) circle ;}%
% ~ ~ ~ ~ ~ ~ ~ ~ ~ ~ ~ ~ ~ ~ ~ ~ ~ ~ ~ ~ ~ ~ ~ ~ ~ ~ ~ 
% COLOURED BOXES FOR DEFINITIONS AND THEOREMS
\usepackage{tikz}
\usetikzlibrary{shapes,snakes}   
\usetikzlibrary{arrows,automata}
% ~ ~ ~ ~ ~ ~ ~ ~ ~ ~ ~ ~ ~ ~ ~ ~ ~ ~ ~ ~ ~ ~ ~ ~ ~ ~ ~ 
% WATERMARKS
\usepackage{lipsum}
\usepackage{tikz}
\setbeamertemplate{background}{%
\begin{tikzpicture}[overlay,remember picture]
\node[anchor=north west,scale=1.5] at ([shift={(-0.2in,0.2in)}]current page.north west) {\includegraphics[width=0.4\textwidth]{images/GTHexes.png} };
\node[anchor=north west,scale=0.25] at ([shift={(5.5in,-0.1in)}]current page.north west) {\includegraphics[width=0.4\textwidth]{images/GT-Logo.png} };
\end{tikzpicture}%
}
% ~ ~ ~ ~ ~ ~ ~ ~ ~ ~ ~ ~ ~ ~ ~ ~ ~ ~ ~ ~ ~ ~ ~ ~ ~ ~ ~ 
% REMOVE NAV BAR
\beamertemplatenavigationsymbolsempty
% ~ ~ ~ ~ ~ ~ ~ ~ ~ ~ ~ ~ ~ ~ ~ ~ ~ ~ ~ ~ ~ ~ ~ ~ ~ ~ ~ 
% CUSTOM TITLE SLIDE FOR GTPE
\setbeamertemplate{title page}{%
  \vbox{}
    \begin{flushleft}
    \begin{beamercolorbox}[sep=8pt]{title}
      \usebeamerfont{title}{\textbf{\insertinstitute}}\par%
      \ifx\insertsubtitle\@empty%
      \else%
        \vskip0.25em%
        {\usebeamerfont{subtitle}\usebeamercolor[fg]{subtitle}\insertsubtitle\par}%
      \fi%     
    \end{beamercolorbox}%
    \vskip1em\par
    \vfill%<- added
    \begin{beamercolorbox}[sep=8pt]{author}
      \usebeamerfont{author}\insertauthor
    \end{beamercolorbox}
    \vfill%<- added
    \begin{beamercolorbox}[sep=8pt]{institute}
      \usebeamerfont{institute}{{\color{Teal}\Large \inserttitle}}
    \end{beamercolorbox}
    \end{flushleft}
}
% ~ ~ ~ ~ ~ ~ ~ ~ ~ ~ ~ ~ ~ ~ ~ ~ ~ ~ ~ ~ ~ ~ ~ ~ ~ ~ ~ 
% ASPECT RATIO
\usepackage[orientation=landscape,size=custom,width=16,height=9,scale=0.5,debug]{beamerposter} 
% ~ ~ ~ ~ ~ ~ ~ ~ ~ ~ ~ ~ ~ ~ ~ ~ ~ ~ ~ ~ ~ ~ ~ ~ ~ ~ ~ 
% MATH THING: AUGMENTED MATRIX
% augmeted matrix environment, from Hefferon
\newenvironment{amatrix}[1]{%
  \left(\begin{array}{@{}*{#1}{c}|c@{}}
}{%
  \end{array}\right)
}
% ~ ~ ~ ~ ~ ~ ~ ~ ~ ~ ~ ~ ~ ~ ~ ~ ~ ~ ~ ~ ~ ~ ~ ~ ~ ~ ~ 
% OTHER MATH THINGS
\newcommand{\R}{{\mathbb R}} % The real numbers
\usepackage{spalign} % Joe Rabinoff's matrix package
% SUBSPACES AND ORTHOGONALITY
\newcommand{\Perp}{^{\perp}}
\newcommand{\Row}{\text{Row}}
\newcommand{\Col}{\text{Col}}
\newcommand{\Nul}{\text{Nul}}
\newcommand{\Null}{\text{Nul}}
\newcommand{\proj}{\text{proj}}
\newcommand{\Span}{\text{Span}}
\newcommand{\Rank}{\text{rank}}

\usetikzlibrary{decorations.pathreplacing,calligraphy, matrix, backgrounds, positioning, calc}


% ~ ~ ~ ~ ~ ~ ~ ~ ~ ~ ~ ~ ~ ~ ~ ~ ~ ~ ~ ~ ~ ~ ~ ~ ~ ~ ~ 
\begin{document}

% COLORED BOX FORMATTING
% These boxes are used for definitions and theorems 
% This code has to appear after begin{document}
\tikzstyle{mybox} = [draw=black, fill=gray!2, very thick, rectangle, rounded corners, inner sep=10pt, inner ysep=10pt]
\tikzstyle{fancytitle} =[draw=black, fill=HeaderBlue!10, text=black]


\title{
MATH 1554 Course Overview \\
\vspace{24pt} 
{\small School of Mathematics, Georgia Tech \\[2pt]}

\vspace{1cm} 

}


\date{} 
\maketitle

\setcounter{section}{1}

\begin{frame}\frametitle{What is MATH 2552?}

    \textbf{One Overall Goal}
    \begin{center}
        \textit{prepare students for more advanced courses that use Math 2552 as a pre-requisite.}
    \end{center}
    Math 2552 is:
    \begin{itemize}
        \item a 4 credit course
        \item a course that tends to focus more on applications and procedures
        \item combines linear algebra with integral calculus
    \end{itemize}
    \vspace{12pt}
    \textbf{Topics Include}\\
    Linear systems, determinants, eigenvalue problems, orthogonality and least squares, matrix decompositions, Google Page rank.

\end{frame}


\begin{frame}\frametitle{Grade Breakdown}

\begin{itemize}
    \item 6\% Exploration, 10\% Homework, 10\% Quizzes, 54\% Midterms, 20\% Final Exam
    \item GT only issues letter grades on transcripts
    \item Numerical grades converted to letters using standard intervals:\begin{itemize}
        \item A: [90\%, 100\%]
        \item B: [80\%, 90\%)
        \item C: [70\%, 80\%)
        \item D: [60\%, 70\%)
        \item F: [0\%, 60\%)
    \end{itemize}
    For example, a final grade of 89.9999\% is a B. 
\end{itemize}
\end{frame}



\begin{frame}\frametitle{Midterms and Quizzes}

    
    \textbf{Dates}
    \begin{itemize}
        \item Midterms: 1/30, 2/27, 4/2
        \item Quizzes: 1/16, 2/13, 3/12, 4/16
    \end{itemize}
    \textbf{Logistics}
    \begin{itemize}
        \item Midterms are 50 minutes, Quizzes are 20 minutes 
        \item The topics covered stated in syllabus
        \item No formula sheet, closed book, no electronic devices.
        \item All exams graded and returned electronically to their GT account when grades are released via Gradescope. 
        \item If unable to take midterm or quiz (illness, athletics, etc.) contact me ASAP, you \textbf{may} be able to write make-up, details in syllabus. 
        \item Sample quizzes/midterms give you rough idea of  the format/length/question types
    \end{itemize}

\end{frame}



\begin{frame}\frametitle{Final Exam}

    \begin{itemize}
        \item common final exam: every instructor uses the same exam
        \item Tuesday Apr 28 at 6:00 pm
        \item location will be determined by Office of the Registrar and announced towards end of semester
        \item final exam not returned but grades will be posted to canvas
    \end{itemize}
    \textit{Final exam dates for every course taught at GT this semester are already posted on a public website that the Registrar maintains. }
\end{frame}



% \begin{frame}\frametitle{Related Websites}

%     \begin{itemize}
%         \item Canvas: grades, announcements, common syllabus, documents (lecture slides, worksheets, exam review materials)
%         \item Homework and Textbook: link to MyMathLab in Canvas 
%         \item 1554 Master Website: https://gatech.instructure.com/courses/114544
%         \begin{itemize}
%             \item everything here also on our Canvas page for our class
%             \item public website with worksheets, schedule, common syllabus
%         \end{itemize}

%     \end{itemize}

% \end{frame}



% \begin{frame}\frametitle{Piazza}

%     Online forum/discussion.

%     \begin{itemize}
%         \item Can access Piazza from Canvas.
%         \item Please use Piazza:
%         \begin{itemize}
%             \item to ask and answer questions related to the course
%             \item in a positive and constructive manner
%             \item All 1554 classes use the same Piazza site
%         \end{itemize}
%     \end{itemize}
%     Note that different instructors use different Exploration activities, office hours, TAs, etc. 

% \end{frame}



\begin{frame}\frametitle{Textbook and Online Homework}

    \begin{itemize}
        \item Lay, Algebra and Its Applications, 5th edition. 
        \item Bundled with another textbook you will use for Multivariable Calculus (MATH 2550 or 2551)
        \item Because we are using Canvas, you do not need a Course ID. 
        \item If you are asked for a Course ID, or get stuck at any point in accessing mymathlab, let us know on Piazza, we will help you through getting started 
        \item Homework tends to be due before material is covered in studio. This is intentional: we want students to be more prepared for studio. 
    \end{itemize}

\end{frame}








\begin{frame}\frametitle{FAQ}

    Can we have written solutions to worksheets? 
    \begin{itemize}
        \item No: the goal of 1554 is to prepare you for upper level courses. We want students to get used to solving exercises that do not have solutions. 
    \end{itemize}
    If I need/want help with something, what is the best way to reach out? 
    \begin{itemize}
        \item Technical issues: Piazza
        \item Conceptual questions: Piazza, TA or instructor office hours
    \end{itemize}

\end{frame}



\begin{frame}\frametitle{How to Succeed in MATH 1554}

    For many students, 1554 is their first college level math course, and their first math course after integral calculus 
    \begin{center}
    	\textit{some students may need to adjust their study strategies to this course}
	\end{center}
    General advice:
    \begin{itemize}
        \item Find ways to make your studying more efficient:
        \begin{itemize}
            \item review the textbook and start homework in advance of lectures
            \item ask questions on Piazza and/or office hours
        \end{itemize}
        \item prepare for exams by:
        \begin{itemize}
            \item solving many problems, check your answers with others
            \item solving all exam review materials
            \item solve additional problems from textbook
            \item review studio worksheet problems
            \item spread studying out over many days
        \end{itemize}
        
    \end{itemize}


\end{frame}








\begin{frame}\frametitle{Your Instructor}

    Office hours: \\
    \vspace{24pt}
    
    TA information: \\
    \vspace{24pt}
    
    Exploration activities: 

\end{frame}


\end{document}
