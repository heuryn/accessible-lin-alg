\title{The SVD and the Condition Number of a Matrix}
\subtitle{\SubTitleName}
\institute[]{\Course}
\author{\Instructor}
\maketitle   


\frame{\frametitle{Topics and Objectives}
\Emph{Topics} \\
%\TopicStatement
\begin{itemize}

    \item the Singular Value Decomposition (SVD) of a matrix and the condition number of a matrix

\end{itemize}

\vspace{0.5cm}

\Emph{Learning Objectives}\\

%\LearningObjectiveStatement

\begin{itemize}

    % \item compute the SVD for a rectangular matrix
    \item apply the SVD to estimate the rank and condition number of a matrix
    %     \item construct a basis for the four fundamental spaces of a matrix, and
    %     \item construct a spectral decomposition of a matrix.
    % \end{itemize}
\end{itemize}

} 

\begin{frame} \frametitle{Applications of the SVD}

    The SVD has been applied to many modern applications in CS, engineering, and mathematics. % (our textbook mentions the first four).
    
    \begin{small}
    \begin{itemize}
        \item estimating the rank and condition number of a matrix
        \item constructing bases for the four fundamental spaces
        \item computing the pseudoinverse of a matrix
        \item linear least squares problems
        % \item Non-linear least-squares  https://en.wikipedia.org/wiki/Non-linear\_least\_squares        
        \item machine learning and data mining % \\ https://en.wikipedia.org/wiki/K-SVD
        \item facial recognition % \\ https://en.wikipedia.org/wiki/Eigenface
        \item principle component analysis % \\ https://en.wikipedia.org/wiki/Principal\_component\_analysis
        \item image compression
    \end{itemize}
    \end{small}
    
    % \textit{Students are expected to be familiar with the 1$^{st}$ two items in the list}. 
 
\end{frame}

\begin{frame} \frametitle{The Condition Number of a Matrix}

    If $A$ is an invertible $n\times n$ matrix, the ratio $$\frac{\sigma_1}{\sigma_n}$$ is the \Emph{condition number} of $A$.

    \vspace{12pt} 
    \Emph{Example}: Suppose $A = \spalignmat{2 -1;2 1}$. We found that $\sigma_1 = \sqrt8, \sigma_2 = \sqrt2$. \pause Therefore, the condition number of $A$ is
    
    $$\frac{\sigma_1}{\sigma_n} = \frac{\sqrt8}{\sqrt2}=2$$
    
\end{frame}

\begin{frame} \frametitle{Notes on the Condition Number}


        \begin{itemize}
        \item<2-> In some applications of linear algebra, entries of $A$ and contain errors. The condition number of a matrix describes the sensitivity that any approach to determining solutions to $A\vec x = \vec b$ might have to errors in $A$.
        \item<3-> The larger the condition number, the more sensitive the system is to errors. 
        \item<4-> We could define the condition number for a rectangular matrix, but that would go beyond the scope of this course.
    \end{itemize}

\end{frame}




\frame{\frametitle{Summary}

    \SummaryLine \vspace{4pt}
    \begin{itemize}\setlength{\itemsep}{8pt}

        \item applying the SVD to compute the condition number of a matrix

    \end{itemize}
    
    \vspace{6pt}
}



