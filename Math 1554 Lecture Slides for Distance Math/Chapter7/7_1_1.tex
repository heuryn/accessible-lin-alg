\title{Symmetric Matrices}
\subtitle{\SubTitleName}
\institute[]{\Course}
\author{\Instructor}
\maketitle   
  

\frame{\frametitle{Topics and Objectives}
\Emph{Topics} \\
%\TopicStatement
\begin{itemize}

    \item symmetric matrices
    % \item Orthogonal diagonalization
    % \item Spectral decomposition

\end{itemize}

\vspace{0.5cm}

\Emph{Learning Objectives}\\

%\LearningObjectiveStatement

\begin{itemize}

    \item determine whether a matrix is symmetric
    % \item Construct an orthogonal diagonalization of a symmetric matrix, $A = PDP^T$. 
    % \item Construct a spectral decomposition of a matrix. 
    
\end{itemize}

\vspace{0.25cm} 

%\Emph{Motivating Question}

} 


%% FRAME
\begin{frame} \frametitle{Symmetric Matrices}

    \begin{center}\begin{tikzpicture} \node [mybox](box){\begin{minipage}{0.55\textwidth} \vspace{2pt}
        If matrix $ A = A^T$, then $A$ is \Emph{symmetric}.    
    \end{minipage}};
    \node[fancytitle, right=10pt] at (box.north west) {Definition};
    \end{tikzpicture}\end{center}

    \pause 
    \Emph{Example} Which of the following matrices are symmetric? 
    \vspace{6pt}
    \pause 
    \begin{center}
    \begin{tikzpicture}
        \draw (0,2) node {$ A = ( 2 )$}; 
        \draw (5,2) node {$ B = \begin{pmatrix} 0 & 1 \\ 1 & 0 \end{pmatrix}$}; 
        \draw (0,0) node {$ C = \begin{pmatrix} 1 & 1 \\0 & 0 \end{pmatrix} $};     
        \draw (5,0) node {$ D = \begin{pmatrix} 4 & 2\\0 & 0 \\0 & 0 \end{pmatrix} $};     
    \end{tikzpicture}        
    \end{center}

\end{frame}

\begin{frame}\frametitle{$A^TA$ is Symmetric}
    A very common example: $A^TA$. \\[6pt] \pause For \Emph{any} rectangular matrix $ A $ with columns $ a_1 ,\dotsc, a_n$, 
    \begin{align*}
    A ^{T} A& = 
    \begin{pmatrix}
    \---  & a_1 ^{T} & \--- 
    \\
    \---  & a_2 ^{T} & \--- 
    \\
    \vdots & \vdots & \vdots  
    \\
    \---  & a_n ^{T} & \--- 
    \end{pmatrix}
    \begin{pmatrix}
    \vert & \vert & \cdots & \vert \\ 
    a_1 & a_2 & \cdots & a_n 
    \\ 
    \vert & \vert & \cdots & \vert 
    \end{pmatrix}
    = \underbrace{
    \begin{pmatrix}
    a_1 ^{T} a_1 & a_1 ^{T} a_2 & \cdots & a_1 ^{T} a_n 
    \\
    a_2 ^T a_1 & a_2 ^{T} a_2  &  \cdots & a_2 ^{T}a_n 
    \\
    \vdots   & \vdots & \ddots & \vdots 
    \\
    a_n ^T a_1 & a_n ^{T} a_2  &  \cdots & a_n ^{T}a_n 
    \end{pmatrix} } _{\textup{entries are dot products of columns of $A$}} 
\end{align*}
\pause 
And because $$a_i^T a_j = a_i \cdot a_j = a_j \cdot a_i$$ $A^TA$ is symmetric. 
\end{frame}






%% FRAME
\begin{frame} \frametitle{Example}

    Suppose $A$ and $C$ are $n \times n$ matrices, $\vec x \in \mathbb R^n$, and $C$ is symmetric. Which of the following products are equal to a symmetric matrix? 
    
    \begin{enumerate}
        \item<2-> $AA^T$
        \item<2-> $\vec x \vec x \, ^T$
        \item<2-> $C^2$
    \end{enumerate}
    
    \onslide<3->{\Emph{Solutions}}
    
    \begin{enumerate}
        \item<3-> $(AA^T)^T =  (A^{T})^T A^T = AA^T \quad \Rightarrow \quad \text{symmetric} $
        \item<4-> $(\vec x \vec x \, ^T )^T = (\vec x \, ^T)^T \vec x \, ^T = \vec x \vec x \, ^T \quad \Rightarrow \quad \text{symmetric} $
        \item<5-> $(C^2)^T = (C\,C)^T = C^T C^T = C\,C = C^2 \quad \Rightarrow \quad \text{symmetric}$
    \end{enumerate}
    
\end{frame}






%% FRAME
\begin{frame} \frametitle{Additional Notes on Symmetric Matrices}    

\begin{itemize}
    \item<1-> If a matrix is symmetric, the matrix must be square. 
    \item<2-> If a matrix is square and diagonal, the matrix must be symmetric. 
    \item<3-> Symmetric matrices have other useful properties that we will introduce and take advantage of.
\end{itemize}
\end{frame}




\frame{\frametitle{Summary}

    \SummaryLine \vspace{4pt}
    \begin{itemize}\setlength{\itemsep}{8pt}

        \item symmetric matrices
        \item determining whether a given matrix is symmetric

    \end{itemize}
    
    \vspace{16pt}
    \pause 
    
}





