\title{The SVD of a $3\times2$ Matrix with Rank 1}
\subtitle{\SubTitleName}
\institute[]{\Course}
\author{\Instructor}
\maketitle   


\frame{\frametitle{Topics and Objectives}
\Emph{Topics} \\
%\TopicStatement
\begin{itemize}

    \item the Singular Value Decomposition (SVD) of a matrix and how to compute the SVD for a given matrix % and some of its applications

\end{itemize}

\vspace{0.5cm}

\Emph{Learning Objectives}\\

%\LearningObjectiveStatement

\begin{itemize}

    \item compute the SVD for a rectangular matrix
    % \item Apply the SVD to 
    % \begin{itemize} 
    %     \item estimate the rank and condition number of a matrix, 
    %     \item construct a basis for the four fundamental spaces of a matrix, and
    %     \item construct a spectral decomposition of a matrix.
    % \end{itemize}
\end{itemize}

} 

\begin{frame}\frametitle{A Procedure for Constructing the SVD of $A$}
    Suppose $A$ is $m \times n$ and has rank $r$. 
    
    \begin{enumerate}
        \item<2-> Compute the squared singular values of $A^TA$, $\sigma_i^2$, and construct $\Sigma$. \vspace{6pt}
        \item<3-> Compute the unit singular vectors of $A^TA$, $\vec v_i$, use them to form $V$. \vspace{6pt}
        \item<4-> Compute an orthonormal basis for Col$A$ using $$\vec u_i = \frac{1}{\sigma_i}A\vec v_i, \quad i = 1, 2, \ldots r$$ If necessary, extend the set $\{\vec u_i\}$ to form an orthonormal basis for $\mathbb R^m$ and use the basis to form $U$. 
    
    \end{enumerate}
\end{frame}



\begin{frame}\frametitle{A $3\times2$ Matrix with Rank 1}
Construct the singular value decomposition of $ A = \begin{pmatrix}
1 & -1 \\ -2 & 2 \\ 2 & -2 
\end{pmatrix}$.  

    
    \Emph{Solution}\\
    Singular values: \onslide<3->{
    $$A^TA = \spalignmat{9 -9;-9 9} \quad \Rightarrow \quad \lambda_1 = 18, \ \lambda_2=0 \quad \Rightarrow \quad  \sigma_1 = \sqrt{18}, \ \sigma_2 = 0$$}
    \onslide<4->{Don't forget that }
    \begin{itemize}
        \item<4-> Singular matrices have eigenvalue 0 and the trace of a matrix is the sum of its eigenvalues. 
        \item<5-> The positive square roots of the eigenvalues are the singular values. 
        \item<6-> $\sigma_1$ is the largest singular value.
    \end{itemize}

    
\end{frame}




\begin{frame}\frametitle{Example: Construct $\Sigma$ and $V$}    

    Using the singular values we can construct $\Sigma$. 
    $$\sigma_1 = \sqrt{18} = 3\sqrt{2}, \quad \sigma_2 = 0 \qquad \Rightarrow \qquad \Sigma = \spalignmat{3\sqrt{2} 0;0 0;0 0}$$
    
    \onslide<2->{Next we construct the right-singular vectors $\{\vec v_i\}$ and form $V$. }
    \begin{align*}
        \onslide<3->{A^TA - \lambda_1 I &= \spalignmat{-9 -9;-9 -9} \quad \Rightarrow \quad \vec v_1 = \frac{1}{\sqrt{2}} \spalignmat{1;-1}} \\
        \onslide<4->{A^TA - \lambda_2 I &= \spalignmat{9 -9;-9 9} \quad \Rightarrow \quad \vec v_2 = \frac{1}{\sqrt{2}} \spalignmat{1;1} } \onslide<5->{\quad \Rightarrow \quad
        V  = \frac{1}{\sqrt2}\spalignmat{1 1;-1 1}}
    \end{align*}
\end{frame}




\begin{frame}\frametitle{Construct $\vec u_1$}    
    Next we construct left-singular vectors $\{\vec u_i \}$. The rank of $A$ is $r=1$, so we may use $$\vec u_i = \frac{1}{\sigma_i}A\vec v_i$$ for $i = 1$. \onslide<2->{Vector $\vec u_1$ will be a unit vector in $\mathbb R^3$.}
    \begin{align*}
        \onslide<3->{\vec u_1 = \frac{1}{\sigma_1}A\vec v_1 &= \frac{1}{3\sqrt2}
        \begin{pmatrix}
            1 & -1 \\ -2 & 2 \\ 2 & -2 
        \end{pmatrix}
        \spalignmat{1/\sqrt2;-1/\sqrt2} }\\
        \onslide<4->{&=\frac{1}{3\cdot2}\spalignmat{2;-4;4} = \frac{1}{3} \spalignmat{1;-2;2}}
    \end{align*}
    
\end{frame}




\begin{frame}\frametitle{Constructing $U$}  

    How can we construct the remaining left-singular vectors to construct $U$? 
    \begin{itemize}
        \item<2-> In this example, $A$ has rank $r=1$, and $U$ will be a $3 \times 3$ orthogonal matrix.
        \item<3-> By inspection, two vectors orthogonal to $\vec u_1$ are
        $$\vec x_2 = \spalignmat{2;1;0}, \quad \vec x_3 = \spalignmat{-2;0;1}, \quad \text{recall: } \vec u_1 = \frac{1}{3} \spalignmat{1;-2;2}$$
        \item<4-> Because $\vec u _1 \in \Col A$, these two vectors are in $(\Col A)\Perp$. 
        \item<5-> But $U$ is an orthogonal matrix, so how might we create an orthogonal basis for $(\Col A)\Perp$? 
    \end{itemize}
\end{frame}




\begin{frame}\frametitle{Constructing $U$}  

    Gram-Schmidt. 
    \begin{align*}
        \bar u_2 &= \vec x_2 = \spalignmat{2;1;0}\\
        \bar u_3 &= \vec x_3 - \frac{\vec x_3 \cdot \bar u_2 }{\bar u_2 \cdot \bar u_2}\bar u_2 = \spalignmat{-2;0;1} - \frac{-4}{5}\spalignmat{-2;0;1} = \frac 15 \spalignmat{-2;4;-5}
    \end{align*}
    Normalizing these vectors yields the remaining left singular vectors. 
    \begin{align*}
        \vec u_2 = \frac{1}{\sqrt 5} \spalignmat{2;1;0}, \quad \vec u_3 = \frac{1}{\sqrt{45}}\spalignmat{-2;4;-5}
    \end{align*}
\end{frame}




\begin{frame}\frametitle{Example: Construct $SVD$}  

    Thus, $A = U\Sigma V^T$, where
    \begin{align*}
        U &= \spalignmat{1/3 2/\sqrt5 -2/\sqrt{45};-2/3 1/\sqrt5 4/\sqrt{45}; 2/3 0 -5/\sqrt{45}}\\[4pt]
    \Sigma &= \spalignmat{3\sqrt2 0;0 0;0 0} \\[4pt]
    V &= \frac{1}{\sqrt2}\spalignmat{1 1;-1 1} 
    \end{align*} 
    
    
\end{frame}



\frame{\frametitle{Summary}

    \SummaryLine \vspace{4pt}
    \begin{itemize}\setlength{\itemsep}{8pt}

        % \item the definition of the SVD
        \item computing the SVD for a matrix
        \item applying Gram-Schmidt to construct the SVD of a matrix % form an orthonormal basis for $\mathbb R^m$

    \end{itemize}
    
    \vspace{6pt}
    
    \pause 
}




